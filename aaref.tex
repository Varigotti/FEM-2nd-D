\markboth{Literaturverzeichnis}{Literaturverzeichnis}
\addcontentsline{toc}{chapter}{Literaturverzeichnis}
\begin{thebibliography}{Literaturverzeichnis}{}
\bibitem{Allman} Allman, D.J. (1988) A quadrilateral finite element including vertex rotations
for plane elasticity problem, Int. J. Num. Methods in Engng, Vol.26
\bibitem{Altenbach} Altenbach H, Altenbach J, Naumenko K (1998) Ebene Fl\"{a}chentragwerke.
Springer Verlag
\bibitem{Allman2} Allman DJ (1984) \lqq A compatible triangular element including vertex rotations for plane elasticity analysis\rqq. Computers \& Structures 19, S. 1-8 (1984)
\bibitem{Argyris} Argyris J, Mljenek H-P (1988) Die Methode der finiten Elemente, Band I
Verschiebungsmethode in der Statik, Band II Kraft- und gemischte Methoden,
Nichtlinearit\"{a}ten, Band III Einf\"{u}hrung in die Dynamik, Vieweg
\bibitem{babpit} Babu\v{s}ka I, Pitk\"{a}ranta J (1990) \lqq The Plate
Paradox for Hard and Soft Simple Support\rqq. SIAM Journal for Numerical Analysis 21, 551-576.
\bibitem{Kraetzig0} Ba\c{s}ar Y,  Kr\"{a}tzig W B (1985) Mechanik der Fl\"{a}chentragwerke. Vie\-weg
\bibitem{Barth1} Barth Ch, Petrasch K, Zumpe G (1988)
\lqq Kontrolle der Geometrie- und Vernetzungsparameter im Rahmen einer FE-L\"{o}sung\rqq.
Bauinformation Wissenschaft und Technik, Heft 4,
\bibitem{Barth2} Barth Ch (1993)
\lqq Fehlerabsch\"{a}tzungen von FE-L\"{o}sungen - Stand und Perspektive aus der Sicht der Baupraxis\rqq.
Bauinformatik, Heft 6
\bibitem{Barth3} Barth Ch, Rustler W (2013) Finite Elemente in der Baustatik-Praxis. Bauwerk
\bibitem{Bathe} Bathe K-J (1996) Finite Element Procedures. Prentice Hall
\bibitem{Bathe2} Bathe K.-J, Dvorkin E N (1985) \lqq A Four-Node Plate Bending Element Based on
Mindlin/Reissner Theory and a Mixed Interpolation\rqq. Int. J. Num. Methods Engng. Vol. 21,
367-383
\bibitem{Batoz1} Batoz JL, Bathe K-J, Ho LW (1980) \lqq A Study of Three-Node Triangular Plate Bending Elements\rqq. Int. J. Num. Meth. Engng., Vol 15, No. 12, 1980, 1771-1812
\bibitem{Batoz3} Batoz JL, Katili I (1992) \lqq Simple triangular Reisner-Mindlin plate based on
incompatible modes and discrete contacts\rqq. Int. J. Num. Meth. Engng., Vol. 35, 1992
\bibitem{Bar} Barthold, F-J, Stein E (1997), Elastizit\"{a}tstheorie, in: Der Ingenieurbau Ed. G. Mehlhorn Band Werkstoffe, Elastizit\"{a}tstheorie Ernst \& Sohn Berlin
\bibitem{Baumann1} Baumann Th (1972) \lqq Zur Frage der Netzbewehrung von Fl\"{a}chentragwerken\rqq. Der
Bauingenieur 47 (1972), 367-377, Springer Verlag
\bibitem{Baumann2} Baumann A (2000) Scheibenbemessung unter Anwendung der
Finite-Elemente-Software Sofistik, Diplomarbeit Nr. 369 im Fachgebiet Baustatik, TU
M\"{u}nchen Oktober 2000
\bibitem{Bechert} Bechert H, Furche J (1993) \lqq Bemessung von Elementdecken mit der Methode der
Finiten Elemente\rqq Betonwerk + Fertigteil-Technik, 5, 1993, 47-51
\bibitem{Beer} Beer G, Watson JO (1992) Introduction to Finite and Boundary Element Methods for Engineers, John Wiley  \& Sons
\bibitem{Bellmann} Bellmann J (1987) Hierarchische Finite-Element-Ans\"{a}tze und adaptive
Methoden f\"{u}r Scheiben- und Plattenprobleme, Dissertation, TU M\"{u}nchen 1987, Mitteilungen
aus dem Institut f\"{u}r Bauingenieurwesen I TU M\"{u}nchen Heft 21
\bibitem{Bergan} Bergan, P.G., C.A. Felippa (1985) A Triangular Membrane Element with Rotational
Degrees of Freedom, Computer Methods in Applied Mechanics and Engineering, 50,
25-69
\bibitem{Bernadou} Bernadou M (1996) Finite Element Methods for Thin Shell Problems. John Wiley \&
Sons Chichester, Masson Paris
\bibitem{Betten} Betten J (2004) Finite Elemente f\"{u}r Ingenieure 1. Springer Verlag
\bibitem{Bletzinger2} Bletzinger K-U, Ziegler R (2000) Theoretische Grundlagen der numerischen
Formfindung von Membrantragwerken und Minimalfl\"{a}chen, Betonkalender 2000, Teil II, 441-456, Ed. J. Eibl, Ernst \& Sohn Berlin 2000
\bibitem{Bletzinger} Bletzinger K-U, Bischoff M, Ramm E (2000)
\lqq A Unified Approach for Shear - Locking - Free Triangular and Rectangular Shell Finite
Elements\rqq. Comp. \& Structures (2000), Vol. 73, pp. 321-334.
\bibitem{Brandes} Brandes R  (1977) \lqq M$_{ik}$-Integrale f\"{u}r sich einseitig und f\"{u}r sich beidseitig gradlinig stetig \"{a}ndernde Querschnitte\rqq, Bauingenieur 52 , S 25-26, 1977
\bibitem{Buerg} B\"{u}rg M, Schneider J (1994) \lqq Variability in Professional Design\rqq. Strct.
Engineering Intern. 1994, 247-250
\bibitem{vpi} Bundesvereinigung der Pr\"{u}fingenieure f\"{u}r Bautechnik e.V. (2001) Richtlinie f\"{u}r das Aufstellen und Pr\"{u}fen EDV-unterst\"{u}tzter Standsicherheitsnachweise (Ri-EDV-AP-2001), in: Der Pr\"{u}fingenieur 18, 2001, 49-54,
http://www.bvpi.de
\bibitem{Chen} Chen W-F (200) \lqq Toward practical Advanced Analysis for Steel Frame Design\rqq,
Structural Engineering International 234-239
\bibitem{Cook0} Cook, R.D. (1987) A Plane Hybrid Element with Rotational D.O.F. and Adjustable
Stiffness, Int. J. Num. Meth. Engng., Vol. 24, 8, 1499-1508
\bibitem{Cook1} Cook, R.D., D. S. Malkus, M. E. Plesha (1989) Concepts and Applications of Finite Element Analysis, John Wiley \& Sons New York 3rd Ed.
\bibitem{Czerny} Czerny, F. (1999) Tafeln f\"{u}r Rechteckplatten, Betonkalender 1999 Band I, Ernst \& Sohn, Berlin 1999
\bibitem{Deger} Deger Y (2017) Die Methode der finiten Elmente: Grundlagen und Einsatz in der Praxis. Export-Verlag
\bibitem{Duddeck} Duddeck H (1983) \lqq Die Ingenieuraufgabe, die Realit\"{a}t in ein Berechnungsmodell
zu \"{u}bersetzen \rqq. Die Bautechnik, 1983
\bibitem{Duddeck2} Duddeck H (1989) \lqq Wie konsistent sind unsere Ingenieurmodelle?\rqq. Bauingenieur 64, 1989, 1-8
\bibitem{EAU} EAU (1996) Empfehlungen des Arbeitsausschusses Ufereinfassungen, H\"{a}fen und Wasserstra{\ss}en. Ernst \& Sohn 1996.
\bibitem{Girkmann} Girkmann, K. (1963) Fl\"{a}chentragwerke, Springer Verlag Wien
\bibitem{Gr} Gr\"{a}tsch T (2002) $L_2$-Statik, Dissertation Universit\"{a}t Kassel
\bibitem{Grasser} Grasser E, Thielen G (1991) Hilfsmittel zur Berechnung der Schnittgr\"{o}{\ss}en und Form\"{a}nderungen von Stahlbetontragwerken, Deutscher Ausschu{\ss} f\"{u}r Stahlbeton, Heft 240, 3. Auflage Beuth
\bibitem{Harte} Harte, R. (1982) Doppelt gekr\"{u}mmte finite Dreieckelemente f\"{u}r die lineare und
geometrisch nichtlineare Berechnung allgemeiner Fl\"{a}chentragwerke,
Technisch-wissenschaftliche Mitteilungen des Instituts f\"{u}r konstruktiven Ingenieurbau Nr.
82-10, Ruhr-Universit\"{a}t Bochum
\bibitem{Ha1} Hartmann, F (1985) The Mathematical Foundation of Structural Mechanics. Springer Verlag
\bibitem{Ha2} Hartmann, F (1987) Methode der Randelemente. Springer Verlag
\bibitem{Ha3} Hartmann F (1989) Introduction to Boundary Elements. Springer Verlag
\bibitem{Ha4} Hartmann, F, Pickhardt S (1985) \lqq Der Fehler bei finiten Elementen\rqq. Bauingenieur, 60, 1985, 463-468
\bibitem{Ha8} Hartmann F, Katz C (2000) Statik mit finiten Elementen, Springer Verlag
\bibitem{Ha5} Hartmann F, Katz C (2010) Structural Analysis with Finite Elements, 2nd ed. Springer Verlag
\bibitem{Ha6} Hartmann F (2013) Green's Functions and Finite Elements. Springer Verlag
\bibitem{HaJa} Hartmann F, Jahn P (2014) \lqq Steifigkeits\"{a}nderungen bei finiten Elementen\rqq, Bau\-ingenieur 89, 209-215
\bibitem{Ha7} Hartman F, Jahn P (2017) Statics and Influence Functions---From a Modern Perspective. Springer Verlag
\bibitem{HaJa2} Hartman F, Jahn P (2018) Statik und Einflussfunktionen---vom modernen Standpunkt.http://nbn-resolving.de/urn:nbn:de:hebis:34-2018030554714
\bibitem{Heyman} Heyman J (1969) \lqq Hambly's paradox: why design calculations do not reflect real
behaviour\rqq. Proc. Inst. Civil Engng. 114 161-166
\bibitem{Hinton} Hinton E (1990) Finite Elemente Programme f\"{u}r Platten und Schalen. Springer Verlag
\bibitem{Hobst} Hobst E (2000) \lqq Methode der finiten Elemente im Stahlbetonbau
Randbedingungen und Singularit\"{a}ten - wie genau ist die Finite-Elemente-Methode?\rqq. Beton
und Stahlbetonbau, Heft 10, 2000
\bibitem{Holzer} Holzer S (1997) \lqq Gestaltung ingenieurgem\"{a}{\ss}er Statiksoftware\rqq. Bauingenieur
72, 103-110, 1997
\bibitem{Hughes} Hughes, T.J.R. (1987) The Finite Element Method, Prentice-Hall, Englewood Cliffs, New Jersey 1987
\bibitem{Hughes2} Hughes TJR (1981) \lqq Finite Elements Based Upon Mindlin Plate Theory With
Particular Reference to the Four-Node Bilinear Isoparametric Element\rqq. Journal of Applied
Mechanics, Vol. 48, 587-596, 1981
\bibitem{Hughes3} Hughes TJR, Evans JA (2010) Isogeometric Analysis.
The Institute for Computational Engineering and Sciences Report 10-18
The University of Texas, Austin (2010)
\bibitem{Hughes4} Hughes TJR, Brezzi F (1989) \lqq On drilling degrees of Freedom\rqq
Comp.Meth.Appl.Mechanics and Engineering 72, S. 105-121 (1989)
\bibitem{Jeyachandrabose} Jeyachandrabose C, Kirkhope J (1985) \lqq An Alternative Formulation for the DKT Plate
Bending Element\rqq. Int. J. Num. Meth. Engng., Vol 21, No.7, 1985, 1289-1293
\bibitem{Jung} Jung M, Langer U (2013) Methode der finiten Elemente f\"{u}r Ingenieure. Springer Vieweg
\bibitem{Katz1} Katz C, Stieda J (1993) \lqq Praktische FE-Berechnung mit Plattenbalken\rqq. Bauinformatik 1/92 30-34
\bibitem{Kany} Gra{\ss}hoff, H., M. Kany (1997) Berechnung von
Fl\"{a}chengr\"{u}ndungen, in: Grundbau-Taschenbuch, Teil 3, 5. Aufl. Ed. U. Smoltczyk, Ernst \&
Sohn, Berlin 1997
\bibitem{Katz2} Katz C, Werner H (1982) \lqq Implementation of nonlinear boundary conditions in
finite element analysis\rqq. Computers \& Structures Vol. 15, No. 3, 299-304, 1982
\bibitem{Katz3} Katz C (1995) \lqq Kann die FE-Methode wirklich alles?\rqq
    FEM 95 - Finite Elemente in der Baupraxis, Ed. E. Ramm, E. Stein, W. Wunderlich
    Ernst \& Sohn, Berlin
\bibitem{Katz4} Katz C (1986) \lqq Berechnung von allgemeinen Pfahlwerken\rqq. Bauingenieur 61 Heft 12, (1986)
\bibitem{Katz5} Katz C (1997) \lqq Flie{\ss}zonentheorie mit Interaktion aller Stabschnittgr\"{o}{\ss}en bei Stahltragwerken\rqq.  Stahlbau 66, Heft 4, pp. 205-213
\bibitem{Katz6} Katz C (1996) \lqq  Vertrauen ist gut, Kontrolle ist besser\rqq, in:
Software f\"{u}r Statik und Konstruktion, Eds. C. Katz, B. Protopsaltis, A.A. Balkema
Rotterdam, (1996)
\bibitem{Katz7} Katz C (1997) Flie{\ss}zonentheorie mit Interaktion aller Schnittgr\"{o}{\ss}en bei
Stahltragwerken, Stahlbau 66, Heft 4, 205-213
\bibitem{Katz8} Katz C (2013)  \lqq Software f\"{u}r Stahlbauer � M\"{o}glichkeiten und Module\rqq
Stahlbauseminar der FH Biberach
\bibitem{Ramm3} Kemmler R, Ramm E (2001) \lqq Modellierung mit der Methode der Finiten Elemente\rqq. Betonkalender 2001
Ernst \& Sohn, Berlin 2001, 381-446
\bibitem{Kiener1} Kiener G, Henke P (1983) \lqq Die Anwendung der Methode der Variation der Konstanten in der Baustatik\rqq. Bauingenieur 58 429-436, (1983)
\bibitem{Kiener2} Kiener G., Henke P (1988) \"{U}bertragungsmatrizen, Lastvektoren, Steifigkeitsmatrizen und Volleinspannschnittgr\"{o}{\ss}en einer Gruppe konischer St\"{a}be mit linear ver\"{a}nderlichem
Querschnitt\rqq  Bauingenieur 63, 567-574,  (1988)
\bibitem{Kindmann} Kindmann R, Kraus M (2007) Finite-Elemente-Methoden im Stahlbau. Ernst \& Sohn
\bibitem{Kindmann2} Kindmannm R, Kr\"{u}ger U (2015) Stahlbau, Teil 1: Grundlagen, Ernst \& Sohn
\bibitem{Klein} Klein B (2014) FEM, Grundlagen und Anwendungen der Finite-Elemente-Methode. Vieweg
\bibitem{Knothe}  Knothe K, Wessels H (2017) Finite Elemente. Springer Verlag
\bibitem{Kraus} Kraus M (2005) Computerorientierte Berechnungsmethoden f\"{u}r beliebige Stabquerschnitte des Stahlbaus. Dissertation, Universit\"{a}t Dortmund
\bibitem{Kurrer} Kurrer K-E (2018) The History of the Theory of Structures. Wiley Ernst \& Sohn
\bibitem{Koenig} K\"{o}nig G., Tue N. (1998) Grundlagen des Stahlbetonbaus. B. G. Teubner
\bibitem{Leonhardt} Leonhardt F, M\"{o}nnig E (1974) Vorlesungen \"{u}ber Massivbau, zweiter Teil, Sonderf\"{a}lle der Bemessung im Stahlbetonbau. Springer Verlag
\bibitem{Lesaint} Lesaint P (1985) \lqq On the convergence of Wilson's nonconforming element for
solving elastic problems\rqq. Comp. Meth. Appl. Mech. Engng., 7, 1
\bibitem{Link} Link M (2014) Finite {E}lemente in der {S}tatik und {D}ynamik. B.G. Teubner-Verlag, 4. Auflage
\bibitem{Lumpe} Lumpe G, Gensichen V (2014) Evaluierung der linearen und nichtlinearen Stabstatik in Theorie und Software. Ernst \& Sohn
\bibitem{MacNeal} MacNeal, R.H. (1994) Finite elements: their design and performance, Dekker
 New York
\bibitem{Mehlhorn3} Mehlhorn G (1995) \lqq Grundlagen zur physikalischen nichtlinearen FEM-Berechnung von Tragwerken aus Konstruktionsbeton. Materialmodelle f\"{u}r Bewehrung und Beton\rqq. Bauingenieur 70 (1995), S.313-320
\bibitem{Mehlhorn1} Mehlhorn G, Kollegger J (1995) \lqq Anwendung der Finite Elemente Methode im Stahlbetonbau\rqq, in: Der Ingenieurbau Grundwissen (Ed.: G. Mehlhorn), Band Rechnerorientierte Baumechanik, Ernst \& Sohn, Berlin, 1995.
\bibitem{Mehlhorn2} Mehlhorn G (1996) \lqq Grundlagen zur physikalisch nichtlinearen FEM-Berechnung von Tragwerken aus bewehrtem
Konstruktionsbeton. Verbund zwischen Beton und Bewehrung und Modellierung von bewehrtem Konstruktionsbeton\rqq. Bauingenieur 71 (1996), S.187-193
\bibitem{Meissner} Meissner U, Maurial A (2009) Die Methode der finiten Elemente: Eine Einf\"{u}hrung in die Grundlagen. Springer
\bibitem{Melzer} Melzer H, Rannacher R (1980) \lqq Spannungskonzentrationen in
     Eckpunkten der Kirch\-hoff\-schen Platte\rqq. Bauingenieur 55 (1980) 181-184
\bibitem{Merkel} Merkel M, \"{O}chsner A (2015) Eindimensionale Finite Elemente: Ein Einstieg in die Method. Springer Vieweg
\bibitem{Nasitta} Nasitta K, Hagel H (1992) Finite Elemente, Mechanik, Physik und nichtlineare Prozesse Springer Verlag
\bibitem{Osterrieder} Osterrieder P (2005) \lqq Plastic bending and torsion of open thin-walled steel members\rqq.  Proceedings of EUROSTEEL 2005, 4th European Conference on Steel and Composite Structures, Maastricht
\bibitem{Petersen1} Petersen C (1980) Statik und Stabilit\"{a}t der Baukonstruktionen. Vieweg
\bibitem{Petersen2} Petersen C, Werkle H (2018) Dynamik der Baukonstruktionen. 2. Aufl. Springer Vieweg
\bibitem{Pickhardt} Pickhardt S (1987) Fehlerabsch\"{a}tzungen bei ausgesuchten finiten
Elementen. Dissertation Universit\"{a}t Dortmund
\bibitem{Pilkey} Pilkey WD, Wunderlich W (1994) Mechanics of Structures, Variational and Computational Methods, CRC Press
\bibitem{Pimpinelli} Pimpinelli G (2004) \lqq An assumed strain quadrilateral element with drilling degrees of freedom\rqq, Finite Elements in Analysis and Design 41, S. 267-283
\bibitem{Pit} Pitkaeranta, J., A.-M. Matache, C. Schwab, (1999) Fourier mode analysis of layers in shallow shell  deformations, Research Report 1999 ETH Seminar for Applied Mathematics, Z\"{u}rich
\bibitem{Ramm1} Ramm E, M\"{u}ller L (1989) \lqq Flachdecken und Finite Elemente - Einfluss des Rechenmodells im
St\"{u}tzenbereich\lqq, in: Finite Elemente - Anwendungen in der Baupraxis, Ernst \& Sohn,
Berlin 1989
\bibitem{Ramm2} Ramm E, Hofmann TJ (1995) \lqq Stabtragwerke\rqq, in: Der Ingenieurbau, Ed. G. Mehlhorn, Band Baustatik Baudynamik, Ernst \& Sohn, Berlin 1995
\bibitem{Ramm4} Ramm, E., N. Fleischmann, A. Burmeister, (1993) Modellierung mit Faltwerkselementen,
    Baustatik Baupraxis, Tagung Universt\"{a}t M\"{u}nchen, 1993
\bibitem{Rank} Rank E, Ro{\ss}mann A (1987) \lqq Fehlersch\"{a}tzung und automatische {N}etzanpassung bei {F}inite-{E}lement-{B}erechnungen\rqq, Bauingenieur, 62, (1987), 449-454
\bibitem{Rank3} Rank E, R\"{u}cker E, Schweingruber M (1994) \lqq Automatische Generierung von Finite-Element-Netzen\rqq. Bauingenieur, 69, 373 � 379, 1994
\bibitem{Rank4} Rank E, Halfmann A, R\"{u}cker M, Katz C, Gebhard S, (2000) \lqq Integrierte Modellierungs- und Berechnungssoftware f\"{u}r den konstruktiven Ingenieurbau: Systemarchitektur und Netzgenerierung\rqq. Bauingenieur, 75, 60 - 66, 2000
\bibitem{Rank5} Rank E, Br\"{o}ker H, D\"{u}ster A, R\"{u}cker M (2001) \lqq Integrierte Modellierungs- und Berechnungssoftware f\"{u}r den konstruktiven Ingenieurbau: Die p-Version und geometrische Elemente\rqq. Bauingenieur 76, 53 - 61, 2001
\bibitem{Rasmussen} Rasmussen Kim JR, Zhang H, Sena Cardoso F (2016) \lqq The next generation of design specifications for steel structures\rqq, SEMC 2016, A. Zingoni ed.
\bibitem{Rieg} Rieg F, Hackenschmidt R (2014) Finite Elemente Analyse f\"{u}r Ingenieure.
Hanser-Verlag
\bibitem{Roessle} R\"{o}ssle, A., A.-M. S\"{a}ndig, (2001) Corner singularities
and regularity results for the Reissner/Mindlin plate model. Preprint 01/04 des
Sonderforschungsbereiches 404 \hlq Mehrfeldprobleme in der Kontinuumsmechanik\grq\, an
der Universit\"{a}t Stuttgart, 2001.
\bibitem{Roessle2}  R\"{o}ssle A (2000) \lqq Corner Singularities and Regularity of Weak Solutions for
the Two-Dimensional Lam\'{e} Equations on Domains with Angular Corners\rqq. Journal of
Elasticity, 60, 57-75, 2000
\bibitem{Rombach} Rombach G (2016) Anwendung der Finite-Elemente-Methode im Betonbau, Ernst \& Sohn,
Berlin, 2. Auflage
\bibitem{Rubin} Rubin H (1993) Baustatik ebener Stabwerke, in Stahlbau-Handbuch Teil A. Stahlbau-Verlagsgesellschaft
\bibitem{Ruesch} R\"{u}sch H, Hergenr\"{o}der A (1969) Einflussfelder der Momente schiefwinkliger Platten. Werner-Verlag 3. Auflage
\bibitem{Schade} Schade D (1987) \lqq Zur Berechnung von Querschnittswerten und Spannungsverteilungen f\"{u}r Torsion und Profilverformungen von prismatischen St\"{a}ben mit d\"{u}nnwandigen Querschnitten\rqq
Z. Flugwiss.Weltraumforschung 11 , 167-173
\bibitem{Schiefer} Schiefer S, Fuchs M, Brandt B, Maggauer G, Egerer A, (2006) \lqq Besonderheiten beim Entwurf semi-integraler Spannbetonbr\"{u}cken\rqq, Beton und Stahlbetonbau, Oktober 2006, 790-802
\bibitem{Schier} Schier K (2010) Finite Elemente Modelle der Statik  und Festigkeitslehre: 101 Anwendungsf\"{a}lle zur Modellbildung. Springer
\bibitem{Schikora1} Schikora K, Eierle B (1998) \lqq Ebene und r\"{a}umliche
Finite-Element-Berechnungen\rqq, in: Software f\"{u}r Statik und Konstruktion, Eds. C. Katz, B.
Protopsaltis, A.A. Balkema, Rotterdam Brookfield 1998
\bibitem{Schikora3} Schikora K, Fink T (1982) \lqq Berechnungsmethoden moderner bergm\"{a}nnischer Bauweisen
beim U-Bahn-Bau\rqq. Bauingenieur, 57, 193-198, 1982
\bibitem{Schmoll} Schmoll J, Uhrig R, (1990) \lqq Zum Tragverhalten von im Grundri{\ss} gekr\"{u}mmten
Stabtragwerken\rqq, Bautechnik 97, 73-76, (1990)
\bibitem{Schroeter} Schroeter H (2000) \lqq Die Kunst des Rechenmodells oder wie wird aus der 6B-Skizze ein
Bewehrungsplan\lqq, Vortrag vor der Bauakademie Biberach, 2000
\bibitem{Schroeter2} Schroeter, H. (1980) Berechnung idealer Kipplasten von Tr\"{a}gern linear
ver\"{a}nderlicher H\"{o}he mit Hilfe Hermite'scher Polynome, Mitteilungen aus dem Institut f\"{u}r
Bauingenieurwesen I, TU M\"{u}nchen Heft 5, (1980)
\bibitem{Schuetz} Sch\"{u}tz K, Sch\"{u}ren P (1990) \lqq Statische und
dynamische Berechnung ebener gekr\"{u}mmter Durchlauftr\"{a}ger\rqq. Bautechnik 97, 77-84, (1990)
\bibitem{Schwarz}  Schwarz HR, (1991) Methode der finiten Elemente. B.G. Teubner-Verlag
\bibitem{Sofistik} Sofistik AG (1999) Statik-Anwenderbrief Nr. 20, Dezember 1999, Sofistik AG,
\bibitem{Sopoth} Sopoth M, Sopoth G (2008) Sensitivit\"{a}tsanalyse an einem Br\"{u}ckenbauwerk in semi-integraler Bauweise. Diplomarbeit Universit\"{a}t Kassel
\bibitem{Stein1} Spierig S, Stein E (1999) Technische Mechanik, in: Der Ingenieurbau, Ed. G. Mehlhorn,
Band Mathematik Technische Mechanik, Ernst \& Sohn, Berlin 1999
\bibitem{Stein3}  Stein E, Ohnimus S, Seifert B, Mahnken R (1994)
\lqq Adaptive {F}inite-{E}lement {D}iskretisierungen von {F}l{"a}chentragwerken\rqq. Bauingenieur, 69, 1, (1994),53-62
\bibitem{Stein4} Stein E, Ohnimus ES (1994)
\lqq Integrierte {L"o}sungs- und {M}odelladaptivit{"a}t f{"u}r die {F}inite-{E}lemente
{M}ethode von {F}l{"a}chentragwerken\rqq, Bauingenieur, 73, 11, (1998), 473-483
\bibitem{Steinbuch} Steinbuch R (2008) Finite {E}lemente -- {E}in {E}instieg,
Springer Verlag
\bibitem{Steinke} Steinke P (2015) Finite-Elemente-Methode: Rechnergest\"{u}tzte Einf\"{u}hrung. Springer Vieweg
\bibitem{Stempniewski} Stempniewski L, Eibl J (1996) \lqq Finite Elemente im Stahlbeton\rqq,
Betonkalender 1996, Teil II, Ed. J. Eibl, Ernst \& Sohn Berlin 1996
\bibitem{Stiglat} Stiglat K, Wippel H (2000) \lqq Massive Platten\rqq, in: Betonkalender 2000, Teil
II, 211 - 290, (Ed. Eibl), Ernst \& Sohn Berlin 2000
\bibitem{StiglatWippel} Stiglat K, Wippel H (1968) Platten. Ernst \& Sohn
\bibitem{Szabo} Szab\'{o}, B.,  I. Babu\v{s}ka (1991) Finite Element Analysis John Wiley \& Sons, Inc. New York 1991
\bibitem{Thieme} Thieme D (1996) Einf\"{u}hrung in die Finite-Elemente-Methode f\"{u}r Bauingenieure, 2. Auflage, Verlag f\"{u}r Bauwesen,
\bibitem{Turner} Turner MJ, Clough RW, Martin HC, Topp LJ (1956) \lqq Stiffness and deflection analysis of complex structures\rqq. Journal of the Aeronautical Sciences, Vol. 23, No. 9, 805-823. Auszug in \cite{Kurrer} S. 882-883
\bibitem{VDI} VDI (2014/17) VDI Richtlinie 6201, Softwaregest\"{u}tzte Tragwerksberechnung
Teil 1 -  Grundlagen, Anforderungen, Modellbildungen, 2014
Teil 2 � Verifikationsbeispiele, 2017
\bibitem{Wagenknech} Wagenknecht G (2018) Baustatik - Weggr\"{o}{\ss}enverfahren: Grundlagen - Finite Elemente der Stabstatik. Beuth Verlag
\bibitem{Werkle} Werkle H (2008) Finite Elemente in der Baustatik. Springer Vieweg. 3. Auflage
\bibitem{Werkle1} Werkle H (2000) \lqq Konsistente Modellierung von St\"{u}tzen bei der Finite-Element-Berechnung von Flachdecken\rqq, Bautechnik 77 (2000) 416--425
\bibitem{Werkle2} Werkle H (2000) \lqq Konsistente Modellierung von St\"{u}tzen bei der Finite-Elemente-Berechnung von Flachdecken\rqq. Bautechnik, 6, 2000
\bibitem{Werkle3} Werkle H (2006) Vorlesung Baustatik III, Skriptum
\bibitem{Wiki1} https://de.wikipedia.org/wiki/Prinzip\_von\_St.\_Venant
\bibitem{Williams} Williams ML (1952) \lqq Stress singularities resulting from various boundary conditions in angular corners of plates in extension\rqq, Jounal of Applied Mechanics, 12, 526-528, 1952
\bibitem{Wilson} Wilson, E., R.L. Taylor, W.P. Doherty, J.
Ghaboussi, (1971) \lqq Incompatible displacement models\rqq. Symposium on Numerical Methods, University
of Illinois 1971
\bibitem{Wriggers} Wriggers P (2001), Nichtlineare Finite-Element-Methoden, Springer Verlag
\bibitem{Wu1} Wunderlich W, Kiener G, Ostermann W (1994) \lqq Modellierung und Berechnung von Deckenplatten mit
Unterz\"{u}gen\rqq. Bauingenieur 69 (1994) 381-390
\bibitem{Wu2} Wunderlich W (1996) Die Methode der Finiten Elemente, in: Der Ingenieurbau, Ed. G. Mehlhorn, Band
Rechnerorientierte Baumechanik, Ernst \& Sohn, Berlin
\bibitem{Yuan} Yuan M, Sun S, Chen P (1998) \lqq Applications of drilling degrees of freedom for 2D and 3D structural analysis\rqq, Idelsohn, Onate Dvorkin, Computational Mechanics
\bibitem{Z1} Zienkiewicz OC, Taylor RL (1994) The Finite Element
Method, 4. Ed. McGraw-Hill
\bibitem{Zimmermann} Zimmermann S (1989) \lqq Parameterstudie an Platte mit Unterzug\rqq,
	1. FEM Tagung Kaiserslautern

\end{thebibliography}
