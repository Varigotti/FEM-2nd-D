
\begin{flushleft}\large{\bf{Zweite Auflage}} \end{flushleft}

Der deutschen Auflage folgte bald die englische Auflage und drei Jahre sp\"{a}ter die zweite englische Auflage des Buchs. Die zweite Auflage des deutschen Originals war also \"{u}berf\"{a}llig -- erst recht nachdem heute der Druck in Farbe m\"{o}glich ist.

In der zweiten Auflage haben wir versucht das Gute beizubehalten und gleichzeitig Raum f\"{u}r die Darstellung der neueren Entwicklungen zu schaffen. Der Schwerpunkt der Darstellung liegt, wie schon in der ersten Auflage, auf der Anwendung der finiten Elemente in der Tragwerksplanung, denn das Buch will dem Tragwerksplaner das f\"{u}r einen erfolgreichen Einsatz der finiten Elemente notwendige und unerl\"{a}ssliche Verst\"{a}ndnis an Beispielen vermitteln.

Was das Thema so reizvoll macht ist zu sehen, wie sich die finiten Elemente als Jungbrunnen f\"{u}r die klassische Statik erwiesen haben, denn das Thema Einflussfunktionen zieht sich wie ein roter Faden durch die finiten Elemente und erweist sich als Schl\"{u}ssel zum Verst\"{a}ndnis der finiten Elementen. Statik ist Kinematik -- unsere Tragwerke sind im Grunde sorgf\"{a}ltig ausbalancierte Getriebe -- und die finiten Elemente sind die Seele des Ganzen, weil sie Bewegung in die Tragwerke bringen.

\begin{flushright}\noindent
Kassel/M\"{u}nchen  {\hfill {\it Friedel Hartmann, Casimir Katz}}\\\vspace{0.1cm}
Dezember 2018   {\hfill {hartmann@be-statik.de, Casimir.Katz@sofistik.de}}\\
\end{flushright}



\begin{acknowledgement}
Her Kollege Werkle, Hochschule Konstanz, hat uns bei der korrekten Formulierung der {\em \"{A}quivalenten Spannungs Tranformation\/}, Abschnitt 4.6, tatkr\"{a}ftig unterst\"{u}tzt. Daf\"{u}r sei ihm an dieser Stelle gedankt.\\
\end{acknowledgement}


