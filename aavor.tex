\chapter*{Vorwort}

\markboth{Vorwort}{Vorwort} Die Methode der finiten Elemente ist heute aus den
technischen B\"{u}ros und Ingenieurb\"{u}ros nicht mehr wegzudenken. Insofern hat sie eine
beispiellose Erfolgsgeschichte hinter sich und wohl auch noch ein gutes St\"{u}ck vor sich.
Pl\"{o}tzlich war es m\"{o}glich, Tragwerke zu analysieren, die vorher einer Berechnung nicht
zug\"{a}nglich waren.

Mit dem Erfolg kam aber auch die Kritik, denn es wurde sp\"{u}rbar, dass die Kenntnis
der Grundlagen nicht im gleichen Sinne mitgewachsen ist, ja Pr\"{u}fingenieure klagen, dass
die Methode zunehmend unkritisch eingesetzt wird, ohne dass die Ergebnisse in
irgendeiner Form hinterfragt werden. Oft fehlt es an dem einfachsten Verst\"{a}ndnis f\"{u}r die
Grenzen und M\"{o}glichkeiten der Methode.

Das ist eigentlich schade, denn hinter den finiten Elementen steckt, so merkw\"{u}rdig das
jetzt hier klingen mag, \hlq richtige\grq, klassische Statik. Die Methode der finiten
Elemente bedeutet mehr, als dass man ein Tragwerk in kleine Elemente zerlegt, sie in
den Knoten verbindet und die Belastung durch Knotenkr\"{a}fte ersetzt. Das ist ein beliebtes
Modell, aber dieses Modell verk\"{u}rzt die statische Wirklichkeit in einem solchen Ma{\ss}e,
dass es schon wieder irref\"{u}hrend ist. Zu oft wird vergessen, dass dieses Modell nur ein
Modell 'als ob' ist.

Die Unkenntnis der Grundlagen ist um so mehr zu bedauern, als die Idee hinter den
finiten Elementen eigentlich sehr einfach ist. Es ist das klassische Prinzip der
virtuellen Verr\"{u}ckungen: So wie ein Waagebalken ins Gleichgewicht kommt, wenn wir die
eine Last mit der anderen Last aufwiegen, so ist es mit den finiten Elementen: In der
linken Waagschale liegt, im \"{u}bertragenen Sinn, die urspr\"{u}ngliche Belastung $p$, und wir
legen in die rechte Waagschale eine Ersatzlast $p_h$ so, dass bei jeder Drehung des
Waagebalkens beide Lasten die gleiche Arbeit leisten. Das ist -- nat\"{u}rlich etwas
verk\"{u}rzt -- die Methode der finiten Elemente.

Ein Verst\"{a}ndnis f\"{u}r die Grundlagen der finiten Elemente ist Voraussetzung daf\"{u}r, dass man
FE-Programme sinnvoll einsetzen kann, dass man Ergebnisse bewerten kann, denn erst aus dem Wissen
um die Grundlagen kommt die n\"{o}tige Souver\"{a}nit\"{a}t und Gelassenheit im Umgang mit
FE-Programmen.

Das Ziel des Buches war es daher, die Grundlagen der finiten Elemente in einer an die
Vorstellungswelt des Ingenieurs angepassten Sprache darzustellen, sie so
aufzubereiten, dass die Statik hinter den finiten Elementen sichtbar wird. Dabei kam es
uns vor allem darauf an, die Ideen zu vermitteln. Sie waren wichtig und nicht unbedingt
die technischen Details, denn die Statik sollte im Vordergrund stehen und nicht das
Programmieren der Elemente. Und so haben wir auch viel Wert auf illustrative Beispiele gelegt.

Die finiten Elemente wurden nicht von Mathematikern sondern von Ingenieuren erfunden
(Argyris, Clough, Zienkiewicz). In der Tradition der mittelalterlichen Baumeister wurden
Elemente ersonnen und ausprobiert, ohne dass man die genauen Hintergr\"{u}nde gekannt h\"{a}tte.
Die Ergebnisse waren empirisch brauchbar, und man war dankbar, dass man \"{u}berhaupt
Antworten auf Fragen erhielt, die vorher unl\"{o}sbar waren. In einem zweiten Zeitalter, das
man barock nennen k\"{o}nnte, wurden dann immer komplexere Elemente entwickelt, ein
FE-Programm bot dem Benutzer 50 oder mehr verschiedene Elemente an. Die dritte Phase,
die der Aufkl\"{a}rung, begann damit, dass sich die Mathematiker mit der Methode
besch\"{a}ftigten und versuchten, die Hintergr\"{u}nde der Methode zu finden. In Teilen war ihr
Bem\"{u}hen vergeblich oder extrem schwierig, da die Ingenieure in der Vergangenheit \hlq
Kunstgriffe\grq\, angewandt hatten (reduzierte Integration, nichtkonforme Elemente,
diskrete Kirchhoff-Elemente), die nicht in das Muster der Voraussetzungen passten. Hinzu
kam das Problem, dass manche mathematischen Gesetze (z.B. das Maximumsprinzip) auf
diskrete Systeme angepasst werden mussten. Zug um Zug wuchs aber die Erkenntnis, und
heute haben zumindest Mathematiker kein schlechtes Gef\"{u}hl mehr, wenn sie sich mit den
Grundlagen der finiten Elemente auseinandersetzen, ja es gelingt sogar jetzt zutreffende
Voraussagen \"{u}ber die Eigenschaften eines Elements zu machen, das man analysiert hat.
Insofern muss man anerkennen, dass zu den Grundlagen der finiten Elemente auch
mathematische Aspekte geh\"{o}ren.

Die Autoren dieses Buches sind beide Ingenieure mit mathematischem Hintergrund, die
zwischen der Welt der angewandten Mathematik und der Praxis stehen. Mit ein Ziel dieses
Buches ist es, dem Praktiker die mathematischen Grundlagen in einer Form nahe zu
bringen, die es ihm erlaubt, ohne detaillierte Kenntnisse der Mathematik vertiefte Einsicht in die
Grundlagen der Methode zu erwerben. Wenn es die letztlich unersetzliche kritische
Besch\"{a}ftigung mit den Ergebnissen der FE-Methode unterst\"{u}tzt, hat es seinen Zweck
erreicht.

\vskip 0.2cm \noindent Im Mai 2001  \hfill {\it Friedel Hartmann, Casimir Katz}\\

\begin{acknowledgement}
Die Kolleginnen und Kollegen
Dipl.-Ing. Baumann, Prof. Dr.-Ing. Barth, Dr.-Ing. Bellmann, Dipl.-Ing. Filus, Dipl.-Ing. Gr\"{a}tsch,
Prof. Dr.-Ing. Holzer, Dr.rer.nat Dr.-Ing. Jahn, Dipl.-Ing. Kemmler, Dr.-Ing. Kimmich, Prof. Dr.-Ing. Pauli, Dr.-Ing. Pflanz,
Prof. Dr.-Ing. Ramm, Prof. Dr.-Ing. Schikora, Prof. Dr.-Ing. Schnellenbach-Held,
Dr.-Ing. Schroeter, Dipl.-Ing. v. Spiess
haben uns tatkr\"{a}ftig unterst\"{u}tzt. Ihnen gilt unser besonderer Dank.\\
\end{acknowledgement}
