Das gilt auch f\"{u}r Funktionen. Die {\em Fourierreihe\/} $f_n(x)$ einer Funktion $f(x)$ ist die Projektion
von $f(x)$ auf die trigonometrischen Funktionen im Sinne des $L_2$-Skalarproduktes und
die L\"{a}nge (= $L_2$-Norm) der Fourierreihe $f_n$ ist gem\"{a}{\ss} der \index{Besselsche
Ungleichung}{\em Besselschen Ungleichung\/} immer kleiner gleich der L\"{a}nge des Originals
\bfoo
||f_n||_{\,0} = [\int_0^{\,l} f_n^2(x)\,dx ]^{1/2} \leq [\int_0^{\,l} f^2(x)\,dx ]^{1/2}
= ||f||_{\,0}\,.
\efoo

\hypersetup{pdftitle={Statik mit finiten Elementen},
pdfsubject = {Statik, Stabtragwerke, Fl\"{a}chentragwerke, finite Elemente},
pdfauthor = {Friedel Hartmann, Casimir Katz},
pdfkeywords = {Statik, Einflussfunktionen, Satz von Betti, Maxwell, finite Elemente, Randelemente, Energieprinzipe, Variationsprinzipe, Prinzip der virtuellen Verr\"{u}ckung, Prinzip der virtuellen Arbeit, Potentielle Energie, Gleichgewicht, Greensche Identit\"{a}ten, Dualit\"{a}t, Stabtragwerke, Fl\"{a}chentragwerke}}
%%%%%%%%%%%%%%%%%%%%%%%%%%%%%%%%%%%%%%%%%%%%%%%%
Aber Vorsicht! Aus diesen Beobachtungen folgt nicht, dass die FE-Durchbie\-gung\-en immer
kleiner sind, als die echten Durchbiegungen einer Platte. Das kann f\"{u}r einzelne Knoten
richtig sein, wird aber nicht f\"{u}r alle Knoten gelten.

Es gibt nur ein Beispiel, wo der Schluss -- wenigstens f\"{u}r einen Knoten -- richtig ist
und zwar, wenn eine Einzelkraft $P$ an der Stelle $\vek x_P$ auf einer schubstarren
Platte steht. Die potentielle Energie in der Gleichgewichtslage ist dann gerade die
(negative) Eigenarbeit der Kraft $P$
\bfoo
 -\frac{1}{2} P \, w(\vek x_P) = \Pi(w) < \Pi(w_{h}) = -\frac{1}{2} P \, w_{h}(\vek x_P)\,.
\efoo
Diese Ungleichung kann nur richtig sein, wenn die FE-Durchbiegung im Angriffspunkt der
Einzelkraft wirklich kleiner als die wahre Durchbiegung ist, $ w_{h} < w $.
%----------------------------------------------------------------------------------------------------------
\begin{figure}[tbp]
\if \bild 2 \sidecaption \fi
\includegraphics[width=.8\textwidth]{\Fpath/BALKENKRAFT}
\caption{Je mehr Lager vorhanden sind, um so kleiner wird die potentielle Energie $\Pi$,
um so kleiner wird die Durchbiegung $w$ in der Mitte des Tr\"{a}gers, und um so kleiner wird
der Umfang des Verformungsraums $\mathcal{V}$} \label{BalkenKraft}
\end{figure}%%
%----------------------------------------------------------------------------------------------------------

Bei einem Balken, der in Feldmitte, $x = l/2$, eine Einzelkraft $P$ tr\"{a}gt, und wo also
\bfoo
\Pi(w) = -\frac{1}{2} P\,w(\frac{l}{2})
\efoo
ist, passiert genau das, was die Mengenlehre voraussagt: Je mehr Zwischenlager man
anordnet, s. Bild \ref{BalkenKraft}, um so kleiner wird die Durchbiegung $w(l/2)$ in der
Balkenmitte. Denn die wachsende Zahl von Zwischenlager verringert den Umfang von $\mathcal{V}$,
verringert damit auch den Betrag der potentiellen Energie und verringert daher die
Durchbiegung $w$.
%%%%%%%%%%%%%%%%%%%%%%%%%%%%%%%%%%%%%%%%%%%%%%%%

{\small Aus Gr\"{u}nden der Symmetrie wollen wir noch kurz erw\"{a}hnen, dass das {\em
Kraftgr\"{o}{\ss}enverfahren\/} auf dem {\em Prinzip vom Minimum der komplement\"{a}ren
Energie}\footnote{Wenn man das Vorzeichen umdreht, dann hat man das Prinzip vom Maximum
der komplement\"{a}ren Energie. Die Bezeichnungen in der Literatur sind nicht eindeutig.}
\bfoo
\Pi_c(w) = - \frac{1}{2}\, \int_0^{\,l} \frac{M^2}{EI}\,dx + V\,\delta \qquad \rightarrow
\qquad \mbox{Minimum}
\efoo
beruht ($\delta $ soll hier f\"{u}r eine vorgegebene Lagersenkung stehen, die auch Null sein
kann). Dieses Prinzip ist praktisch das Gegenst\"{u}ck zum {\em Prinzip vom Minimum der
potentiellen Energie\/}. Das Minimum von $\Pi_c$ suchen wir unter allen Biegelinien $w =
w_0 + X_1 \,w_1 + \ldots X_n\,w_n$, die partikul\"{a}re L\"{o}sungen der Gleichung $EI\,w^{IV} =
p$ sind. Der Ansatz besteht aus der Biegelinie $w_0$ des Lastspannungszustandes,
$EI\,w^{IV}_0 = p$, und den Biegelinien $w_i$ der Einheitsspannungszust\"{a}nde,
$EI\,w^{IV}_i = 0$. Besser bekannt ist das Moment $M$ dieses Ansatzes
\bfoo
M = M_0 + X_1\,M_1 + X_2\,M_2 + \ldots + X_n\,M_n \,.
\efoo
Die Forderung $\partial \,\Pi_c /\partial\, X_i = 0$ f\"{u}hrt dann auf das Gleichungssytem
\bfoo
\vek F\,\vek x = -\,\vek \delta_0
\efoo
mit der {\em Flexibilit\"{a}tsmatrix\/} $\vek F = [\delta_{\,ij}]$, dem Vektor $\vek x =
[X_1,X_2,\ldots, X_n]$ und dem Vektor $\vek \delta_0 = [\delta_{10}, \delta_{20}, \ldots
,\delta_{n0}]$. Beim Prinzip vom Minimum der komplement\"{a}ren Energie sind die \hlq
\"{a}quivalenten Knotenkr\"{a}fte\grq\, also gerade die \"{U}berlagerung des Momentes $M_0$ des
Lastspannungszustandes mit den Momenten $M_i$ der Einheitsspannungszust\"{a}nde
\bfoo
\delta_{i0} = \int_0^{\,l} \frac{M_0\,M_i}{EI}\,dx\,.
\efoo
Theoretisch k\"{o}nnte man daher ein FE-Programm auch so schreiben, dass man einen
Lastspannungszustand w\"{a}hlt und diesen mit $n$ Einheitsspannungszust\"{a}nden anreichert. Bei
eindimensionalen Problemen (Rahmen), wo $n$ gleich dem Grad der statischen
Unbestimmtheit ist, erh\"{a}lt man so immer die exakte L\"{o}sung. Bei Fl\"{a}chentragwerken
br\"{a}uchte man, wegen $n = \infty$, dagegen unendlich viele $X_i$.
} % ENDE SMALL

%%%%%%%%%%%%%%%%%%%%%%%%%%%%%%%%%%%%%%%%%%%%%%%%%%%%%%%%%%%%%%%%%%%%%%%%

%-----------------------------------------------------------------
\begin{figure}[tbp]
\if \bild 2 \sidecaption \fi
\includegraphics[width=.6\textwidth]{\Fpath/AUSGLEICHSGERADE}
\caption{Ausgleichsgerade} \label{Ausgleichsgerade}
\end{figure}%
%-----------------------------------------------------------------

Fehlerquadrat ist ein Begriff aus der Ausgleichsrechnung. Wenn man durch eine Reihe von
Messdaten, s. Bild \ref{Ausgleichsgerade}, eine Gerade $M_h(x) = a\,x + b$ so legt,
\bfo\label{NormalenG}
a\,x_1 + b &=& M(x_1)\,, \nn \\
a\,x_2 + b &=& M(x_2)\,, \\
\ldots\quad &=&\quad \ldots\,,  \nn \\
a\,x_n + b &=& M(x_n) \nn\,,
\efo
dass das Fehlerquadrat, also das Quadrat des Abstands aller Messwerte
\bfoo
F = \sum_{i = 1}^n (M_h(x_i) - M(x_i))^2  \qquad\rightarrow \qquad \mbox{Minimum}
\efoo
von der Geraden in der Summe ein Minimum wird,
\bfo\label{EMin}
\frac{\partial\,F}{\partial\,a} = 0\,, \qquad \frac{\partial\,F}{\partial\,b} = 0\,,
\efo
so hat man das Fehlerquadrat minimiert. Die Aufgabe (\ref{EMin}) zu l\"{o}sen, ist
gleichbedeutend damit, das $2 \times 2$ Gleichungssystem
\bfoo
\vek A^T_{(2 \times n)}\vek A_{(n \times 2)} \, \left[ \barr {c} a \\ b \earr \right] =
\vek A^T_{(2 \times n)}\, \vek m_{(n)}\,,
\efoo
die {\em Normalengleichung\/}\index{Normalengleichung} zu l\"{o}sen, wobei $\vek A$ die
Koeffizientenmatrix aus (\ref{NormalenG}) ist, und der Vektor $\vek m = [M(x_1), M(x_2)
\ldots]$ die rechte Seite aus (\ref{NormalenG}) ist.

In der FEM ist der Momentenansatz nat\"{u}rlich vielgliedriger
\bfoo
M_h(x) = \sum_i w_i \,M_i(x)\,, \qquad M_i(x) = - EI\,\Np_i''(x)\,,
\efoo
und da wir den Ansatz bez\"{u}glich aller Punkte $0 < x < l$ eines Balkens optimieren, steht
in der FEM statt der Summe ein Integral, s. (\ref{A12FehlerQ}).

Die Beziehung $\vek K\vek u = \vek f$ ist genau die zugeh\"{o}rige {\em
Normalengleichung\/}. Sie entsteht, wenn man die unendlich vielen Forderungen $M_h(x) =
M(x)$ (es gibt unendlich viele Punkte $x_1,x_2,\ldots$ im Intervall $[0,l]$) an den
FE-Ansatz -- sinngem\"{a}{\ss} also eine Matrix $\vek A_{\infty \times n}$ -- mit ihrer
Transponierten $\vek A^T_{n \times \infty}$ von links multipliziert, und die
Diagonalmatrix $\vek C_{\infty \times \infty}$ mit den Gewichten $EI$ dazwischen schiebt
\bfo\label{Grundmuster}
\vek A_{(n \times \infty)}^T \,\vek C_{(\infty \times \infty)}^{} \vek A_{(\infty \times
n)}^{} = \vek K_{(n \times n)}\,.
\efo
Auf der Diagonalen der Matrix $\vek C$ steht in jeder Zeile die Zahl $EI$, \cite{Strang2}.

\subsection{Gewichtetes Mittel}\index{gewichtetes Mittel}
Das $EI$ kommt aus der Wechselwirkungsenergie\index{Wechselwirkungsenergie}
\bfoo
a(w,\hat{w}) = \int_0^{\,l} \frac{M\,\hat{M}}{EI}\,dx\,,
\efoo
denn die FEM stellt die FE-Momente so ein, dass im {\em gewichteten\/} quadratischen
Mittel die Abweichungen gegen\"{u}ber den exakten Momenten m\"{o}glichst klein sind
\bfo\label{NormalenE}
F = \int_0^{\,l} \,\frac{(M - M_h)^2}{EI}\,dx = \int_0^{\,l} (M - M_h)\,(\kappa -
\kappa_h) \,dx \,\, \to \,\,\mbox{Min.}
\efo
was bedeutet, dass sie den Fehler in den Momenten mit dem Fehler in den Kr\"{u}mmungen
multipliziert und dieses Produkt minimiert.

Wenn die Biegesteifigkeit $EI$ konstant ist, besteht nat\"{u}rlich kein Unterschied zwischen
dem einfachen quadratischen Mittel und dem gewichteten quadratischen Mittel.

%----------------------------------------------------------------------------------------------------------
\begin{figure}[tbp]
\if \bild 2 \sidecaption \fi
\includegraphics[width=.65\textwidth]{\Fpath/MBALKEN}
\caption{Kragarm mit Streckenlast und FE-Ersatzbelastung. Die Gr\"{o}{\ss}e der Knotenkr\"{a}fte und
Knotenmomente wird so eingestellt, dass das Fehlerquadrat von $M - M_h$ zum Minimum wird.
Die richtigen Knotenkr\"{a}fte und Knotenmomente sind gerade die inversen Festhaltekr\"{a}fte}
\label{MBalken}
\end{figure}%
%----------------------------------------------------------------------------------------------------------
\subsection{Global und lokal}
Das Fehlerquadrat ist ein {\em globales Mass\/}, in das alle Fehler l\"{a}ngs der Strecke
$[0,l]$ eingehen. In dem Algorithmus steckt aber noch ein {\em lokaler Abgleich\/}, wie
wir im folgenden zeigen wollen. Zur Herleitung setzen wir kurz $EI = 1$ und benutzen die
Kurzschreibweise f\"{u}r Integrale, s. Kapitel \ref{Notation}.

Notwendig daf\"{u}r, dass
\bfoo
F &=& (M-M_h,M-M_h) = (M,M) - 2\,(M,M_h) + (M_h,M_h) \\
&=& (M,M) - 2\,\sum_i\,(M,M_i)\,w_i + \sum_{i,j}(M_i,M_j)\,w_i\,w_j
\efoo
zum Minimum wird, ist, dass der Gradient von $F$ bez\"{u}glich der Knotenwerte $w_i$
verschwindet
\bfoo
\frac{\partial F}{\partial w_i} = 2\, \sum_{j}(M_i,M_j)\,w_j - 2\,(M,M_i) = 2\,(M_h,M_i)
- 2 (M,M_i) = 0\,.
\efoo
Die L\"{o}sung der Aufgabe (\ref{NormalenE}) ist also gleichbedeutend damit, den Fehler $M -
M_{h}$ so einzustellen, dass er orthogonal zu den Ans\"{a}tzen $M_i $ f\"{u}r die Momente ist --
nun wieder mit $EI$ --
\bfoo
\int_0^{\,l} \frac{(M - M_{h} )\,M_i}{EI}  \,dx = 0 \,,\qquad i = 1,2\ldots \qquad
\mbox{(lokaler Abgleich)}\,.
\efoo
\subsection{Die Me{\ss}strecke}\index{Me{\ss}strecke}
Die Me{\ss}strecken f\"{u}r den lokalen Abgleich sind nicht die einzelnen Elemente, sondern die
Tr\"{a}ger der einzelnen Dachfunktionen $M_i$. (Die Momente aus den Einheitsverformungen
$\Np_i$ der Knoten sind Dachfunktionen). Au{\ss}erhalb ihres Tr\"{a}gers $\Omega_i$ ist eine
Dachfunktion ja Null, und damit reduziert sich der lokale Abgleich auf ein Integral \"{u}ber
das Intervall $\Omega_i = [x_{i-1},x_{i+1}]$
\bfoo
\int_0^{\,l} \frac{(M - M_h)\,M_i}{EI}\,dx = \int_{x_{i-1}}^{x_{i+1}} \frac{(M -
M_h)\,M_i}{EI}\,dx =  0\,.
\efoo
Bei einem Durchlauftr\"{a}ger besteht die Teststrecke pro Dachfunktion $M_i$ also immer aus
zwei aufeinander folgenden Elementen. Und das Plus und Minus des Fehlers $M - M_h$ muss
\"{u}ber diese Teststrecke so verteilt sein, dass im gewichteten Mittel der Fehler
verschwindet.

\subsection{Zusammenfassung} Schreiben wir das ganze noch einmal in der \"{u}blichen
FE-Notation, dann hei{\ss}t dies: Das gewichtete, quadratische Mittel des Fehlers $\vek
\sigma - \vek \sigma_h$ in den Spannungen wird zum Minimum gemacht
\bfoo
F = a(u - u_h,u - u_h) = \int_{\Omega} (\vek \sigma - \vek \sigma_h) \dotprod (\vek
\varepsilon- \vek \varepsilon_h) \,d\Omega \quad \rightarrow \quad \mbox{Minimum}\,,
\efoo
was lokal bedeutet, dass der Fehler in den Spannungen orthogonal zu den Verzerrungen aus
den Einheitsverformungen der Knoten ist
\bfoo
\int_{\Omega_i} (\vek \sigma - \vek \sigma_h) \dotprod \vek \varepsilon_i \,d\Omega =
0\,,
\efoo
oder wegen $\delta A_i = \delta A_a$, dass die Differenz der Lasten, $\vek p - \vek
p_h$, keine Arbeit leistet, wenn die Knoten Einheitsverformungen ausf\"{u}hren
\bfoo
\delta A_i = \int_{\Omega_i} (\vek \sigma - \vek \sigma_h) \dotprod \vek \varepsilon_i
\,d\Omega = \delta A_a = \int_{\Omega} (\vek p - \vek p_h)\dotprod \vek \Np_i \,d\Omega
= 0\,.
\efoo
Das ist wieder das schwache Gleichheitszeichen. Zwei Lastf\"{a}lle $p$ und $p_h$ sind f\"{u}r ein
FE-Programm identisch, wenn sie sich gegen\"{u}ber den Einheitsverformungen gleich verhalten.

%%%%%%%%%%%%%%%%%%%%%%%%%%%%%%%%%%%%%%%%%%%%%%%%%%%%%%%%%%%%%%%%%%%%%%%%%%%%%%%%%%%%%%



Bei einer Scheibe hat man ganz analog
\bfo\label{Ortho4}
\delta A_i &=&\int_{\Omega}\underbrace{(\vek  \sigma - \vek \sigma_{h})}_{unbekannt}
\dotprod \vek \varepsilon_i\, d\Omega
= \int_{\Omega} \underbrace{(\vek p - \vek p_{h})}_{berechenbar} \dotprod \vek \Np_i\, d\Omega \nn \\
&+& \int_{\Gamma} \underbrace{(\vek t - \vek t_{h})}_{berechenbar}\dotprod \vek \Np_i \,
ds + \sum_k \int_{\Gamma_k} \underbrace{\vek t_{\Delta}}_{berechenbar}\dotprod \,\,\vek
\Np_i \, ds  = \delta A_a\,,
\efo
wobei die Beitr\"{a}ge
\bfoo
\sum_k \int_{\Gamma_k} \vek t_{\Delta} \dotprod \vek \Np_i \, ds
\efoo
die virtuellen Arbeiten der Spannungsspr\"{u}nge $\vek t_\Delta$ (= Linienkr\"{a}fte) auf den
Netzkanten $\Gamma_k$ sind.

Statisch bedeutet die Orthogonalit\"{a}t in den Spannungen, dass die Einheitslastf\"{a}lle $\vek
p_i$ -- die {\em actio} hinter den Verschiebungsfeldern $\vek \Np_i $ -- keine Arbeit
leisten auf den Wegen des Verschiebungsfehlers $\vek e(\vek x) = \vek u(\vek x) - \vek
u_{h}(\vek x)$
\bfoo
\delta A_i = \int_{\Omega} (\vek \sigma - \vek \sigma_{h}) \dotprod \vek \varepsilon_i
\,d\Omega = \int_{\Omega} (\vek p - \vek p_{h}) \dotprod \vek \Np_i \, d\Omega = \delta
A_a = 0\,.
\efoo
Dies ist der richtige Augenblick, an das {\em Kraftgr\"{o}{\ss}enverfahren} zu erinnern. Der
endg\"{u}ltige Biegemomentenverlauf $M = M_0 + X_1 \,M_1 + X_2 \, M_2 + \ldots$ eines
Durchlauftr\"{a}gers ist orthogonal zu den Einheitsspannungszust\"{a}nden $X_i$
\bfoo
\int_0^{\,l} \frac{M\,M_i}{EI} \,dx = 0 \qquad \Rightarrow \qquad w(x_i) = 0, \quad
\mbox{oder} \quad\Delta w'(x_i) = 0\,\quad\mbox{etc.}\,,
\efoo
was, wie wir wissen, bedeutet, dass die (urspr\"{u}nglich gel\"{o}sten) Verformungsbedingungen
wie $w(x_i) = 0$ oder $\Delta w'(x_i) = 0$ (Relativverdrehung) von der exakten L\"{o}sung
eingehalten werden.

Nun ist man versucht zu vermuten, dass dies gerade die Bedeutung der Orthogonalit\"{a}t
(\ref{Ortho4}) ist, dass aus ihr also folgt, dass der Verschiebungsfehler in den Knoten
Null ist
\bfo
\int_{\Omega} (\vek \sigma - \vek \sigma_{h}) \dotprod \vek \varepsilon_i \,d\Omega = 0
\qquad \stackrel{\raisebox{1.0mm}{?}}{\Rightarrow} \qquad\vek u(\vek x_i) - \vek u_{h}
(\vek x_i) = \vek 0\,.
\efo
Dies ist aber nicht richtig. Die FE-L\"{o}sung ist {\em nicht} die Interpolierende, denn das
w\"{a}re ja gerade die Bedeutung dieses Schlusses.

Nur bei eindimensionalen Problemen, wie einem Balken, ist die Orthogonalit\"{a}t
\bfo\label{Ortho41}
\delta\, A_i = \int_0^{\,l} \frac{(M - M_{h})\, \,M_i}{EI} \,\,dx = 0 \,,
\efo
gleichbedeutend damit, dass der Fehler in den Durchbiegungen in den Knoten Null ist, wie
man leicht verifiziert.

Das Moment $M_i$ ist hier das Moment, das zu der Einheitsverformung $\Np_i$ geh\"{o}rt, und
der Schluss ist richtig, weil -- wegen $\delta A_a = \delta A_i$ -- die Orthogonalit\"{a}t
(\ref{Ortho41}) gleichbedeutend ist mit
\bfoo
0 = \delta A_i = \delta A_a = (w(x_i) - w_{h} (x_i)) \times \, P = 0 \times \, P\,.
\efoo
Die Knotenkraft $P$ ist die Kraft, die die Einheitsverformung des Knotens $x_i$
hervorruft. (Sind die Abst\"{a}nde zu den Knoten links und rechts ungleich, so ist auch noch
ein Knotenmoment $M$ beteiligt, aber dies leistet, weil die Verdrehung im Knoten Null
ist, $\Np_i'(x_i ) = 0$, keine Arbeit).

Bei Fl\"{a}chentragwerken ist die die Einheitsverformung $\vek \Np_i $ erzeugende Kraft
keine konzentrierte Einzelkraft, sondern eine diffuse {\em  Wolke} von Fl\"{a}chen- und
Linienkr\"{a}ften in der Umgebung des Knotens, und deswegen ist die punktgenaue
Knotenbedingung $\vek u(\vek x_i) = \vek u_{h} (\vek x_i)$ nicht erf\"{u}llt.

Einem FE-Programm geht es ja auch nicht darum, das exakte Verschiebungsfeld $\vek u$ in
den Knoten zu interpolieren, sondern das FE-Programm strebt danach, den Fehler in den
Spannungen im Mittel zu Null zu machen. Das ist baustatisch gesehen sicherlich auch viel
vern\"{u}nftiger, als die Interpolation in den Knoten. Nur bei eindimensionalen Problemen
bekommt man beides: {\em  Interpolation + minimalen Abstand in der Energie}.

%%%%%%%%%%%%%%%%%%%%%%%%%%%%%%%%%%%%%%%%%%%%%%%%%%%%%%%%%%%%%%%%%%%%%%%%%%%%%%%%%%%%%%
%----------------------------------------------------------------------------------------------------------
\begin{figure}[tbp]
\if \bild 2 \sidecaption \fi
\includegraphics[width=.8\textwidth]{\Fpath/ALSOB}
\caption{Alle drei Lastf\"{a}lle sind \"{a}quivalent. Aber nur die oberen beiden LF sind \hlq
echt\grq. Der LF Knotenkr\"{a}fte ist fiktiv. Er repr\"{a}sentiert rechentechnisch die
\"{A}quivalenzklasse, zu der die beiden LF geh\"{o}ren} \label{Alsob}
\end{figure}%
%----------------------------------------------------------------------------------------------------------
%-------------------------------------------------------------------------
\begin{figure}[tbp]
\if \bild 2 \sidecaption \fi
\includegraphics[width=.75\textwidth]{\Fpath/EINLOAD}
\caption{Eine solche Lastverteilung f\"{u}hrt, wenn die Streckenlast richtig skaliert wird,
bei linearen Ansatzfunktionen f\"{u}r die vertikale Verschiebung zu einer \"{a}quivalenten
Knotenkraft $P = 1$ im mittleren Knoten und zu Null Knotenkr\"{a}ften in den abliegenden
Knoten} \label{Einload}
\end{figure}%%
%-------------------------------------------------------------------------
In der Fr\"{u}hzeit der finiten Elemente haben wir die Umrechnung der Blocklasten,
Linienlasten, etc. in \"{a}quivalente Knotenkr\"{a}fte selbst vorgenommen. Dabei haben wir
praktisch die Belastung mit linearen Ansatzfunktionen \"{u}berlagert, denn das entspricht
dem \"{u}blichen \hlq auf die Knoten verteilen\grq.

Die Knotenkr\"{a}fte $f_i$ hei{\ss}en {\em konsistente Knotenkr\"{a}fte\/}, wenn sie mit der Formel
(\ref{Kons}) berechnet werden. Bei der \hlq Verteilung von Hand\grq\, kann es dagegen zu
einer (leichten) Abweichung von den exakten $f_i$ kommen. Dann spricht man von nicht
konsistenten Knotenkr\"{a}ften.

%------------------------------------------------------------------------------------------------
\begin{figure}[tbp]
\if \bild 2 \sidecaption \fi
\includegraphics[width=.8\textwidth]{\Fpath/INGENIEURBALKEN}
\caption{Die FE-L\"{o}sung ist n\"{a}her an der exakten L\"{o}sung als die Ingenieurl\"{o}sung}
\label{Ingenieurbalken}
\end{figure}%
%----------------------------------------------------------------------------------------------------------

Betrachten wir einen Einfeldtr\"{a}ger mit einer linear ansteigenden Streckenlast wie in
Bild \ref{Ingenieurbalken}. Eine Reduktion der Streckenlast in die Knoten mit
Knotenkr\"{a}ften allein (\hlq Ingen\-ieur\-l\"{o}\-sung\grq), s. Bild \ref{Ingenieurbalken}b,
w\"{a}re nicht konsistent\footnote{In der angels\"{a}chsischen Literatur unterscheidet man
zwischen {\em work-equivalent loads \/} und {\em lumped loads\/}}. Es w\"{u}rden die
Knotenmomente fehlen. (Der klassisch ausgebildete Statiker w\"{u}rde sagen, der Aufsteller
hat die Festhaltemomente vergessen).

Die Knotenmomente entstehen, wenn man die zu den Drehfreiheitsgraden der Knoten
geh\"{o}renden Einheitsverformungen $\Np_i(x)$ mit der Streckenlast $p$ \"{u}berlagert.

L\"{a}sst man diese Knotenmomente weg, so sind der Ersatzlastfall und der Lastfall
Streckenlast nicht mehr einander \"{a}quivalent. Bei einer Drehbewegung eines Knotens w\"{u}rde
die Streckenlast eine virtuelle Arbeit $\delta A_a(p,\Np_i) \neq 0$ leisten, nicht aber
die \"{a}quivalenten Knotenkr\"{a}fte, $\delta A_a(p_{h},\Np_i) =  0$,  und das w\"{a}re ein
Widerspruch.

Dies hat auch eine praktische Bedeutung. Die L\"{o}sung des Ingenieurs weist einen gr\"{o}{\ss}eren
Fehler auf, als die FE-L\"{o}sung. Dies sieht man, wenn man sich die potentielle Energie der
verschiedenen L\"{o}sungen anschaut, (P = Praktiker = Ingenieur),
\bfoo
\Pi_{exakt} = - 108,36\, \frac{1}{EI}  <  \Pi_{h} = - 108,28 \,\frac{1}{EI} <  \Pi_{P} =
-107,85 \,\frac{1}{EI}\,.
\efoo
Die FE-L\"{o}sung ist dichter an dem exakten Wert der potentiellen Energie.

%%%%%%%%%%%%%%%%%%%%%%%%%%%%%%%%%%%%%%%%%%%%%%%%%%%%%%%%%%%%%%%%%%%%%%%%%%%%%%%%%%%%%%

\subsection{Herleitung} Die Steifigkeitsmatrizen eines Stab- oder Balkenelements
\bfoo
 \vek K_{\,\mbox{\small Stab}}   = \frac{EA}{l} \left [\barr {r
r} 1 & -1 \\ -1 & 1 \earr \right ]\,,\qquad \vek K_{\,\mbox{\small Balken}} =
\frac{EI}{l^3} \left[
\begin{array}{r r r r}
 12 & -6l & -12 &-6l \\
 -6l & 4l^2 & 6l &2l^2 \\
 -12 & 6l & 12 & 6l \\
 -6l &2l^2 &6l &4l^2
 \end{array}
  \right]\,,
\efoo
leiten wir in der Regel auf statischem Wege ab.

Im Prinzip geht es aber auch einfacher, denn das Element $k_{\,ij}$ einer
Steifigkeitsmatrix ist gerade die Wechselwirkungsenergie\index{Wechselwirkungsenergie}
zwischen den Einheitsverformungen der Knoten
\bfo\label{kijA14}
\fbox{ \parbox{4cm} {
\[k_{\,ij} = a(\Np_i,\Np_j)
\]
} }
\efo
In der Stabstatik ist, ($A\,dx = dV$),
\bfoo
k_{\,ij} = a(\Np_i,\Np_j) &=& \int_0^{\,l} EA\,\Np_i'\,\Np_j'\,dx = \int_0^{\,l} \sigma_i\,\varepsilon_j\,A\,dx \qquad \mbox{Stab} \\
k_{\,ij} = a(\Np_i,\Np_j) &=& \int_0^{\,l} EI\,\Np_i''\,\Np_j''\,dx = \int_0^{\,l} m_i\,\kappa_j\,dx \,\,\,\qquad \mbox{Balken} \\
\efoo

%%%%%%%%%%%%%%%%%%%%%%%%%%%%%%%%%%%%%%%%%%%%%%%%%%%%%%%%%%%%%%%%%%%%%%%%%%%%%%%%%%%%%%

\subsubsection{Arbeitsmatrizen}\index{Arbeitsmatrizen} Steifigkeitsmatrizen sind also {\em  Arbeitsmatrizen\/}.
Um das energetische Verhalten des Tragwerks kennenzulernen, ziehen wir am ersten Knoten,
d.h. wir lenken den Knoten in Richtung des Freiheitsgrades $u_1$ um eine L\"{a}ngeneinheit
aus. Dazu sind gewisse Kr\"{a}fte n\"{o}tig, einmal um den Knoten zu ziehen, und zum andern, um
die anderen Knoten festzuhalten, denn nur dieser eine Knoten soll sich bewegen. All
diese Kr\"{a}fte in ihrer Verteilung und charakteristischen Gr\"{o}{\ss}e stellen den
Einheitslastfall $ p_1$ dar. Dem so deformierten und belasteten Tragwerk erteilen wir
dann in Gedanken nacheinander die virtuellen Verr\"{u}ckungen $ \Np_1, \Np_2, \ldots, \Np_n$
und berechnen dabei jedesmal die virtuellen Arbeiten, die die Kr\"{a}fte des
Einheitslastfalls $ p_1$ auf diesen Wegen leisten. Diese $n$ Zahlen
\bfoo
\delta A_a( p_1, \Np_i) = \mbox{Arbeit der Lasten $ p_1$ auf den Wegen $ \Np_i$}  \qquad
i = 1,2,\ldots n
\efoo
stellen die erste Spalte der Steifigkeitsmatrix dar. Danach aktivieren wir den
Freiheitsgrad $u_2$ (alle anderen $u_i$ sind Null) und wiederholen das Spiel, bis die
ganze Steifigkeitsmatrix $ \vek K$ aufgebaut ist.

Nachdem das Produkt $\vek K\,\vek u = u_1\,\vek s_1 + u_2\,\vek s_2 + \ldots + u_n\,\vek
s_n$ gleich die Summe der mit $u_i$ gewichteten Spalten $\vek s_i$ von $\vek K$ ist, ist
klar, dass wegen
\bfoo
\vek K\,\vek e_i = \vek s_i \qquad  \vek e_i = \mbox{i-ter Einheitsvektor}\,,
\efoo
die Spalten $\vek s_i$ von $\vek K$ gerade die \"{a}quivalenten Knotenkr\"{a}fte sind, die zu
den einzelnen Einheitsverformungen der Knoten geh\"{o}ren.

%%%%%%%%%%%%%%%%%%%%%%%%%%%%%%%%%%%%%%%%%%%%%%%%%%%%%%%%%%%%%%%%%%%%%%%%%%%%%%%%%%%%%%

\subsubsection{Erste Greensche Identit\"{a}t}\index{erste Greensche Identit\"{a}t}
Wie die Steifigkeitsmatrizen mit der Ersten Greenschen Identit\"{a}t zusammenh\"{a}ngen, sei am
Beispiel des Balkens erl\"{a}utert.

Setzt man zwei Einheitsverformungen in die erste Greensche Identit\"{a}t so folgt
\bfoo
G(\Np_i,\Np_j) = \underbrace{\int_0^{\,l} EI \Np_i^{IV} \,\Np_j\,dx + \left [V_i \,\Np_j
- M_i \,\Np_j' \right ]_0^l}_{p_{\,ij}} - \underbrace{ \int_0^{\,l} EI \Np_i'' \Np_j''
dx}_{k_{\,ij}} = 0\,.
\efoo
Wegen
\bfoo
 \delta A_a(p_i,\Np_j) = p_{\,ij} = k_{\,ij} = \delta A_i(\Np_i,\Np_j)
\efoo
ist also die virtuelle innere Energie $k_{\,ij}$ zwischen zwei Einheitsverformungen
$\Np_i$ und $\Np_j$ gleich der virtuellen \"{a}u{\ss}eren Arbeit $p_{\,ij}$, die die \"{a}u{\ss}eren
Kr\"{a}fte, die zur Biegelinie $\Np_i$ geh\"{o}ren, auf den Wegen der virtuellen Verr\"{u}ckung
$\Np_j$ leisten.

Die \"{a}u{\ss}eren Kr\"{a}fte der Biegelinie $\Np_i$ sind die Kr\"{a}fte, die den Balken in die Form
$\Np_i$ zwingen. Das sind also die Streckenlast $EI\,\Np_i^{IV}$ und die Balkenendkr\"{a}fte
$V_i(0), V_i(l)$ und die Einspannmomente $M_i(0), M_i(l)$ links und rechts. F\"{u}r diesen
Lastfall benutzen wir das Symbol $p_{\,i}$, w\"{a}hrend die doppelt indizierte Gr\"{o}{\ss}e
$p_{\,ij}$ die virtuelle \"{a}u{\ss}ere Arbeit $\delta A_a(p_i,\Np_j)$ zwischen dem Lastfall
$p_i$ und der virtuellen Verr\"{u}ckung $\Np_j$ bedeutet.

Das Gleichungssystem $\vek K\,\vek u = \vek f$ ist also \"{a}quivalent mit dem
Gleichungssystem $\vek P\,\vek u = \vek f$. Setzen wir f\"{u}r den Vektor $\vek P\,\vek u$
den Vektor $\vek f_h$, so ist $\vek K\,\vek u = \vek P\,\vek u = \vek f$ identisch mit
\bfoo
\vek f_h = \vek f\,,
\efoo
was komponentenweise gelesen das bekannte Resultat ist, dass bei jeder virtuellen
Verr\"{u}ckung $\Np_i$ der FE-Lastfall $p_h = \sum_j u_j\,p_j$ mit dem Originallastfall $p$
arbeits\"{a}quivalent ist
\bfoo
f_{h_i} = \delta A_a(p_h,\Np_i) = \delta A_a(p,\Np_i) = f_i\,.
\efoo

%%%%%%%%%%%%%%%%%%%%%%%%%%%%%%%%%%%%%%%%%%%%%%%%%%%%%%%%%%%%%%%%%%%%%%%%%%%%%%%%%%%%%%
Um zu verstehen, wie aus der Kopplung der Freiheitsgrade die Addition der Steifigkeiten
folgt, erinnern wir zun\"{a}chst an zwei Rechenregeln aus der Matrizenrechnung.

a) Wenn man die Spalten einer Einheitsmatrix $\vek I$ in irgendeiner Reihenfolge
vertauscht und eine gleich gro{\ss}e Matrix $\vek K$ von rechts mit dieser Matrix $\vek I_V$
multipliziert, dann werden die {\em Spalten\/} von $\vek K$ genau so vertauscht.
Multipliziert man die Matrix $\vek K$ von links mit der transponierten Matrix $\vek
I^T_V$, so werden die {\em Zeilen\/} von $\vek K$ nach dem gleichen Muster vertauscht.

b) Wenn man eine Matrix von rechts mit einem Vektor multipliziert, in dem zwei Einsen
\"{u}bereinander stehen, dann werden die entsprechenden Spalten der Matrix addiert.
Multipliziert man die Matrix von links mit dem transponierten Vektor, dann werden die
entsprechenden Zeilen der Matrix addiert,
\bfoo
\left [ \barr {c@{\hspace{2mm}} c} a & b \\ c & d \earr \right] \left [\barr {c} 1 \\ 1
\earr \right] = \left [\barr {c} a + b \\ c + d \earr \right]\qquad \left [ \barr {c c}
1, & 1 \earr \right] \left [ \barr {c@{\hspace{2mm}} c} a & b \\ c & d \earr \right] =
\left [ \barr {c c} a + c, & b + d  \earr \right]\,.
\efoo
%----------------------------------------------------------------------
\begin{figure}[tbp]
\if \bild 2 \sidecaption \fi
\includegraphics[width=.6\textwidth]{\Fpath/ZUGSTAB2}
\caption{Zugstab aus zwei Elementen} \label{Zugstab2}
\end{figure}%%
%----------------------------------------------------------------------
Betrachten wir nun den Stab in Bild \ref{Zugstab2}, der sich aus zwei Stabelementen
zusammensetzt, so dass
\bfoo
\left [ \barr {r r r r} k_1 & - k_1 & 0 & 0 \\ - k_1 & k_1 & 0 & 0 \\ 0 & 0 &k_2 & -
k_2\\ 0 & 0 & -k_2 & k_2 \earr \right ] \left [ \barr {r} u_1^{(1)} \\ u_2^{(1)}
\\u_1^{(2)} \\u_2^{(2)} \earr \right] = \left [ \barr {r} f_1^{(1)} \\ f_2^{(1)}
\\f_1^{(2)}
\\f_2^{(2)} \earr \right] \qquad \mbox{oder} \qquad \vek
K_l\,\vek u_l = \vek f_l\,.
\efoo
Nun gilt f\"{u}r den Zusammenhang zwischen den globalen, $u_i$, und lokalen Freiheitsgraden,
$u_i^{(j)}$,
\bfoo
\left [ \barr {r} u_1^{(1)} \\ u_2^{(1)} \\u_1^{(2)} \\u_2^{(2)} \earr \right] = \left
[\barr {r r r} 1\,\, & 0\,\, & 0
\\ 0\,\, & 1\,\, & 0
\\ 0\,\, & 1\,\, & 0
\\ 0\,\, & 0\,\, & 1 \earr \right ]
\left [ \barr {r} u_1 \\ u_2 \\ u_3 \earr \right ] \qquad \mbox{oder} \qquad \vek u_l =
\vek A\,\vek u\,,
\efoo
und wegen
\bfoo
\vek u_l^T\,\vek K_l\,\vek u_l = \vek u^T\,\vek A^T\,\vek K_l\,\vek A\,\vek u = \vek
u^T\,\vek K\,\vek u
\efoo
folgt f\"{u}r die Steifigkeitsmatrix $\vek K$ des Stabes
\bfoo
\left [\barr {r @{\hspace{2mm}}r@{\hspace{2mm}} r@{\hspace{2mm}} r} 1 & 0 & 0 & 0
\\ 0 & 1 & 1 & 0
\\ 0 & 0 & 0 & 1
\earr \right ] \left [ \barr {r r r r} k_1 & - k_1 & 0 & 0 \\ - k_1 & k_1 & 0 & 0 \\ 0 & 0 &k_2 & - k_2\\
0 & 0 & -k_2 & k_2 \earr \right ] \left [\barr {r@{\hspace{2mm}} r@{\hspace{2mm}} r} 1 &
0 & 0
\\ 0 & 1 & 0
\\ 0 & 1 & 0
\\ 0 & 0 & 1 \earr \right ] = \left [ \barr {c c c} k_1\,\, & -k_1\,\, & 0 \\ -k_1\,\, & k_1 + k_2\,\, & -k_2\,\, \\ 0 & -k_2\,\, & k_2
\earr \right]\,.
\efoo
Durch die Multiplikation von rechts werden also erst die Spalten 2 und 3 addiert und
dann durch die Multiplikation von links die Zeilen 2 und 3. Nach diesem Schema verl\"{a}uft
formal der Aufbau der Gesamtsteifigkeitsmatrix, und nat\"{u}rlich l\"{a}sst sich diese Logik auch
zur Behandlung von allgemeineren Koppelbedingungen einsetzen.
%%%%%%%%%%%%%%%%%%%%%%%%%%%%%%%%%%%%%%%%%%%%%%%%%%%%%%%%%%%%%%%%%%%%%%%%%%%%%%%


Am
Beispiel der Kopplung von Scheibe und Balken wollen wir das prinzipielle Vorgehen kurz
erl\"{a}utern.
%----------------------------------------------------------------------
\begin{figure}[tbp]
%\sidecaption
\includegraphics[width=.6\textwidth]{\Fpath/BALKENSCHEIBE4}
\caption{Ankopplung eines Balkens an eine Scheibe} \label{BalkenScheibe4}
\end{figure}%%
%----------------------------------------------------------------------

\subsubsection{Scheibe + Balken} Der Anschluss eines Balkens an eine Scheibe macht Schwierigkeiten, weil die
Scheibenknoten keine Drehfreiheitsgrade haben. Am einfachsten ist es, den Balken bis zum
n\"{a}chsten Innenknoten zu verl\"{a}ngern, also praktisch das Biegemoment in ein Kr\"{a}ftepaar
aufzul\"{o}sen.

Eine andere Variante ist in Bild \ref{BalkenScheibe4} skizziert. Der Mittelwert der
Drehbewegungen der Elementkanten links und rechts vom Anschluss soll gerade dem
Enddrehwinkel $\tan \varphi = u_3$ der St\"{u}tze sein, $(u_7 - u_2)/l_e = \tan \varphi$,
was bedeutet, dass an den Vektor $\vek u$ die Bedingung
\bfoo
\{ 0, - \frac{1}{l_e},-1,0,0, 0,\frac{1}{l_e}, 0, \ldots,0 \} \,\vek u = 0 \,.
\efoo
gestellt wird.

Die L\"{a}nge des Vektors $\vek s = \vek Z\,\vek u = \vek 0$ bildet, bei nicht eingehaltenen
Koppelbedingungen, ein Ma{\ss} daf\"{u}r, wie sehr die Bedingungen verletzt sind. Wichtet man
die einzelnen Fehler unterschiedlich, so kommt man auf die Formel $\vek s^T\,\vek
D\,\vek s$ f\"{u}r die L\"{a}nge, wobei auf der Diagonalen der Diagonalmatrix $\vek D =
[\alpha_1,\alpha_2,\ldots,\alpha_n]$ die unterschiedlichen Gewichte $\alpha_i$ stehen.
Dieser Gesamtfehler wird nun zur potentiellen Energie addiert
\bfoo
\Pi(\vek u) = \frac{1}{2}\,\vek u^T\,\vek K\,\vek u - \vek f^T\,\vek u +
\frac{1}{2}\,\vek s^T \vek D\,\vek s\,,
\efoo
und gefordert, dass die Ableitungen nach den $u_i$ und den Gewichten $\alpha_i$ Null
sind.
%%%%%%%%%%%%%%%%%%%%%%%%%%%%%%%%%%%%%%%%%%%%%%%%%%%%%%%%%%%%%%%%%%%%%%%%%%%%%%%

%----------------------------------------------------------------------------------------------------------
\begin{figure}[tbp]
\if \bild 2 \sidecaption \fi
\includegraphics[width=.5\textwidth]{\Fpath/PROJEKTION}
\caption{Der Schatten $\vek x'$ ist der Vektor in der Ebene mit dem k\"{u}rzesten Abstand
zur Spitze des Vektors $\vek x$. Der Fehler $\vek e$ steht senkrecht auf der Ebene. Der
Schatten ist {\em immer\/} k\"{u}rzer als das Original. K\"{u}rzer bedeutet in der Energiemetrik
kleinere Energie: Das FE-Tragwerk ist steifer} \label{Projektion1}
\end{figure}%
%----------------------------------------------------------------------------------------------------------

%%%%%%%%%%%%%%%%%%%%%%%%%%%%%%%%%%%%%%%%%%%%%%%%%%%%%%%%%%%%%%%%%%%%%%%%%%%%%%%


%%%%%%%%%%%%%%%%%%%%%%%%%%%%%%%%%%%%%%%%%%%%%%%%%%%%%%%%%%%%%%%%%%%%%%%%%%%%%%%%%%%%%%%%%%%%%%%%%%%%
\section{Polynome}\label{Polynome}\index{Polynome}
%%%%%%%%%%%%%%%%%%%%%%%%%%%%%%%%%%%%%%%%%%%%%%%%%%%%%%%%%%%%%%%%%%%%%%%%%%%%%%%%%%%%%%%%%%%%%%%%%%%%
Jede Funktion kann man in eine {\em Taylor-Reihe\/}\index{Taylor-Reihe} entwickeln
\bfoo
u(x) = u(0) + u'(0)\,x + u''(0)\,\frac{x^2}{2} + u'''(0) \,\frac{x^3}{3!} + \ldots\,,
\efoo
also in erster N\"{a}herung durch ein Polynom darstellen. Und so wie diese Reihe mit einem
konstanten Term, $u(0)$, beginnt, dann ein zweiter, linearer Term, $u'(0)\,x$, hinzukommt
und so weiter w\"{a}chst, so k\"{o}nnen wir auch die Spannungen in einem Element durch konstante,
lineare oder gar quadratische Spannungen ann\"{a}hern.
%--------------------------- --------------------------------------
\begin{figure}[tbp]
\if \bild 2 \sidecaption \fi
\includegraphics[width=1.0\textwidth]{\Fpath/POLYDEMO}
\caption{Lineare und quadratische Ansatzfunktionen} \label{PolyDemo}
\end{figure}%
%-----------------------------------------------------------------

Sind die Ans\"{a}tze f\"{u}r die Verschiebungen linear, dann sind die Spannungen konstant. Sind
die Ans\"{a}tze quadratisch, dann sind die Spannungen linear, etc. Der Polynomgrad der
Spannungen ist also immer eine Ordnung niedriger als der der Verschiebungen.

Sind die Verschiebungen linear, dann sprechen wir von linearen Elementen\index{lineare
Elemente}, sind sie quadratisch, dann sprechen wir von quadratischen
Elementen\index{quadratische Elemente}, etc.. Je h\"{o}her der Ansatz ist, desto mehr Knoten
hat ein Element. Bei {\em Lagrange-Elementen\/}\index{Lagrange-Elemente} gibt es Knoten
im Innern und auf dem Rand, bei {\em Serendipity-Elementen\/}\index{Serendipity-Element}
nur auf dem Rand. Lagrange-Elemente basieren auf den {\em
Lagrange-Polynomen\/}\index{Lagrange-Polynome}, s. Bild \ref{PolyDemo}.

Lagrange-Polynome der Ordnung $n$ in einer Dimension st\"{u}tzen sich auf $n$ Knoten ab und
sind von der Bauart
\bfoo
\Np_1^{(n)}(x) &=& \frac{(x_2 - x) \,(x_3 - x) \,(x_4 - x) \cdots (x_n - x)}{(x_2 - x_1)
\,(x_3 - x_1) \,(x_4 - x_1) \cdots (x_n -
x_1)}\,,\\
\Np_2^{(n)}(x) &=& \frac{(x_1 - x) \,(x_3 - x) \,(x_4 - x) \cdots (x_n - x)}{(x_1 - x_2)
\,(x_3 - x_2) \,(x_4 - x_2) \cdots (x_n - x_2)}\,,\\\vspace{0.2cm}
\ldots && \ldots \\
\Np_n^{(n)}(x) &=& \frac{(x_1 - x) \,(x_2 - x) \,(x_3 - x) \cdots (x_{n-1} - x)}{(x_1 -
x_n) \,(x_2 - x_n) \,(x_3 - x_n) \cdots (x_{n-1} - x_n)}\,.
\efoo
Im Nenner von $\Np_1^{(n)}(x)$ stehen die Momente $(x_i - x_1)$ der Knoten $x_i$ bezogen
auf den Knoten $x_1$ und im Z\"{a}hler stehen die Momente der Koordinate $x$ bezogen auf
alle Knoten bis auf den Knoten $x_1$. Durch Produktbildung erh\"{a}lt man daraus die
Formfunktionen f\"{u}r rechteckige Elemente
\bfoo
\Np_{...}^{(n)}(x,y) = \Np_j^{(n)}(x)\,\Np_k^{(n)}(y)\qquad j,k = 1,2,\ldots n\,.
\efoo
Ein Verschiebungsansatz vom Grade $n$ hei{\ss}t vollst\"{a}ndig, wenn er alle Polynome bis zum
Grad $n$ enth\"{a}lt
\bfoo
f(x) = a_0 + a_1\,x + a_2\,x^2 + \ldots + a_n\,x^n\,.
\efoo
Die Einhaltung dieser Bedingung ist deswegen nicht selbstverst\"{a}ndlich, weil wir in der
FEM die Verschiebungen nach den Einheitsverformungen\index{Kronecker-Delta}
\bfoo
\Np_i(x_j) = \delta_{\,ij} = \left \{ \barr {l l} 1 \qquad &i = j \\ 0 \qquad & i \neq j
\earr \right. \qquad \mbox{($\delta_{\,ij}$ = Kronecker-Delta)}
\efoo
der Knoten $x_j$ entwickeln. Das sind zwar alles Polynome vom Grad $\leq n$, aber es ist
nicht garantiert, dass sie zusammen jedes beliebige Polynom vom Grade $n$ auf dem Element
darstellen k\"{o}nnen
\bfoo
f(x) = a_0 + a_1\,x + a_2\,x^2 + \ldots + a_n\,x^n \stackrel{{\raisebox{1.0mm}{?}}}{=}
 \sum_{i = 1}^{n + 1} u_i\,\Np_i(x)\,.
\efoo
Ein Polynom $n$-ten Grades hat $n+1$ Koeffizienten $a_0, a_1, \ldots, a_n$ und daher muss
ein dazu passendes (Lagrange) Element $n+1$ Knoten haben. Ein lineares Element hat in
einer Dimension zwei Knoten, ein quadratisches drei Knoten, etc..
%--------------------------- --------------------------------------
\begin{figure}[tbp]
\if \bild 2 \sidecaption \fi
\includegraphics[width=.7\textwidth]{\Fpath/POLYBALKEN}
\caption{Die Einheitsverformungen eines Balkenelements sind Polynome dritten Grades}
\label{PolyBalken}
\end{figure}%
%-----------------------------------------------------------------
%--------------------------- --------------------------------------
\begin{figure}[tbp]
\if \bild 2 \sidecaption \fi
\includegraphics[width=.5\textwidth]{\Fpath/PASCAL}
\caption{Grafische Umsetzung des Pascalschen Dreiecks, Dreieckselemente mit
vollst\"{a}ndigen Ans\"{a}tzen ersten, zweiten, dritten, vierten und f\"{u}nften Grades}
\label{BildPascal}
\end{figure}%
%-----------------------------------------------------------------
Manche Ans\"{a}tze sind unvollst\"{a}ndig, weil gewisse Terme fehlen. Die Anzahl der
Polynomterme, die man f\"{u}r ein vollst\"{a}ndiges Polynom $n$-ter Ordnung in der $x-y$-Ebene
ben\"{o}tigt, w\"{a}chst rasch an, wie man am {\em Pascalschen Dreieck\/}\index{Pascalsches
Dreieck} ablesen kann
\begin{eqnarray*}
& 1 &\\
& x\;\;y &\\
& x^2\;\;\; xy\;\;\; y^2\;\;\; &\\
& x^3\;\;\; x^2y\;\;\; xy^2\;\;\; y^3 & \\
& x^4\;\;\; x^3y\;\;\; x^2y^2\;\;\; xy^3\;\;\; y^4 &\\
& x^5\;\;\; x^4y\;\;\; x^3y^2\;\;\;x^2y^3\;\;\; xy^4\;\;\; y^5 &
\end{eqnarray*}
F\"{u}r einen vollst\"{a}ndigen Ansatz nullter, erster, zweiter, dritter, vierter oder f\"{u}nfter
Ordnung (letzte Zeile in dem obigen Dreieck) ben\"{o}tigt man 1, 3, 6, 10, 15 oder 21
Glieder, was umgekehrt bedeutet, dass nur Elemente mit 1, 3, 6, 10, 15 oder 21 Knoten
vollst\"{a}ndige Ans\"{a}tze besitzen k\"{o}nnen, s. Bild \ref{BildPascal}.
%--------------------------- --------------------------------------
\begin{figure}[tbp]
\if \bild 2 \sidecaption \fi
\includegraphics[width=.6\textwidth]{\Fpath/PHIMITTE}
\caption{Die Einheitsverformung $\Np_5$ des Balkens ist im mittleren Teil ein Polynom
dritten Grades und in den anderen Elementen identisch Null} \label{PhiMitte}
\end{figure}%
%-----------------------------------------------------------------

In der Balken- und Plattenstatik (schubstarre Balken und Platten) reicht es nicht, dass
die Biegeverformungen der Elemente stetig ineinander \"{u}bergehen, sondern es d\"{u}rfen auch
keine Knicke auftreten. Knicke w\"{a}ren Flie{\ss}gelenke. Mathematisch bedeutet dies, dass die
Einheitsverformungen der Knoten $C^1$-Funktionen sein m\"{u}ssen, Funktionen mit einer
stetigen ersten Ableitung. Die Tangente an die Biegelinie darf an keiner Stelle abrupt
ihre Neigung \"{a}ndern. Die Einheitsverformungen eines Balkenelements, die bekannten {\em
Hermite-Polynome\/}  \index{Hermite-Polynome} s. Bild \ref{PolyBalken}, sind aber gerade
so beschaffen, dass man mit ihnen die Durchbiegung und die erste Ableitung in den Knoten
interpolieren kann, und so kann man mit diesen kubischen Elementen Einheitsverformungen
ohne Knicke generieren, s. Bild \ref{PolyBalken}.

Nicht verschwiegen sei, dass es viele Formulierungen gibt, wie etwa die nichtkonformen
Elemente, die gegen diese Regeln versto{\ss}en und trotzdem funktionieren. So w\"{a}hlt man
z.Bsp. bei Reissner-Mindlin-Elementen f\"{u}r $w$ und die Verdrehung $\Np$ lineare Ans\"{a}tze,
obwohl dann in der Gleitung $\gamma = w' + \Np$ unterschiedliche Polynome stehen.

Polynome sind glatte Funktionen. Wenn man die $e$-Funktion\index{$e$-Funktion}
\bfoo
e^x = 1 + x + \frac{1}{2}\,x^2 + \frac{1}{3!}\,x^3 + \ldots \frac{1}{n!}\,x^n + \ldots
\efoo
differenziert, dann ist das so, als ob man die ganze Reihe um eins nach links {\em
shiftet\/}, was aber, weil es unendlich viele Terme sind, bedeutet, dass sich nichts
\"{a}ndert. Die Ableitung von $e^x$ ist $e^x$. Wenn wir ein Polynom differenzieren, dann
schneiden wir also vorne den konstanten Term ab und shiften den um einen Grad
erniedrigten Rest nach links. Bei Polynomen vom Grad $n$ gibt es nat\"{u}rlich nach $n+1$
Ableitungen nichts mehr zu {\em shiften\/}, aber die Botschaft ist: Die Ableitung eines
Polynoms ist ein Polynom.

Die Verzerrungen, die Spannungen, die Fl\"{a}chenkr\"{a}fte, die Volumenkr\"{a}fte in einem FE-Modell
sind praktisch die ersten Terme in imagin\"{a}ren Taylor-Reihen. Es sind in einfachen F\"{a}llen
elementweise Polynome\footnote{Wegen der Kettenregel, $u,_x = u,_\xi \, \xi,_x + \ldots $, k\"{o}nnen
es bei isoparametrischen Elementen aber auch rationale Funktionen sein.}.

Aber weil die Verschiebungsans\"{a}tze nur st\"{u}ckweise aus Polynomen bestehen -- auf jedem
Element beginnt praktisch eine neue Taylor-Reihe -- springen die Ableitungen zwischen
den Elementen, und das bedeutet, dass es zu Spannungsspr\"{u}ngen zwischen den Elementen
kommt.
%--------------------------- --------------------------------------
\begin{figure}[tbp]
\if \bild 2 \sidecaption \fi
\includegraphics[width=.7\textwidth]{\Fpath/STABEPS}
\caption{St\"{u}ckweise lineare Ansatzfunktionen beim Stab} \label{StabEps}
\end{figure}%
%-----------------------------------------------------------------

Im Sinne einer {\em backward error analysis\/}\index{backward error analysis} kann man
nun aus den Spannungen in den Elementen und den Spannungsspr\"{u}ngen zwischen den Elementen
auf die Belastung schlie{\ss}en, die eine solche Spannungsverteilung verursacht. So erh\"{a}lt
man die Lasten, die den FE-Lastfall $p_h$ ausmachen, s. Bild \ref{StabEps}, denn aus der
Form folgt die Kraft. Wenn man ein Seil \"{u}ber eine Rolle lenkt, dann ist der Anpressdruck
$p$ umgekehrt proportional dem Radius $R$ der Rolle
\bfoo
p = - H w'' = - H \frac{1}{R}\,.
\efoo
Oder: Wenn die FE-Biegelinie im Element die Gestalt $w_h(x) = (1 + 0.2\,x + 3\,x^2 -
5\,x^3 + 3\,x^5 - x^6)/ EI $ hat, dann wirkt auf das Element die Streckenlast
\bfoo
p_h(x) = EI\,w^{IV}_h(x) = 360\,(x - x^2)\,\,\, \,\mbox{kN/m}\,.
\efoo
Dazu kommen noch an den Elementgrenzen die Querkr\"{a}fte und die Momente, s. Bild
\ref{PolyEins}.
%--------------------------- --------------------------------------
\begin{figure}[tbp]
\if \bild 2 \sidecaption \fi
\includegraphics[width=1.0\textwidth]{\Fpath/POLYEINS}
\caption{Jedes Polynom ist im Gleichgewicht} \label{PolyEins}
\end{figure}%
%-----------------------------------------------------------------

Das Erstaunliche ist nun, dass die Kr\"{a}fte und Momente, die wir so einem Polynom zuordnen
k\"{o}nnen, nicht \hlq willk\"{u}rliche\grq\, Gr\"{o}{\ss}en sind, sondern wohl aufeinander abgestimmt,
denn es gilt:
\begin{itemize}
\item Jedes Polynom ist im Gleichgewicht.
\end{itemize}
Weil das so ist, ist jedes Element im Gleichgewicht und daher nat\"{u}rlich auch das ganze
Tragwerk. Dieser Schluss gilt f\"{u}r {\em alle\/} Elemente, ob nun Stab-, Balken-,
Scheiben- oder Plattenelement. Die Schnittkr\"{a}fte an den R\"{a}ndern eines Elements stehen
{\em immer\/} im Gleichgewicht mit den Kr\"{a}ften, die rechnerisch auf das Element wirken,
und es spielt dabei auch keine Rolle, ob das Element konform oder nichtkonform ist, da
Gleichgewicht am Element ja eine rein lokale Feststellung ist.

Der Beweis beruht auf der Ersten Greenschen Identit\"{a}t: F\"{u}r jede glatte Funktion $u$ --
also nicht nur Polynome -- ist $G(u,r) = 0$, wenn $r$ eine Starrk\"{o}rperbewegung darstellt.

Wir haben oben davon gesprochen, dass die FEM eine Interpolation mit {\em piecewise
polynomials\/} ist. Nur darf man sich dabei nicht \"{u}bereilen, denn es gilt: Nicht
interpolieren, sondern minimieren! Angenommen wir w\"{u}rden die Biegefl\"{a}che $w$ der Platte
in den Knoten des Netzes interpolieren,
\bfoo
w_I(\vek x) = w(\vek x_1)\,\Np_1 (\vek x) + w,_x(\vek x_1)\,\Np_2 (\vek x) + \ldots +
w,_y(\vek x_{3n})\,\Np_{3n} (\vek x)\,,
\efoo
so w\"{a}re diese Interpolierende $w_I$ aus der Sicht der Statik -- im Sinne des
Fehlerquadrates der Spannungen -- eine schlechtere L\"{o}sung als die FE-L\"{o}sung $w_h$! Sie
ist hinter der FE-L\"{o}sung nur zweiter Sieger,
\bfoo
\Pi(w) < \Pi(w_h) < \Pi(w_I)\quad \leftarrow \quad\mbox{Nicht so dicht an $w$ wie $w_h$}
\,.
\efoo
Ihr Abstand in der Energie zur exakten L\"{o}sung ist gr\"{o}{\ss}er als der Abstand der FE-L\"{o}sung!
Auf dieser Eigenschaft beruhen viele asymptotische Absch\"{a}tzungen f\"{u}r den Fehler, denn
wenn man wei{\ss}, dass der Interpolationsfehler auf dem Raum $\mathcal{V}_h$ von der Gr\"{o}{\ss}e
\bfoo
|| w - w_I||_m \leq h^{t - m} \,||w||_t
\efoo
ist, dann hat man damit automatisch auch eine Schranke f\"{u}r den Fehler der FE-L\"{o}sung,
denn sie ist ja auf jeden Fall noch dichter {\em  im Sinne der Energie\/} an der exakten
L\"{o}sung als die Interpolierende, ({\em C\'{e}as Lemma\/}).

F\"{u}r die Praxis bedeutet dies: Wenn wir das Gef\"{u}hl haben, dass wir auf dem gew\"{a}hlten Netz
die Biegefl\"{a}che der Platte gut interpolieren k\"{o}nnten, dann haben wir ein gutes Netz,
denn die Aufgabe, das Fehlerquadrat in den Schnittkr\"{a}ften zum Minimum zu machen, wird vom
FE-Programm mindestens so gut gel\"{o}st, wie die Interpolation der Biegefl\"{a}che.

{\small {\em Anmerkung: \/} Der Unterschied zwischen der Interpolierenden $w_I$ und der
FE-L\"{o}sung $w_h$ besteht nur in den Koeffizienten $w_i$, da ja die Ansatzfunktionen
$\Np_i $ dieselben sind. Bei der Interpolierenden sind die $w_i$ die Knotenwerte von
$w(x,y)$ w\"{a}hrend sie im Falle der FE-L\"{o}sung die L\"{o}sung des Gleichungssystems $\vek K\vek
w = \vek f$ sind, also so ausgew\"{a}hlt werden, dass der Abstand in der Energie zum Minimum
wird. W\"{a}re die Interpolierende $w_I$ eine bessere FE-L\"{o}sung, dann w\"{u}rde das System $\vek
K\vek w = \vek f$ genau die Knotenwerte der exakten L\"{o}sung liefern. Da dies aber bei 2D-
und 3D-Problemen nicht der Fall ist, muss die Interpolierende eine
schlechtere L\"{o}sung sein.} %% ENDSMALL
%--------------------------- --------------------------------------
\begin{figure}[tbp]
\if \bild 2 \sidecaption \fi
\includegraphics[width=.3\textwidth]{\Fpath/HKREIS}
\caption{Die Maschenweite $h$ in den asymptotischen Absch\"{a}tzungen ist der Durchmesser
des kleinsten Umkreises des gr\"{o}{\ss}ten Elements} \label{HKreis}
\end{figure}%
%-----------------------------------------------------------------

\subsubsection{Asymptotische Fehlerabsch\"{a}tzungen}\index{asymptotische Fehlerabsch\"{a}tzungen}
{\small Darunter versteht man Absch\"{a}tzungen f\"{u}r den Fehler einer FE-L\"{o}sung in
Abh\"{a}ngigkeit von der Maschenweite $h$ des FE-Netzes. Wir wollen das Thema hier nur kurz
streifen und auch nur kurz die grundlegenden Ideen skizzieren.

Die Taylor-Reihe einer Funktion
\bfoo
u(x) = \underbrace{u(0) + u'(0)\,x + \ldots +
u^{(n)}(0)\,\frac{x^n}{\vphantom{()}n!}}_{\mbox{Polynom $n$-ten Grades}} +
\underbrace{u^{(n+1)}(\xi)\frac{x^{(n+1)}}{(n+1)!}}_{\mbox{Restglied}}
\efoo
besteht aus einem Polynom $n$-ten Grades und einem Restglied, in dem die Ableitung
$u^{(n+1)}$ an einer unbekannten Stelle $\xi$ zwischen $0$ und $x$ steht.

Nun wenden wir diese Darstellung auf die exakte L\"{o}sung $u(x)$ in einem Element
$[x_i,x_{i+1}]$ der L\"{a}nge $h = x_{i+1} - x_i$ an. Brechen wir die Taylor-Reihe nach der
ersten Ableitung ab, dann gilt f\"{u}r einen Punkt $0 \leq x \leq h$
\bfoo
u(x) = u(x_i) + u'(x_i)\,x + u''(\xi)\frac{x^{2}}{2} \qquad x_i \leq \xi \leq x_{i+1}\,.
\efoo
Angenommen der FE-Ansatz besteht elementweise aus Polynomen ersten Grades, wie z.B. bei
linearen Stabelementen, und wir w\"{u}rden mit den Einheitsverformungen die exakte L\"{o}sung in
den Knoten interpolieren, dann folgt f\"{u}r den Fehler $e_I(x) = u(x) - u_I(x)$ der
Interpolierenden $u_I(x)$,
\bfo\label{Ab1}
e_I(x) &=& u(x_i) + u'(x_i)\,x + u''(\xi)\frac{x^{2}}{2} - u_I(x_i) - u_I'(x_i)\,x \nn \\
&=& e'(x_i)\,x + u''(\xi)\frac{x^{2}}{2} \qquad \mbox{wegen $u_I(x_i) = u(x_i)$}\,.
\efo
Weil auch der Fehler am anderen Ende des Elements Null ist, $u_I(x_{i+1}) = u(x_{i+1})$,
muss $e_I(x)$ irgendwo dazwischen an einer Stelle $s$ maximal sein, und somit gilt wegen
$e_I'' = u'' - u_I'' = u''$
\bfo\label{Ab2}
e_I'(x) = \int_s^{\,x} u''(z)\,dz \leq \int_{x_i}^{\,x_{i+1}} |u''(z)|\,dz \leq h\,
\max_{x_i \leq z \leq x_{i+1}} \,|u''(z)|\,.
\efo
Fasst man (\ref{Ab2}) und (\ref{Ab1}) zusammen, so erh\"{a}lt man die Absch\"{a}tzung
\bfo\label{Fehler13}
|e_I(x)| \leq h^2\, \max_{x_i \leq \xi \leq x_{i+1}} \,|u''(\xi)| \,.
\efo
Dieses Ergebnis d\"{u}rfen wir dahingehend verallgemeinern, dass wir sagen: Kann man mit den
Ansatzfunktionen der Elemente Polynome bis zum Grad $k$ exakt darstellen
(Vollst\"{a}ndigkeit!), und sind die Ableitungen der Ansatzfunktionen gleichm\"{a}{\ss}ig
beschr\"{a}nkt\footnote{\cite{Strang0} S. 137}, dann ist in der Scheibenstatik der Fehler
{\em der Interpolierenden\/} in den Verschiebungen von der Gr\"{o}{\ss}enordnung $O(h^{k+1})$
und der Fehler in den Spannungen von der Gr\"{o}{\ss}enordnung $O(h^k)$, und der konstante
Faktor in der Fehlerabsch\"{a}tzung, s. (\ref{Fehler13}), h\"{a}ngt von der oder den
Ableitung(en) $k+1$ der L\"{o}sung $u(x)$ ab.
%--------------------------- --------------------------------------
\begin{figure}[tbp]
\if \bild 2 \sidecaption \fi
\includegraphics[width=1.0\textwidth]{\Fpath/ASYMP}
\caption{Eine ung\"{u}nstige Anordnung der Elemente reduziert die Konvergenzordnung}
\label{ASymp}
\end{figure}%
%-----------------------------------------------------------------

Bei quadratischen Ans\"{a}tzen f\"{u}r Scheiben, $k = 2$, bedeutet das kubische Konvergenz der
Verschiebungen, $u_i = O(h^3)$, und quadratische Konvergenz, $O(h^2)$, bei den
Spannungen. Kubische Ans\"{a}tze bei schubstarren Balken oder Platten, $k = 3$, bedeuten
quartische Konvergenz der Durchbiegung, $O(h^4)$, quadratische Konvergenz, $O(h^2)$, bei
den Momenten und lineare Konvergenz, $O(h)$, bei den Querkr\"{a}ften. Mit jeder
Differentiationsstufe reduziert sich also die Ordnung der Konvergenz um Eins.

All dies gilt nat\"{u}rlich nur f\"{u}r glatte L\"{o}sungen. Sind Singularit\"{a}ten vorhanden oder ist
die L\"{o}sung nicht so glatt, wie angenommen, dann verringert sich die Konvergenzrate, weil
man in der Taylor-Reihe mit der letzten regul\"{a}ren Ableitung von $u$ aufh\"{o}ren muss.

Springt z.B. die Streckenlast im Element, wie in Bild \ref{ASymp} a, dann ist die dritte
Ableitung $u'''$ eine Delta-Funktion, und die Taylor-Reihe von $u(x)$ stoppt mit dem
Restglied $u''(\xi)$
\bfoo
u(x) = u(0) + u'(0)\,x + u''(\xi)\frac{x^2}{2} \qquad \leftarrow \qquad
\mbox{erzwungenes Ende}\,,
\efoo
und die obige Absch\"{a}tzung kommt nicht weiter als $|e_I| \leq h^2 \,\mbox{max}\,|u''|$.
Auch wenn wir quadratische Ans\"{a}tze nehmen, die theoretisch die Konvergenzordnung $O(h^3)$
haben, kommen wir \"{u}ber $O(h^2)$ nicht hinaus.

Greift gar eine Einzelkraft im Element an, dann ist schon die zweite Ableitung $u''$ eine
Delta-Funktion, und die Taylor-Reihe stoppt schon mit dem Restglied $u'(\xi)$
\bfoo
u(x) = u(0) + u'(\xi)\,x  \,,
\efoo
was bedeutet, dass wir \"{u}ber $O(h)$ nicht hinauskommen.

Die Regel lautet also: Sind die Elemente von der Ordnung $k$, und h\"{a}tten wir unter
regul\"{a}ren Umst\"{a}nden f\"{u}r den Interpolationsfehler die Absch\"{a}tzung
\bfoo
|e_I| \leq h^{k+1} \mbox{max\/}\,\, |u^{(k + 1)}|\,,
\efoo
dann verlieren wir mit jeder Differentiationsstufe, um die die L\"{o}sung schlechter ist,
eine Ordnung in der Konvergenzrate.

Diese Beispiele machen noch einmal deutlich, wie wichtig es ist, die Elemente so
anzuordnen, dass Einzelkr\"{a}fte in den Knoten angreifen und Unstetigkeiten der Belastung
auf Elementgrenzen liegen. Dasselbe gilt nat\"{u}rlich sinngem\"{a}{\ss} f\"{u}r Spr\"{u}nge in $E$-Modul
oder der Fl\"{a}che $A$, obwohl all das eigentlich selbstverst\"{a}ndlich ist.

Statt den Interpolationsfehler $e_I(x) = u(x) - u_I(x)$ in einzelnen Punkten des
Elements zu studieren, benutzt man in der Literatur meist integrale Ma{\ss}e f\"{u}r den Fehler
\bfoo
||e_I||_m = \left[\int_0^{\,l} [\,e_I^2 + (e_I')^2 + (e_I'')^2 + \ldots +
(e_I^{(m)})^2]\,dx\,\right]^{1/2}\,,
\efoo
also {\em Sobolev-Normen\/} vom Grade $m$, wobei $m$ typischerweise gleich der Ordnung
der Energie ist, die zu dem betreffenden Problem geh\"{o}rt
\bfoo
2\,m = \left\{ \barr {l l} 2 \qquad & \mbox{Stab, Scheibe, schubweiche Platten} \\
                           4 \qquad & \mbox{schubstarre Balken und Platten} \earr
                           \right.
\efoo
Aus (\ref{Fehler13}) folgt f\"{u}r den Interpolationsfehler in dieser Norm die Absch\"{a}tzung
\bfo\label{Fehler13A}
||e_I||_m = c\,h^{k+1 -m} \,||u||_{k+1}\,.
\efo
Die h\"{o}chsten Ableitungen, die in der Sobolev-Norm $||.||_m$ vorkommen, sind gerade von
derselben Ordnung, wie die Ableitungen in der Wechselwirkungsenergie $a(.,.)$, und es
gilt, dass unter regul\"{a}ren Umst\"{a}nden die Energienorm $\sqrt{a(u,u)}$ und die Sobolev-Norm
$||.||_m$ einander \"{a}quivalent sind,
\bfo
c_1\,||u||_m \leq \sqrt{a(u,u)} \leq c_2\,||u||_m \qquad \mbox{mit Konstanten $c_1$ und
$c_2$}\,.
\efo
Bei einem Bernoulli-Balken, $EI \,w^{IV}$, ist $2m = 4$, also $m = 2$ und die
Sobolev-Norm
\bfoo
||w||_2 = \left[\int_0^{\,l} (w^2 + (w')^2 + (w'')^2)\,dx \,\right]^{1/2}
\efoo
ist, wenn der Balken keine Starrk\"{o}rperbewegungen vollf\"{u}hren kann, mit der Energienorm
\bfoo
||w||_E := a(w,w)^{(1/2)} = \left[\int_0^{\,l} \frac{M^2}{EI}\,dx\,\right]^{1/2}
\efoo
\"{a}quivalent
\bfoo
c_1\,||w||_2 \leq ||w||_E \leq c_2\,||w||_2\,.
\efoo
Das ist schon bemerkenswert, weil die Energienorm ja nur die zweiten Ableitungen, die
Momente, misst, w\"{a}hrend die Sobolev-Norm auch die nullte und die erste Ableitung misst.
Wenn also das Moment in dem Balken und damit $||w|||_E$ Null ist, dann ist auch die
nullte und die erste Ableitung der Biegelinie Null (im $L_2$-Sinne).

Der Index $k+1$ in der Norm $||u||_{k+1}$ auf der rechten Seite von (\ref{Fehler13A})
kommt aus dem Restglied $u^{k+1}(\xi)$ der Taylor-Reihe. F\"{u}r andere Indices $r$ erh\"{a}lt
man f\"{u}r den Interpolationsfehler Absch\"{a}tzung der Art
\bfo\label{Absch16}
||e_I||_m = c\,h^{\alpha} \,||u||_{r} \qquad \alpha = \mbox{min}\,(k + 1 - m, r - m)\,,
\efo
wobei entweder der Grad des Elements ma{\ss}gebend ist -- je gr\"{o}{\ss}er $k$ ist, desto weiter
k\"{o}nnen wir in der Taylor-Reihe gehen und gewinnen so $h$-Potenzen -- oder die G\"{u}te der
L\"{o}sung -- sie steckt in dem Index $r$ -- erlaubt es nicht, so weit zu gehen, und wir
m\"{u}ssen vorher abbrechen. Das schw\"{a}chste Glied in der Kette bestimmt die m\"{o}gliche
Konvergenzrate.

Bisher haben wir nur \"{u}ber den Interpolationsfehler gesprochen, nicht \"{u}ber den Fehler der
FE-L\"{o}sung. Das ist aber dank {\em C\'{e}a's Lemma\/} nur ein kleiner Schritt, denn dieses
ber\"{u}hmte Lemma besagt, dass die FE-L\"{o}sung unter allen m\"{o}glichen L\"{o}sungen $v_h \in \mathcal{V}_h$
den kleinstm\"{o}glichen Abstand zur exakten L\"{o}sung in der Sobolev-Norm $||.||_m$ hat
\bfoo
||u - u_h||_m \leq c\,\underset{v_h \in \mathcal{V}_h}{\mbox{inf}}\,||u - v_h||_m \qquad
\mbox{C\'{e}as Lemma}\,.
\efoo
Die FE-L\"{o}sung $u_h$ ist also im Sinne der Energie dichter an der exakten L\"{o}sung, als jede
andere Funktion $v_h \in \mathcal{V}_h$ also insbesondere auch n\"{a}her an $u$ als die
Interpolierende. Weil das so ist, folgt sofort, s. (\ref{Absch16}), die grundlegende
Fehlerabsch\"{a}tzung: Es gilt n\"{a}mlich f\"{u}r den Fehler $e = u - u_h$ der FE-L\"{o}sung
\bfo\label{Absch16A}
||e||_m \leq\,c\,h^\alpha\,||u||_r \qquad \alpha = \mbox{min}\,(k + 1 - m,r - m)\,.
\efo
Der Beweis ist einfach
\bfoo
||e||_m &\leq& c_1^{-1}\,a(e,e)^{1/2} \leq c_1^{-1}\,a(u - u_I,u - u_I)^{1/2}\\
&\leq& \frac{c_2}{c_1} || u - u_I||_m \leq c_3\,h^\alpha\,||u||_r
\efoo
und benutzt -- in dieser Reihenfolge -- die \"{A}quivalenz der Energienorm mit der
Sobolev-Norm $||.||_m$, C\'{e}a's Lemma, wieder die \"{A}quivalenz der Normen und schlie{\ss}lich
die Absch\"{a}tzung (\ref{Absch16}) f\"{u}r den Fehler.

Um Absch\"{a}tzungen in niedrigeren Normen, $0 \leq s \leq m$, zu erhalten, benutzt man das
{\em Aubin-Nitsche-Lemma\/}, \cite{Braess1}, und erh\"{a}lt so
\bfoo
||e||_s = c\,h^{\alpha} \,||u||_{k+1} \qquad \alpha = \mbox{min}\,(k + 1 - s, 2\,(k + 1 -
m))\,.
\efoo
F\"{u}r $k = m = 1$ (z.B. lineare Ans\"{a}tze bei der Scheibe) erh\"{a}lt man z.B. f\"{u}r den Fehler $e$
der FE-L\"{o}sung die Ergebnisse
\bfoo
||e||_0 = c\,h^2\,||u||_2 \qquad ||e||_1 = c\,h\,||u||_2\,,
\efoo
die wieder stark an die Taylor-Reihe erinnern. Die Konstanten $c$ sind hier generische
Gr\"{o}{\ss}en, die nicht von $u$ und $h$ abh\"{a}ngen.

Oben bei dem eindimensionalen Beispiel des Stabes kam die Singularit\"{a}t aus der
Belastung. Bei Scheiben oder Platten treten Singularit\"{a}ten typischerweise in Eckpunkten
auf. Wenn wir ein Segeltuch \"{u}ber eine L-f\"{o}rmige \"{O}ffnung spannen (Vorspannkraft $H$),
dann kommt es unter Winddruck $p$ zum Anliegen des Segeltuchs an die einspringende Ecke,
und das Segeltuch rei{\ss}t, weil die Spannung
\bfoo
\sigma_n = H\,\nabla w \dotprod \vek n = H\,(w,_x \,n_x + w,_y\,n_y) \qquad \vek n =
\mbox{Schnittnormale}
\efoo
dort unendlich gro{\ss} wird, denn die Durchbiegung $w(\vek x)$ des Segeltuchs,
\bfoo
- H\,\Delta w = p \quad \mbox{in $\Omega$}\quad w = 0 \quad\mbox{auf $\Gamma$}\,,
\efoo
verl\"{a}uft dort sinngem\"{a}{\ss} wie
\bfoo
w = k_1\,s_1(r,\varphi) + w_R = k_1\,r^\alpha\,t(\varphi ) + w_R \quad \alpha =
90^\circ/360^\circ = 0.25 < 1\,,
\efoo
wobei $k_1$ der Spannungsintensit\"{a}tsfaktor ist, und $w_R$ den regul\"{a}ren Teil der L\"{o}sung
darstellt, also den Teil mit beschr\"{a}nkten Spannungen.

Sei nun $s_1^h$ die FE-Approximation der Singul\"{a}rfunktion $s_1$, dann erh\"{a}lt man f\"{u}r den
$L_2$-Fehler die Absch\"{a}tzung
\bfoo
||s_1 - s_1^h||_0 + h^\alpha \,|| \nabla (s_1 - s_1^h)||_0 \leq c\,h^{2\,\alpha}
\efoo
und f\"{u}r den punktweisen Fehler die Absch\"{a}tzung
\bfoo
||s_1 - s_1^h||_{L_\infty} \leq c\,h^\alpha\,.
\efoo
\"{A}hnlich wie oben, bestimmt also die Singularit\"{a}t in der L\"{o}sung die maximal m\"{o}gliche
Konvergenzrate, und selbst mit quadratischen oder kubischen Elementen kann man die Rate
auf diesem einfachen Wege nicht verbessern, \cite{Blum}.
} % ENDE SMALL

%%%%%%%%%%%%%%%%%%%%%%%%%%%%%%%%%%%%%%%%%%%%%%%%%%%%%%%%%%%%%%%%%%%%%%%%%%%%%%%%%%%%%%%%%%%%%%%%%%%%%

%%%%%%%%%%%%%%%%%%%%%%%%%%%%%%%%%%%%%%%%%%%%%%%%%%%%%%%%%%%%%%%%%%%%%%%%%%%%%%%%%%%%%%%%%%%
\section{Das Prinzip von St. Venant}\label{Das Prinzip von St. Venant}\index{Das Prinzip von St. Venant}
%%%%%%%%%%%%%%%%%%%%%%%%%%%%%%%%%%%%%%%%%%%%%%%%%%%%%%%%%%%%%%%%%%%%%%%%%%%%%%%%%%%%%%%%%%%
Das {\em Prinzip von St. Venant\/} besagt, dass es ab einer gewissen Entfernung vom Rand
nicht mehr m\"{o}glich ist, zu unterscheiden, ob die Kraft als konzentrierte Einzellast, als
Linienlast oder als Fl\"{a}chenlast eingeleitet wurde.

Dies bedeutet auch, dass lokal eng begrenzte Gleichgewichtsgruppen, die am Rand eines
Tragwerks angreifen in ihrer Wirkung auf das Tragwerk -- bis auf die St\"{o}rzone am Rand
selbst -- vernachl\"{a}ssigt werden k\"{o}nnen. F\"{u}r die finiten Elemente bedeutet es, dass die
Spannungen in Elementmitte am genauesten sind. Die Methode der Randelemente profitiert
sehr stark vom Prinzip von St. Venant, weil dort die Fehler nur auf dem Rand auftreten.
Man kann diesen positiven Aspekt aber auch ins Negative kehren: Ein Programm zu
schreiben und dann als {\em benchmark} nur die Durchbiegung in Plattenmitte zu benutzen,
w\"{a}re fahrl\"{a}ssig.

Das Prinzip von St. Venant gilt f\"{u}r sogenannte {\em elliptische
Differentialgleichungen}. Die meisten Gleichungen der Statik f\"{u}r ruhende bzw. harmonisch
wechselnde ($p(t) = \sin\,t $) Belastung sind von diesem Typ. L\"{o}sungen elliptischer
Gleichungen klingen au{\ss}erhalb des Einflussbereiches der Belastung rasch ab.

Die G\"{u}ltigkeit des {\em Prinzips von St. Venant\/} kann man praktisch an Hand der
Einflussfunktionen verifizieren. So lautet etwa die Einflussfunktion f\"{u}r das Moment
$m_{xx}$ einer gelenkig gelagerten Platte
\bfo\label{InfMxxEx}
m_{xx}(\vek x)  &= &\int_{\Gamma}[\,g_2 v_\nu +  m_\nu (g_2 ) \frac{\partial w}{\partial
\nu}
]\,ds_{\vek y}  + \int_{\Omega}  g_2 p\,d\Omega_{\vek y}  \nn \\
&+& \sum_c \,g_2(\vek y^c)F_c\,.
\efo
In diese Einflussfunktion gehen die Randkr\"{a}fte $v_\nu$, die Randverdrehung $\partial
w/\partial \nu$, die Belastung $p$ im Gebiet und die Eckkr\"{a}fte $F_c$ ein, und die
Einfl\"{u}sse pflanzen sich in etwa logarithmisch bzw. wie $r^{-2}$ fort
\bfoo
g_2(\vek y,\vek x) = O(\ln\,r) \qquad m_\nu(g_2(\vek y,\vek x)) &=& O(r^{-2})\,.
\efoo
Macht man dasselbe mit der FE-L\"{o}sung, so erh\"{a}lt man f\"{u}r das Moment $m_{xx}^h$ die
Darstellung
\bfo\label{InfMxx}
m_{xx}^h(\vek x)  &= &\int_{\Gamma}\underbrace{[\,g_2 v_\nu^h +  m_\nu (g_2 )
\frac{\partial w^h}{\partial \nu}
]}_{Einfluss\,\, der\,\, Randwerte} \,ds_{\vek y} \nn \\
 &+& \underbrace{\sum_e \int_{\Omega_e}  g_2 p_h\,d\Omega_{\vek y}
+ \sum_i \int_{\Gamma_i} [\,g_2 \, v_h^\Delta - \frac{\partial { }}{\partial \nu}
\,g_2 \, m_h^\Delta] \,ds_{\vek y}}_{Einfluss\,\,der\,\,Quellen\,\, im\,\, Gebiet} \nn \\
&+& \underbrace{\sum_{k = 1}^n g_2(\vek y_k)\,F_k^h}_{Einfluss\,\,der\,\,Knotenkraefte}
+ \underbrace{\sum_c \,g_2(\vek y^c)F_c^h}_{Einfluss\,\,der\,\,Eckkraefte}
\efo
In diese Einflussfunktion gehen die Elementlasten $p_h$, die \hlq Querkraftspr\"{u}nge\grq\,
$v_h^\Delta$, die Momentenspr\"{u}nge $m_h^\Delta$, die Knotenkr\"{a}fte $F_k^h$ (aus den
Unstetigkeiten der Momente $m_{xy}^h$) und die Eckkr\"{a}fte $F_c^h$ ein, die den Lastfall
$p_h$ ausmachen. Die Einflussfunktion (\ref{InfMxx}) ist eine {\em quellenm\"{a}{\ss}ige\/}
Darstellung von $m_{xx}^h$, denn sie beschreibt $m_{xx}^h$ von den Quellen, von den
Ursachen, her.

Sie sieht sehr kompliziert aus. Man muss sich aber immer wieder vor Augen halten, dass
diese Darstellung elementweise nichts anderes liefert als die Momente, die man durch
Differentiation der Einheitsverformungen erh\"{a}lt, also etwa
\bfoo
m_{xx}^h = 3.14 + 2.72\,x + 9.81\,y = \int_{\Gamma} [\,g_2 \,v_\nu^h - \frac{\partial {
}}{\partial \nu} g_2 \,m_\nu^h - \ldots
\efoo
Links steht eine (nicht mehr sichtbare) Entwicklung von $m_{xx}^h$ nach den
Knotenverformungen des betreffenden Elements. Rechts steht eine globale Darstellung
derselben Funktion in die alles eingeht, was nur irgendwie auf die Platte einwirkt:
Elementlasten, Linienlasten, Randlasten und Randverformungen. Beide Darstellungen, die
lokale, wie die globale liefern dasselbe Ergebnis!

Subtrahiert man die beiden Ausdr\"{u}cke (\ref{InfMxxEx}) und (\ref{InfMxx}), so erh\"{a}lt man
eine Darstellung
\bfoo
m_{xx}(\vek x) - m_{xx}^h(\vek x) = \int_{\Gamma}[\,g_2 (v_\nu - v_\nu^h) + m_\nu(g_2)
\frac{\partial { }}{\partial \nu} (w - w_h)\,]\,ds_{\vek y} - \ldots
\efoo
des Fehlers der FE-L\"{o}sung. Man kann den Fehler zwar nicht ausrechnen, weil man die
Lagerkraft $v_\nu$ und die Neigung des Plattenrandes, $\partial w/\partial \nu$, nicht
kennt, aber man kann ablesen, wie sich die Fehler fortpflanzen, denn dies h\"{a}ngt ja von
den Kernen ab
\bfoo
g_2 &=& O(\ln\,r)\nn = \mbox{Biegefl\"{a}che aus $M_x = 1$ im Punkt $\vek x$}  \\
\frac{\partial { }}{\partial \nu} \,g_0 &=& O(r^{-1}) = \mbox{zugeh\"{o}rige Verdrehung des Randes} \nn \\
m_\nu (g_2 ) &=& O(r^{-2}) = \mbox{zug. Einspannmoment auf dem Rand} \nn \\
v_\nu(g_0) &=& O(r^{-3}) = \mbox{zug. Kirchhoffschub} \,,\nn
\efoo
die alle sehr rasch abklingen. Der Logarithmus $\ln\,r$ steigt erst f\"{u}r sehr gro{\ss}e $r$
wieder an, $\ln\,100 = 4.6$.

Das Abklingverhalten der Kerne ist der eine Effekt, auf dem das Prinzip von St. Venant
beruht. Der andere Effekt ist die Mittelung, die durch die Integration (meist)
stattfindet, wie in dem folgenden Beispiel: Die Lagerkraft $A$ des Kragtr\"{a}gers in Bild
\ref{Kernelb14} ist die \"{U}berlagerung der Einflussfunktion $\eta_A$ mit der Belastung $p$
\bfoo
A = \int_0^{\,l} \eta_A(x) \,p(x) \,dx\,.
\efoo
%----------------------------------------------------------------------------------------------------------
\begin{figure}[tbp]
\if \bild 2 \sidecaption \fi
\includegraphics[width=.8\textwidth]{\Fpath/KERNELB}
\caption{{\bf a)} Einflussfunktion f\"{u}r die Lagerkraft $A$, {\bf b)} Nur der Mittelwert
der Last ist wichtig, {\bf c)} Antimetrische Lasten sind orthogonal zum Kern der
Einflussfunktion} \label{Kernelb14}
\end{figure}%%
%----------------------------------------------------------------------------------------------------------
Nun ist $\eta_A(x) = 1$ identisch Eins. Was als Lagerkraft im Lager A ankommt, ist also
immer das {\em  Mittel\/} der Streckenlast $p$
\bfoo
A = \int_0^{\,l} 1 \times p\,\,dy = p_{\,\mbox{\small Mittel}} \times l\,.
\efoo
Der Kern $\eta_A = 1$ l\"{o}scht alle zur Balkenmitte $x = l/2$ antimetrischen \hlq
Oberschwingungen\grq \, einer Lastverteilung $p$ aus, s. Bild \ref{Kernelb14} c. Sie sind
im Sinne des $L_2$-Skalarprodukts {\em orthogonal\/} zu dem Kern $\eta_A$.

Manches kann man also durch die Wahl der {\em sample points\/}, der Punkte in denen man
die Schnittkr\"{a}fte ausrechnet, beeinflussen. Die {\em Superkonvergenz\/} in den
Gausspunkten beruht (wahrscheinlich) auf solchen Eigent\"{u}mlichkeiten der Kerne, dass sie
gewisse Fehler \hlq wegmitteln\grq.

Beispiele von oszillierenden Lagerkr\"{a}ften oder Einspannmomenten bei einer Platte gibt es
die Menge. Verfeinert man das Netz, dann werden diese Oszillationen in der Regel noch
gr\"{o}{\ss}er. Tr\"{o}stlich ist es dann zu wissen, dass -- vereinfacht gesagt -- in einigem
Abstand vom Rand nur der Mittelwert dieser oszillierenden Lagerkr\"{a}fte die Durchbiegung
der Platte, die Feldmomente etc. beeinflusst. {\em Peaks\/} sind also nicht ganz so
dramatisch, wie sie aussehen. Das ist nat\"{u}rlich etwas naiv formuliert, aber es soll nur
zum Ausdruck bringen, dass sich Oszillationen ja nicht von alleine fortpflanzen,
sondern, dass sie zur Fortpflanzung das Vehikel Einflussfunktion brauchen, d.h. es wird
\"{u}ber sie hinweg integriert.
%%%%%%%%%%%%%%%%%%%%%%%%%%%%%%%%%%%%%%%%%%%%%%%%%%%%%%%%%%%%%%%%%%%%%%%%%%%%%%%%%%%%%%%%%%%%%%%%%%%%%

{\small
\subsubsection{Singul\"{a}rfunktionen}
Wenn man wei{\ss}, wie die singul\"{a}ren Anteile einer L\"{o}sung aussehen, dann kann man den
Ansatzraum $\mathcal{V}_h$ um diese Anteile erweitern. So hat z.B. die Biegefl\"{a}che einer
schubstarren Platte in der N\"{a}he einer St\"{u}tze die Gestalt
\bfoo
w(\vek x) = k_1\,\frac{1}{8\,\pi\,K}\,r^2\,\ln\,r + w_R(\vek x)\,,
\efoo
wobei der {\em Spannungsintensit\"{a}tsfaktor\/} die Kraft $P = k_1$ in der St\"{u}tze ist, und
$w_R$ repr\"{a}sentiert das glatte \hlq Restglied\grq.

Ebenso wei{\ss} man, dass in der N\"{a}he einer Ecke, sich die L\"{o}sung der Kirchhoffschen
Plattengleichung meist wie folgt darstellen l\"{a}sst, \cite{Melzer},
\bfoo
w = \sum_{z_\mu \in \,\mathbb{Z}} k_\mu\,r^{z_\mu} \psi_\mu(\varphi) + w_R\,.
\efoo
Die komplexen Exponenten $z_\mu$ in den Singul\"{a}rfunktionen
\bfoo
s_\mu(r,\varphi ) = k_\mu\,r^{z_\mu} \psi_\mu(\varphi)
\efoo
sind die Eigenwerte gewisser Matrizen. Die Eigenwerte h\"{a}ngen von der Gr\"{o}{\ss}e des
Eckenwinkels ab und den Randbedingungen links und rechts von der Ecke. Die Konstanten
$k_\mu$ sind die Spannungsintensit\"{a}tsfaktoren.

Die {\em Methode der Singul\"{a}rfunktionen\/} besteht nun darin, dass man den Ansatzraum
$\mathcal{V}_h$ um diese Singul\"{a}rfunktionen erweitert, also f\"{u}r die FE-L\"{o}sung den Ansatz
\bfoo
w_h(\vek x) = \sum_j k_j s_j(\vek x) + \sum_i \,w_i\,\Np_i(\vek x)
\efoo
macht, und man die FE-L\"{o}sung nun auch mit den Singul\"{a}rfunktionen testet. Man erh\"{a}lt so
ein symmetrisches Verfahren, wo Ansatz- und Testraum zusammenfallen. Technisch sind
hierbei aber noch eine Reihe von Problemen zu \"{u}berwinden, s. \cite{Strang0}.
Insbesondere m\"{u}ssen die Singul\"{a}rfunktionen noch abgeschnitten werden, damit sie zu $\mathcal{V}_h$
passen.

Bei der {\em Methode der dualen Singularit\"{a}tsfunktionen\,} wird dagegen der Ansatzraum
um die Singul\"{a}rfunktionen $s_i$ erweitert und der Testraum um die dualen
Singul\"{a}rfunktionen $s_{-i}$ erweitert. Dazu nehmen wir im folgenden an, dass die exakte
L\"{o}sung die Gestalt
\bfoo
w = s_1(r,\varphi) + w_R \qquad s_1(r,\varphi ) = k_1 r^{1.5} \psi_1(\varphi)
\efoo
habe. Wir f\"{u}hren nun eine zu $s_1$ {\em duale Singul\"{a}rfunktion\/} $s_{-1}(r,\varphi)$
ein. Das ist eine Singul\"{a}rfunktion, deren Exponent am Index $m - 1$ ( = 1) gespiegelt
wird\footnote{Ist $s_1 = r^\alpha$, dann ist $s_{-1} = r^{m - \alpha}$ mit $2m = 4$},
und die von derselben Winkelfunktion $\psi_1(\varphi)$ begleitet wird wie $s_1$
\bfoo
s_{-1}(r,\varphi ) = r^{0.5} \psi_1(\varphi)\,.
\efoo
Dieser \hlq Versatz\grq\, in den Exponenten h\"{a}ngt mit der Zweiten Greenschen Identit\"{a}t
zusammen, denn in den Randintegralen stehen dort immer Produkte konjugierter Gr\"{o}{\ss}en, so
dass die Summe der Ableitungen immer gerade $2m - 1 = 3$ betr\"{a}gt. Angewandt auf das Paar
$s_1, s_{-1}$ bedeutet das also -- wir vereinfachen jetzt stark --
\bfoo
&&\int_{\Gamma} \left[ \ldots \frac{d^3}{dr^3} s_1\, s_{-1} + \frac{d^2}{dr^2}
s_1\,\frac{d}{dr}
s_{-1}\,\ldots \right]\,r\,d\varphi\\
 &=& [\ldots + r^{-1.5}\,r^{0.5} + r^{0.5}\,r^{-1.5} +
\ldots ]\,r = O(1)
\efoo
und dieses $O(1)$ macht, dass beim Grenz\"{u}bergang, wenn der Radius $\varepsilon$ des
gelochten Gebiets gegen Null schrumpft, s. Kapitel \ref{Greensche Identitaeten}, gerade
der Spannungsintensit\"{a}tsfaktor $k_1$ herausspringt.

So wie es Greensche Funktionen $G_0[\vek x]$ f\"{u}r Durchbiegungen gibt,
\bfoo
w(\vek x) = \int_0^l G_0(\vek y,\vek x)\,p(\vek y)\,d\Omega_{\vek y}
\efoo
so gibt es auch eine Greensche Funktion $G_k$ f\"{u}r den {\em stress intensity factor\/} in
der Ecke $\vek x$ einer Platte
\bfo\label{InfK}
k = \int_{\Omega} G_k(\vek y)\,p(\vek y)\,d\Omega\,,
\efo
und der f\"{u}hrende Term in dem Kern $G_k(\vek y,\vek x) = s_{-1} + v_R$ ist gerade die
duale Singul\"{a}rfunktion $s_{-1}(r,\varphi)$. Sie ist gleich dem
Spannungsintensit\"{a}tsfaktor $k$ an der Spitze eines unendlich langen Keils mit demselben
Eckenwinkel wie die Platte, wenn in einem r\"{u}ckw\"{a}rtigen Punkt $\vek y = (r,\varphi )$ eine
Kraft $P = 1$ steht.

Nat\"{u}rlich kennt man den regul\"{a}ren Teil $v_R$ der Greenschen Funktion $G_k$ nicht, und
daher muss man den Spannungsintensit\"{a}tsfaktor $k$ iterativ berechnen. Hierbei macht man
von der Integraldarstellung (Satz von Betti)
\bfoo
k = \int_{\Omega} s_{-1} \,p\,d\Omega_{\vek y} - \int_{\Omega} w \,\Delta \Delta \,s_{-1}
+ \ldots
\efoo
Gebrauch und setzt f\"{u}r die exakte L\"{o}sung $w$ die FE-L\"{o}sung $w_h$ und berechnet so einen
N\"{a}herungswert $k_h$, den man dann wieder in den FE-Ansatz steckt, das Plattenproblem neu
l\"{o}st, etc. Diese Iteration l\"{a}sst sich sogar ganz umgehen, \cite{Blum}.
} %Ende small
%%%%%%%%%%%%%%%%%%%%%%%%%%%%%%%%%%%%%%%%%%%%%%%%%%%%%%%%%%%%%%%%%%%%%%%%%%%%%%%%%%%%%%%%%%%%%%%%%%%%%

\Pi(w) &=& \frac{1}{2}\,\int_0^{\,l} \frac{M^2}{EI}\,dx - \int_0^{\,l} p\,w\,dx\\
&&< \Pi(w_h) = \frac{1}{2}\,\int_0^{\,l} \frac{M_h^2}{EI}\,dx - \int_0^{\,l}
p\,w_h\,dx\,.
\efoo
Indem man nun beobachtet, wie mit feiner werdenden Unterteilungen des Netzes $\Pi(w_h)$
gegen $\Pi(w)$ konvergiert, kann man die Konvergenz der FE-Berechnung verfolgen.

Da nat\"{u}rlich die exakte Energie $\Pi(w)$ unbekannt ist, schlie{\ss}t man, dass $w_h$ im
Sinne der Energie nahe an der exakten L\"{o}sung $w$ liegt, wenn sich die potentielle
Energie kaum noch \"{a}ndert.

Die Energie der FE-L\"{o}sung ist, wie man zeigen kann, gerade das Skalarprodukt aus dem
Vektor $\vek f$ der Knotenkr\"{a}fte und dem Vektor $\vek u$ der Knotenverformungen
($\times$ -\,0.5)
\bfoo
\Pi(w_h) = \frac{1}{2}\,\int_0^{\,l} \frac{M_h^2}{EI}\,dx - \int_0^{\,l} p\,w_h\,dx =
-\frac{1}{2}\,\int_0^{\,l} p_h\,w_h\,dx = -\frac{1}{2}\,\vek f^T\,\vek u\,,
\efoo
so dass ein FE-Programm die Konvergenz sehr leicht kontrollieren kann. Greift z.B. in
der Mitte einer Platte eine Einzelkraft $P = 50$ kN an, dann sind alle $f_i$, bis auf
den belasteten Knoten, Null und daher folgt
\bfoo
\Pi(w_h) = - \frac{1}{2}\,P \,w_i\qquad \mbox{$w_i$ = Durchbiegung des Lastknotens} \,.
\efoo
Jetzt braucht man nur die Werte $w_i$ f\"{u}r verschiedene Maschenweiten auftragen, und man
erh\"{a}lt ein Bild von der Konvergenz der FE-L\"{o}sung. So kann man im Grunde alle Netze
testen.

Die Konvergenz muss nicht monoton sein, weil die Einheitsverformungen $\Np_i$, die zu
verschieden feinen Netzen geh\"{o}ren, nicht unbedingt eine echt geschachtelte Folge $\mathcal{V}_1
\subset \mathcal{V}_{0.5} \subset \mathcal{V}_{0.25} \ldots$ von Ansatzr\"{a}umen $\mathcal{V}_h$ erzeugen, wo der gr\"{o}bere
und kleinere Raum im feineren und gr\"{o}{\ss}eren Raum echt enthalten ist. Das ist nur bei der
$p$-Methode garantiert, wenn man also den Polynomgrad stufenweise erh\"{o}ht, \cite{Szabo}.

Bei nichtkonformen Elementen erlebt man meist, dass die Durchbiegung der FE-L\"{o}sung {\em
gr\"{o}{\ss}er\/} als der exakte Wert ist, und auch hier kann es passieren, dass die Konvergenz
nicht monoton ist, wie man an dem Beispiel in Tabelle \ref{TabwKon} ablesen kann.
%--------------------------------------------------------------------------------------
\begin{table}
\caption{ Konvergenz der Durchbiegung $w$ [mm] in Plattenmitte} \label{TabwKon}
\begin{tabular}{rrrrr}
\noalign{\hrule\smallskip} \\[1mm]
      Netz & FEM  & & BEM             \\
\noalign{\hrule\smallskip}\\[1mm]
     4 x 4 &      0.722 &$\downarrow$    &  0.620 &$\downarrow$      \\
     8 x 8 &        0.700 & $\downarrow$ & 0.624  &$\downarrow$       \\
   16 x 16 &      0.689   &$\downarrow$  &0.625   &$\downarrow$     \\
   32 x 32 &       0.690  &$\uparrow$  &0.626   &$\downarrow$      \\[3mm]
\noalign{\hrule\smallskip}
\end{tabular}
 \end{table}
%--------------------------------------------------------------------------------------
Dabei handelt es sich um eine gelenkig gelagerte Platte der Abmessungen 4 m $ \times $ 3
m mit $d = 0.2$ m, $E = 30\,000$ MN/m$^2$, $\nu = 0.2$, die zentrisch mit einer Kraft $P = 100$ kN
belastet wurde.

Die FEM-Ergebnisse beruhen auf einem nichtkonformen Reissner-Mindlin Element, w\"{a}hrend die
BEM-Ergebnisse f\"{u}r eine schubstarre Platte gelten. Deswegen sind die Ergebnisse nicht
direkt vergleichbar. Die Konvergenz bei der BEM ist hier monoton. Sie verh\"{a}lt sich wie
eine konforme FE-L\"{o}sung. Man kann f\"{u}r die BEM aber keine generelle Regel wie f\"{u}r
konforme FE-Elemente angeben, weil die BEM auf dem {\em Satz von Betti\/} beruht, und
das \hlq mengentheoretische\grq \,  Argument $\mathcal{V}_h \subset V$ somit nicht zur Verf\"{u}gung
steht.

Mit diesem Einzelkrafttest kann man also leicht herausfinden, mit was f\"{u}r Elementen man
es zu tun hat. Da in der Regel Plattenprogramme nichtkonforme Elemente verwenden, wird
man meist das obige Muster wiederfinden, wenn man das Netz verfeinert.

Bei Scheiben kann man nat\"{u}rlich wegen der unendlich gro{\ss}en Verschiebungen und der
unendlich gro{\ss}en inneren Energie bei Einzelkr\"{a}ften, $|\vek u(\vek x)| = \infty$, die
Eigenschaften eines Netzes nicht mit Einzelkr\"{a}ften kontrollieren.

%%%%%%%%%%%%%%%%%%%%%%%%%%%%%%%%%%%%%%%%%%%%%%%%%%%%%%%%%%%%%%%%%%%%%%%%%%%%%%%%%%%%%%%%%%%%%%%%%%%%%

{\textcolor{blue}{\subsection{CST-Element}}}\index{CST-Element}
Das einfachste Scheibenelement ist ein
dreiecksf\"{o}rmiges Element mit drei Knoten und einem {\em linearen Ansatz} f\"{u}r die
horizontalen und vertikalen Verschiebungen, s. Bild \ref{SElemente} a,
\bfo \label{LinearA}
u(x,y) &=& a_0 + a_1 x + a_2 y \nn \\
v(x,y) &=& b_0 + b_1 x + b_2 y \nn \,.
\efo
%----------------------------------------------------------------------------------------------------------
\begin{figure}[tbp]
\if \bild 2 \sidecaption \fi
\includegraphics[width=1.0\textwidth]{\Fpath/SELEMENTE}
\caption{Scheibenelemente, {\bf a)} CST-Element, {\bf b)} bilineares Element}
\label{SElemente}
\end{figure}%%
%----------------------------------------------------------------------------------------------------------
Diese Darstellungen von $u$ und $v$ mit allgemeinen Koeffizienten $a_i$ und $b_i$ eignen
sich f\"{u}r die Praxis aber nicht. Statt dessen konstruiert man zun\"{a}chst f\"{u}r jeden Knoten
$i$ eine Formfunktion $\psi_i^e(x,y)$, die in dem
Knoten $i$ den Wert Eins hat und in allen anderen Knoten $j \neq i$ den Wert Null
\bfoo
\psi_1^e(x,y) &=& \frac{1}{2\,A_e}\, [(x_2\,y_3 - x_3\,y_2) + y_{23}\,\underset{\uparrow}{x} + x_{32}\,\underset{\uparrow}{y}\, ]\\
\psi_2^e(x,y) &=& \frac{1}{2\,A_e}\, [(x_3\,y_1 - x_1\,y_3) + y_{31}\,x + x_{13}\,y\, ]\\
\psi_3^e(x,y) &=& \frac{1}{2\,A_e}\, [(x_1\,y_2 - x_2\,y_1) + y_{12}\,x + x_{21}\,y\, ] \\[0.3cm]
2\, A_e &=& x_{21}\,y_{31} - x_{31} \,y_{21} \,,
\efoo
wobei $x_{\,ij} = x_i - x_j$ und $y_{\,ij} = y_i - y_j$ Differenzen der Eckkoordinaten
$(x_i,y_i)$ sind. Es ist dann ein leichtes, die Ans\"{a}tze f\"{u}r $u^e$ und $v^e$ nach den  $n$
horizontalen und $n$ vertikalen Knotenverschiebungen $u_i, v_i$ zu entwickeln
\bfoo
u^e(x,y) = \sum_{i=1}^3 u_i\,\psi_i^e(x,y)\,,\qquad v^e(x,y) = \sum_{i=1}^3
v_i\,\psi_i^e(x,y)\,.
\efoo
Gesamthaft entsteht so ein Verschiebungsfeld, das eine Entwicklung nach den sechs
Knotenverformungen des Elements ist
\bfoo
\vek u^e(x,y) = \left[ \barr {c} u^e \\ v^e \earr \right] = \left[ \barr {c c c c c c}
\psi_1^e & 0 &\psi_2^e & 0 &\psi_3^e &0\\ 0
&\psi_1^e & 0 &\psi_2^e &0 &\psi_3^e\earr \right] \,\left[\barr {c} u_1 \\
v_1 \\ u_2 \\ v_2 \\ u_3 \\ v_3 \earr \right] \quad \mbox{oder} \quad \vek u^e = \vek
\Psi^e\,\vek d^e\,.
\efoo
Bei diesen Bewegungen sind die Verzerrungen und Spannungen im Element konstant, {\em
constant strain triangle, (CST-Element)\/}\index{constant strain triangle}
\bfoo
\left[ \barr {c} \varepsilon_{xx} \\ \varepsilon_{yy} \\ \gamma_{xy}  \earr\right] =
\frac{1}{2\,A_e} \left[ \barr {c c c c c c} y_{23} &0 &y_{31} &0 &y_{12} &0 \\
0 & x_{32} &0 &x_{13} &0 &x_{21} \\
x_{32} &y_{23} &x_{13} &y_{31} &x_{21} &y_{21} \earr \right] \left[ \barr {c} u_1 \\ v_1
\\u_2 \\v_2 \\u_3 \\v_3 \earr \right] \quad \mbox{oder} \quad \vek \varepsilon = \vek
B\,\vek d\,.
\efoo
Mit der Matrix $\vek E$ aus (\ref{MatrixE41}) ergeben sich die durch die Knotenbewegungen
induzierten Spannungen $\vek \sigma = \{\sigma_{xx},\sigma_{yy},\tau_{xy}\}^T$ zu $\vek
\sigma = \vek E\,\vek \varepsilon  = \vek E\,\vek B\,\vek d$, und somit lautet die
Steifigkeitsmatrix des Elements, s. Kapitel
\ref{Steifigkeitsmatrizen},\index{Steifigkeitsmatrizen, Scheiben}
$$
\vek K^e = t\,\int_{\Omega_e} \vek B^T\,\vek E\,\vek B \,d\Omega \qquad t = \mbox{St\"{a}rke
des Elements}\,.
$$
Ihre Elemente $k^e_{\,ij}$ sind die
Wechselwirkungsenergien\index{Wechselwirkungsenergie} zwischen den Verschiebungsfeldern
$\vek \Np_i^e$ und $\vek \Np_j^e$, den Einheitsverformungen der Elementknoten,
\bfoo
k^e_{\,ij} &=& a(\vek \Np_i^e,\vek \Np_j^e) = t\,\int_{\Omega_e} \vek \sigma^{(i)} \dotprod \vek \varepsilon^{(j)}\,d\Omega\\
 &=& t \,\int_{\Omega_e} \left[\sigma_{xx}^{(i)}\varepsilon_{xx}^{(j)}
 + \tau_{xy}^{(i)}\gamma_{xy}^{(j)} + \sigma_{yy}^{(i)}\varepsilon_{yy}^{(j)} \right] \,d\Omega\,.
\efoo
%----------------------------------------------------------------------------------------------------------
\begin{figure}[tbp]
\if \bild 2 \sidecaption \fi
\includegraphics[width=.6\textwidth]{\Fpath/COOK47}
\caption{Kragtr\"{a}ger mit Endmoment, Berechnung mit CST-Elementen. Die Normalspannung in
der Tr\"{a}gerachse springt hin und her} \label{Cook47}
\end{figure}%%
%----------------------------------------------------------------------------------------------------------
Das CST-Element ist das einfachste Scheibenelement, aber sicherlich nicht das beste, wie
man an Bild \ref{Cook47} erkennt. Um die Biegespannungen richtig zu erfassen, m\"{u}sste man
schon stark verfeinern. Dieses Element ist nur bei regelm\"{a}{\ss}igen Netzen mit geringen
Spannungsgradienten sinnvoll einsetzbar.

{\textcolor{blue}{\subsection{Bilineares Element}}}\index{bilineares Element}\index{Q4}
Der einfachste Ansatz f\"{u}r ein rechteckiges Element
 mit vier Knoten (Q4) ist ein {\em bilinearer Ansatz} f\"{u}r die Verschiebungen
\bfo
u(x,y) &=& a_0 + a_1 x + a_2 y + a_3 x \,y \nn \\
v(x,y) &=& b_0 + b_1 x + b_2 y + b_3 x\, y \nn \,.
\efo
Er hei{\ss}t bilinear, weil die Ausdr\"{u}cke Produkte zweier linearer Polynome, $(c_1 + c_2\,x)
\,(d_1 + d_2\,y)$ sind. Die Formfunktionen der vier Knoten lauten, wenn $a$ und $b$ die
L\"{a}nge und H\"{o}he des Elements bezeichnen, wie folgt
\begin{alignat}{2}
\psi_1^e &= \frac{1}{4\,a\,b} \,(a - x)(b - y)\qquad &\psi_2^e &= \frac{1}{4\,a\,b} \,(a + x)(b - y) \\
\psi_3^e &= \frac{1}{4\,a\,b} \,(a + x)(b + y)\qquad &\psi_4^e &= \frac{1}{4\,a\,b} \,(a
- x)(b + y)\,.
\end{alignat}
Die Verzerrungen und Spannungen in einem solchen Element sind linear ver\"{a}nderlich
\bfoo
\varepsilon_{xx} = a_1 + a_3\,y\,, \qquad \varepsilon_{yy} = b_2 + b_3\,x\,, \qquad
\gamma_{xy} = (a_2 + b_1) + a_3\,x + b_3\,y\,,
\efoo
allerdings in der \hlq falschen Richtung\grq, denn die Normalspannungen verlaufen in der
Beanspruchungsrichtung im wesentlichen konstant und nur quer dazu linear ver\"{a}nderlich.
Nur die Schubspannungen sind nach beiden Richtung linear ver\"{a}nderlich, s. Bild
\ref{Cook52}.
%----------------------------------------------------------------------------------------------------------
\begin{figure}[tbp]
\if \bild 2 \sidecaption \fi
\includegraphics[width=.6\textwidth]{\Fpath/COOK52}
\caption{Kragtr\"{a}ger mit Endkr\"{a}ften, Berechnung mit bilinearen Elementen. Die
Normalspannung $\sigma_{xx}$ ist in den Elementen f\"{u}r $\nu = 0$ konstant. In jedem
Schnitt ist das Biegemoment also gleich gro{\ss}. Die \"{u}ber die Tr\"{a}gerh\"{o}he gemittelten
Schubspannungen springen hin und her} \label{Cook52}
\end{figure}%%
%----------------------------------------------------------------------------------------------------------

Auch dieses Element reagiert auf Biegebeanspruchungen noch zu steif, s. Bild
\ref{Cook49}. Bei Deformationen stellen sich zwar die Kanten schief, aber sie bleiben
gerade, und die Kr\"{u}mmung an der Ober- und Unterseite kann das Element nicht darstellen.
Verdreht ein Momentenpaar $M$ einen Ausschnitt eines Balkens um einen Winkel $\varphi$,
so braucht man f\"{u}r dieselbe Verdrehung eines bilinearen Elements mit dem
Seitenverh\"{a}ltnis $a/b$ das Momentenpaar
$$
M_{FE} = \frac{1}{1 + \nu} \left [ \frac{1}{1-\nu} + \frac{1}{2}\,\left
(\frac{a}{b}\right )^2 \right ]\,M\,.
$$
Es ist, wie man sich leicht \"{u}berzeugt, immer $M_{FE} > M$, und der Unterschied nimmt mit
wachsendem Seitenverh\"{a}ltnis $a/b$ quadratisch zu, so dass sich das Element zunehmend
gegen die Verdrehung der Endquerschnitte sperrt ({\em locking\/}).
%----------------------------------------------------------------------------------------------------------
\begin{figure}[tbp]
\if \bild 2 \sidecaption \fi
\includegraphics[width=.6\textwidth]{\Fpath/COOK49}
\caption{Die Seiten eines bilinearen Elements bleiben immer gerade} \label{Cook49}
\end{figure}%%
%----------------------------------------------------------------------------------------------------------

{\textcolor{blue}{\subsection{LST-Element}}}\index{LST-Element}
Flexibler ist da das {\em linear strain triangle (LST-Element)}
\index{linear strain triangle} mit sechs Knoten, 3 Eck- plus 3 Seitenmittenknoten, das
auf quadratischen Verschiebungsans\"{a}tzen beruht
\bfo\label{Q1}
u(x,y) &=& a_0 + a_1 x + a_2 y + a_3 x \,y + a_4\,x^2 + a_5\,y^2  \\
\label{Q2} v(x,y) &=& b_0 + b_1 x + b_2 y + b_3 x\, y + b_4\,x^2 + b_5\,y^2\,  \,,
\efo
denn hier sind $\varepsilon_{xx}$ und $\varepsilon_{yy}$ nach {\em beiden\/}
Koordinatenrichtungen linear ver\"{a}nderlich
\bfoo
\varepsilon_{xx} &=& a_2 + 2\, a_4\,x + a_5\,y\,, \qquad \varepsilon_{yy} = b_2 + b_3\,x + 2\,b_5\,y\,, \\
 \gamma_{xy} &=& (a_2 + b_1) + (a_3 + 2\,b_4)\,x + (2\,a_5 + b_3)\,y\,.
\efoo
%----------------------------------------------------------------------------------------------------------
\begin{figure}[tbp]
\if \bild 2 \sidecaption \fi
\includegraphics[width=.6\textwidth]{\Fpath/COOK53OBEN}
\caption{Quadratische Ans\"{a}tze $u = (1- \eta^2)\,q_2$ und $v = (1 - \xi^2)\,q_3$}
\label{Cook53Oben}
\end{figure}%%
%----------------------------------------------------------------------------------------------------------
%----------------------------------------------------------------------------------------------------------
\begin{figure}[tbp]
\if \bild 2 \sidecaption \fi
\includegraphics[width=.7\textwidth]{\Fpath/WILSON}
\caption{Kragtr\"{a}ger mit Streckenlast als {\em benchmark\/} f\"{u}r das bilineare Element und
das Element von Wilson} \label{Wilson}
\end{figure}%%
%----------------------------------------------------------------------------------------------------------

{\textcolor{blue}{\subsection{Bilinear + 2}}\index{Wilson Element}\index{Q4 + 2}
Von Wilson \cite{Wilson} stammt die Idee, das bilineare Element in jeder
Richtung durch zwei quadratische Ansatzfunktionen anzureichern, (Q4 + 2),
\bfoo
u &=& \ldots + (1 - \xi^2)\, q_1 + (1- \eta^2)\,q_2 \qquad \xi = x/a\,\qquad \eta = y/b\\
v &=& \ldots + (1 - \xi^2)\, q_3 + (1- \eta^2)\,q_4\,,
\efoo
die also in der Lage sind, konstante Kr\"{u}mmungen darzustellen, s. Bild \ref{Cook53Oben}.
Werden diese Freiheitsgrade $q_i$ aktiviert, dann w\"{o}lbt sich das Element nach oben oder
zur Seite. Weil keine Koordination zwischen den Nachbarelementen stattfindet -- die $q_i$
sind innere Freiheitsgrade, die durch {\em statische Kondensation\/} der Elementmatrix
sp\"{a}ter wieder beseitigt werden -- durchdringen sich die Kanten benachbarter Elemente
bzw. entstehen klaffende Fugen. Das Element ist also nicht konform. Konform wird es erst,
wenn die Elemente sehr klein werden, und die Verzerrungen dann nahezu konstant sind.

Wenn die Verzerrungen aber konstant sind, dann sind die Verschiebungen h\"{o}chstens linear,
was bedeutet, dass sich die Kanten eines Elements zwar schief stellen, aber gerade
bleiben und die inkompatiblen Anteile werden somit \"{u}berfl\"{u}ssig, $q_i = 0$. Das ist wohl
auch der Grund, warum dieses Element trotz seines \hlq Defekts\grq \, so erfolgreich ist
und gerne in kommerziellen Programmen als 4-Knoten-Element eingesetzt wird. Bei
richtiger Implementation ergeben sich nach den Erkenntnissen von Lesaint
\cite{Lesaint} stabile Elemente.
\begin{table}[tbp]\label{TabWilson}
\caption{ Durchbiegung $v$ am Kragarmende, s. Bild \ref{Wilson} mit bilinearen Elementen
(Q4) und Elementen (Q4+2) nach Wilson. Der exakte Wert betr\"{a}gt $v = 1.024$ cm}
\begin{tabular}{rrrr}
\noalign{\hrule\smallskip}
      Netz &         Q4 &       Q4+2 \\
\noalign{\hrule\smallskip}
     1 x 8 &   0,715 cm &      1,035 cm\\
    2 x 16 &   0,939 cm &      1,036 cm \\
    4 x 32 &   1,010 cm &      1,038 cm \\
    8 x 80 &   1,021 cm &      1,039  cm\\
\end{tabular}
\end{table}

Rechnet man den Kragarm in Bild \ref{Wilson} mit bilinearen Elementen, so ben\"{o}tigt man
sehr viele Elemente, um in die N\"{a}he der exakten Kragarmdurchbiegung, $w = 1.024$ cm, zu
kommen, s. Tabelle 4.3. Das Wilson-Element schafft das praktisch mit acht Elementen.
Daran, dass es leicht \"{u}ber das Ziel hinausschie{\ss}t, erkennt man, dass es ein
nichtkonformes Element ist.

Weil die internen Freiheitsgrade $q_i$ durch statische Kondensation eliminiert werden,
werden die eigentlich zu den $q_i$ geh\"{o}rigen \"{a}quivalenten Knotenkr\"{a}fte Null gesetzt bzw.
erst gar nicht berechnet. Die Grundgleichung am Element lautet also vor der Kondensation
\bfoo
\left[ \barr {c c} \vek K_{uu} & \vek K_{uq} \\ \vek K_{qu} & \vek K_{qq} \earr \right]
\left[\barr {c} \vek u_{(8)} \\ \vek q_{(4)} \earr \right] = \left[ \barr {c} \vek
f_{(8)} \\ \vek 0_{(4)} \earr \right]
\efoo
und zerf\"{a}llt danach in zwei Teile
\bfoo
\vek K_{(8 \times 8)} \vek u_{(8)} = \vek f_{(8)} \qquad \vek q_{(4)} = \vek Q_{(4
\times  8)}\,\vek u_{(8)}\,,
\efoo
wobei
\bfoo
\vek K = \vek K_{uu} - \vek K_{uq}^T\,\vek K_{qq}^{-1}\,\vek K_{qu}\,, \qquad \vek Q = -
\vek K_{qq}^{-1}\,\vek K_{uq}\,,
\efoo
mit den $q_i$ als den abh\"{a}ngigen Freiheitsgraden.

In der \"{u}blichen FE-Notation, s. Kapitel \ref{Steifigkeitsmatrizen}, lautet der
Verzerrungsvektor $\vek \varepsilon$ des Elements bei einer Deformation $\vek u, \vek q$
\bfoo
\vek \varepsilon = \vek B_u\,\vek u + \vek B_q\,\vek q\,,
\efoo
und nach der Kondensation wird daraus
\bfoo
\vek \varepsilon = \vek B\,\vek u \qquad \vek B := \vek B_u + \vek B_q\,\vek Q\,.
\efoo
Durch das Hinzuf\"{u}gen der inneren Moden wird also die Verzerrungsmatrix $\vek B_u$ um
Zusatzterme angereichert.
%----------------------------------------------------------------------------------------------------------
\begin{figure}[tbp]
\if \bild 2 \sidecaption \fi
\includegraphics[width=0.7\textwidth]{\Fpath/Q8}
\caption{Scheibenelemente {\bf a)} Q8-Element, {\em Seren\-di\-pi\-ty-Element\/} {\bf b)}
9-Knoten-Ele\-ment mit biquadratischen Ans\"{a}tzen, {\em Lagrange-Element\/} } \label{Q8}
\end{figure}%%
%----------------------------------------------------------------------------------------------------------

{\textcolor{blue}{\subsection{Lagrange und Serendipity Elemente}}}\index{Lagrange-Elemente}\index{Serendipity-Element}\index{Q8}
Baut man die Ansatzfunktionen aus {\em Lagrange-Polynomen\/} auf, so entstehen sogenannte
{\em Lagrange-Elemente\/}, das sind parallelogrammartige Elemente mit Rand- und
Innenknoten. Elemente der sogenannten {\em Serendipity-Klasse\/} kommen ohne Innenknoten
aus, sind daf\"{u}r aber unvollst\"{a}ndig. Das Quadrat mit 8 Knoten (Q8), s. Bild \ref{Q8},
dessen Ans\"{a}tze man erh\"{a}lt, wenn die obigen quadratischen Ans\"{a}tze des LST-Elements
(\ref{Q1}) und (\ref{Q2}) um einige ausgew\"{a}hlte kubische Terme erweitert werden,
\bfoo
u(x,y) &=& \ldots + a_6\,x^2 \,y + a_7\,x\,y^2 \,,\\
v(x,y) &=& \ldots + b_6\,x^2 \,y + b_7\,x\,y^2 \,,
\efoo
ist ein Element der {\em Serendipity-Klasse\/} mit den Verzerrungen
\bfoo
\varepsilon_{xx} &=& a_2\,x + 2\,a_4\,x^2 + 2\,a_7\,x\,y + a_8\,y^2\\
\varepsilon_{yy} &=& b_3 + b_5\,x + 2\,b_6\,y + b_7\,x^2 + 2\,b_8\,x\,y \\
\gamma_{xy} &=& (a_3 + b_2) + (a_5 + 2\,b_4)\,x + (2\,a_5 + b_5)\,y + a_6\,x^2\\
&& + 2(a_7 + b_7) \,x\,y + b_8\,\,y^2\,.
\efoo

{\textcolor{blue}{\subsection{H\"{o}here Polynomgrade}}}\index{h\"{o}here Polynomgrade}
Es gibt jedoch auch Elemente mit noch h\"{o}heren Ans\"{a}tzen. Deren Nachteil liegt zum einen
darin, dass Ableitungen wie $u,_x$ oder $v,_y$ als Knotenwerte vorkommen, was
Schwierigkeiten macht, wenn die Spannungen an einem Knoten springen sollen, weil z.B.
dort eine Linienlast wirkt, und zum andern ist es so, dass bei der Kopplung mit anderen
Elementen der Ansatz auf der gemeinsamen Kante identisch sein muss. So ist es z.B. nicht
empfehlenswert, einen klassischen Biegebalken mit kubischem Ansatz mit einem
isoparametrischen Element zu verkn\"{u}pfen.

{\textcolor{blue}{\subsection{Drilling degrees of freedom}}}\index{drilling degrees of freedom}
Ein Scheibenknoten hat keine Drehfreiheitsgrade, weil es in der Scheibenstatik keine
Drehmomente gibt. Das ist hinderlich bei der Kopplung von Scheiben mit Biegebalken.

Die Grundidee, um dies zu korrigieren, ist ein isoparametrischer Ansatz, bei dem die
Mittelknoten jedoch nicht explizit vorkommen, sondern ihre Verschiebungen aus den
Verschiebungen und Verdrehungen der Eckknoten berechnet werden. So werden auch die
Elemente konstruiert: Man nimmt z.B. das LST-Element (Dreieck mit quadratischen Ans\"{a}tzen
und zus\"{a}tzlichen Knoten in den Seitenmitten) und tut so, als ob das Element
Drehfreiheitsgrade in den Ecken h\"{a}tte. Mit diesen Drehfreiheitsgraden gelingt es dann die
Verschiebungen in den Seitenmitten darzustellen, und so kommt man auf die Idee diese
Freiheitsgrade in den Seitenmitten herzugeben, um Drehfreiheitsgrade in den Ecken
einzuf\"{u}hren. Um brauchbare Elemente zu erhalten bedarf es allerdings noch einiger
mathematischer Kunstgriffe, \cite{Cook0}, \cite{Bergan}, \cite{Allman}.

Der unsch\"{a}tzbare Vorteil dieses Elements liegt in der Ber\"{u}cksichtigung der Verdrehungen,
was es vor allem f\"{u}r r\"{a}umliche Faltwerke oder Schalentragwerke interessant macht. Da
aber nur ein Verdrehungsfreiheitsgrad hinzukommt, die Seitenmitten aber doppelt so viele
Freiheitsgrade haben, sind die M\"{o}glichkeiten dieses Elements gegen\"{u}ber dem vollen
isoparametrischem bzw. nichtkonformen Element etwas eingeschr\"{a}nkt.

%%%%%%%%%%%%%%%%%%%%%%%%%%%%%%%%%%%%%%%%%%%%%%%%%%%%%%%%%%%%%%%%%%%%%%%%%%%%%%%%%%%%%

%%%%%%%%%%%%%%%%%%%%%%%%%%%%%%%%%%%%%%%%%%%%%%%%%%%%%%%%%%%%%%%%%%%%%%%%%%%%%%%%%%%%%%%%%%%
{\textcolor{blue}{\subsection{Isoparametrische Elemente}}}\index{isoparametrische Elemente}
%%%%%%%%%%%%%%%%%%%%%%%%%%%%%%%%%%%%%%%%%%%%%%%%%%%%%%%%%%%%%%%%%%%%%%%%%%%%%%%%%%%%%%%%%%%
%----------------------------------------------------------------------------------------------------------
\begin{figure}[tbp]
\if \bild 2 \sidecaption \fi
\includegraphics[width=1.0\textwidth]{\Fpath/ELEMENTE}
\caption{Lineares und bilineares Element und zwei Elemente mit quadratischen Ans\"{a}tzen.
Alle Elemente sind isoparametrische Elemente. Dieselben Ansatzfunktionen, die die Lage
der Elemente beschreiben, beschreiben auch die Elementverformungen} \label{ElementeIso}
\end{figure}%%
%----------------------------------------------------------------------------------------------------------
Um auf dem Bildschirm eine Linie von einem Punkt $A$ zu einem Punkt $B$ zu ziehen,
beschreibt man die Bewegungen des Zeichenstifts durch zwei Funktionen $x(\xi)$ und
$y(\xi)$. Der Parameter $\xi$ durchl\"{a}uft dabei die Strecke $[-1,+1]$, das {\em master
element\/}\index{master element}. Ist die Kurve sehr komplex, dann wird ein Polynom
zweiten Grades nicht ausreichen, um die Kurve genau wiederzugeben, aber das
Zeichenprogramm wird die Ersatzkurve so legen, dass sie das Original wenigstens in 3
Kontrollpunkten trifft -- soviel M\"{o}glichkeiten bietet ein Polynom zweiten Grades.

Ebenso kann man sich ein beliebiges viereckiges Element auf dem Bildschirm aus einem
zweidimensionalen {\em master element\/} $-1 < \xi,\eta < +1$ mittels Funktionen
$x(\xi,\eta)$ und $y(\xi,\eta)$ erzeugt denken, und wieder wird man, wenn die Form der
Elemente sehr komplex ist, Schwierigkeiten haben, mit niedrigen Polynomen den Umriss
darzustellen, s. Bild \ref{ElementeIso}. Die Zeichenaufgabe ist also ein \"{a}hnliches
Interpolationsproblem, wie die Darstellung der Verformungen eines Elements.

Isoparametrische Elemente hei{\ss}t, dass man die Elemente aus einem {\em master element\/}
erzeugt, und die bildgebenden Funktionen dieselben Polynome sind, die Verformungen
darstellen. Der Begriff selbst ist ein {\em terminus technicus\/}, der an sich kein
Qualit\"{a}tsmerkmal darstellt. Isoparametrische Elemente sind -- von der Schalenstatik
vielleicht abgesehen -- die Regel in der FEM.

Im engeren Sinne hei{\ss}t ein Element isoparametrisch, wenn die bildgebenden Polynome genau
denselben Grad haben, wie die Polynome, die die Verformungen darstellen. Bei einem {\em
superparametrischen Element\/}\index{superparametrisches Element} gibt es
Geometrie-Knoten, die keine Freiheitsgrade haben und bei einem {\em subparametrischen
Element\/}\index{subparametrische Elemente} haben die Verschiebungsans\"{a}tze mehr
Freiheitsgrade als zur Darstellung des Elements n\"{o}tig sind. Das letztere kommt oft vor.
Hat ein Dreieck z.B. gerade Kanten, obwohl der Ansatz auch gekr\"{u}mmte Kanten erlauben
w\"{u}rde, dann ist das Dreieck ein subparametrisches Element. Diese Zur\"{u}ckhaltung bei der
Modellierung ist \"{u}brigens auch die einzige M\"{o}glichkeit, wie man mit isoparametrischen
Elementen linear ver\"{a}nderliche Verzerrungen darstellen kann.

{\em 1. Beispiel\/} In einem Dreieck mit sechs Knoten sind die Ansatzfunktionen auf dem
{\em master element\/} Linearkombinationen der Funktionen
\bfo\label{Basis41}
1, \xi, \eta, \xi^2, \xi\,\eta, \eta^2\,.
\efo
Hat das Dreieck gerade Kanten, dann ist $y = c_0 + c_1\,\xi + c_2\,\eta$ eine lineare
Funktion in $\xi$ und $\eta$ und eine Verschiebung wie $u = y^2$ l\"{a}sst sich dann durch
die Ansatzfunktionen (\ref{Basis41}) darstellen, w\"{a}hrend bei gekr\"{u}mmten Kanten $y$ eine
quadratische Funktion in $\xi$ und $\eta$ w\"{a}re und $u = y^2$ dann nicht mehr exakt
darstellbar w\"{a}re.

{\em 2. Beispiel\/} In einem schiefen Rechteck braucht man zur Darstellung des Verlaufs
von $y$ im Element auf dem {\em master element\/} die Funktionen
\bfo\label{Basis241}
1, \xi, \eta, \xi\,\eta\,.
\efo
Um $u = y^2$ darzustellen, braucht man also die neun Funktionen
\bfo\label{Basis341}
1, \xi, \eta, \xi^2, \xi\,\eta, \eta^2, \xi^2\,\eta, \xi\,\eta^2, \xi^2\,\eta^2\,,
\efo
und das einfachste Element, das all diese Terme enth\"{a}lt, ist das {\em Lagrange
Element\/} mit neun Knoten. Im Umkehrschluss folgt daraus, dass das {\em Serendipity
Element\/} mit nur acht Knoten (ohne den Innenknoten des {\em Lagrange Element\/})
lineare Verzerrungen nicht exakt darstellen kann, \cite{MacNeal}.

Superparametrische Elemente dagegen sind eigentlich unbrauchbar, denn solche Elemente
k\"{o}nnen Starrk\"{o}rperbewegungen nicht darstellen. Einen Rand mit gekr\"{u}mmten Elementen zu
modellieren und dann lineare Ans\"{a}tze f\"{u}r die Elemente zu w\"{a}hlen, gibt keinen Sinn. Bei
isoparametrischen Elementen hat man dagegen die Eigenschaft, dass sie
Starrk\"{o}rperbewegungen darstellen k\"{o}nnen, in der Regel garantiert.

F\"{u}r ein Netz aus schiefen, viereckigen Elementen ben\"{o}tigt man ein rechteckiges {\em
master element\/}\index{master element}, s. Bild \ref{Master}, aus dem man dann mit {\em
bilinearen\/} Koordinatenfunktionen
\bfoo
x(\xi,\eta) = a_0 + a_1 \xi + a_2 \eta + a_3 \xi \,\eta\,,  \qquad y(\xi,\eta) = b_0 +
b_1 \xi + b_2 \eta + b_3 \xi \,\eta\,,
\efoo
die Elemente erzeugt. Krummlinig berandete Elemente entstehen, wenn man quadratische
oder kubische Koordinatenfunktionen benutzt.

%----------------------------------------------------------------------------------------------------------
\begin{figure}[tbp]
\if \bild 2 \sidecaption \fi
\includegraphics[width=1.0\textwidth]{\Fpath/MASTER}
\caption{{\em master element\/}, Gausspunkte und Element} \label{Master}
\end{figure}%%
%----------------------------------------------------------------------------------------------------------

Grunds\"{a}tzlich wird die Integration auf die Integration \"{u}ber das {\em master element\/}
zur\"{u}ckgef\"{u}hrt. Hierbei spielt die Determinante der sogenannten {\em
Jacobi-Matrix\/}\index{Jacobi-Matrix}, in der die Ableitungen der bildgebenden
Funktionen stehen,\index{Jacobi-Matrix, Determinante}\index{Determinante der
Jacobi-Matrix}
\bfoo
\vek J = \left[ \barr {c @{\hspace{2mm}} c} x,_{\,\xi} & y,_{\,\xi} \\ x,_{\,\eta} &
y,_{\,\eta} \earr \right] \,, \qquad \mbox{det} \,\vek J = x,_\xi \,y,_\eta -
y,_\xi\,x,_\eta\,,
\efoo
eine wichtige Rolle. Sie stellt das Verh\"{a}ltnis der Fl\"{a}chenelemente dar
\bfoo
A = \int_{\Omega} dx\,dy = \int_{\Omega_M}  \mbox{det} \,\vek J(\xi,\eta)\,d\xi\,d\eta\,,
\qquad \mbox{det} \, \vek J(\xi,\eta) = \frac{d\Omega}{d\Omega_M}\,.
\efoo
Die Knoten des {\em master elements\/} $\Omega_M$ erfahren grunds\"{a}tzlich dieselben
Verformungen wie sein Bild, das Element $\Omega$. Gleiche Verformungen an ungleich
langen Elementen bedeuten aber ungleiche Verzerrungen. Wird das Element $\Omega$ -- es
sei ein Rechteck mit der L\"{a}nge $a$ -- um $u$ cm gestreckt und das Masterelement
$\Omega_M$ ebenfalls, dann sind die Verzerrungen $\varepsilon_{\xi \,\xi} = u/2$ und
$\varepsilon_{xx} = u/a$ in der Regel ungleich. Nach der Kettenregel \index{Kettenregel}
ist
$$
\varepsilon_{\xi\,\xi}= u,_{\,\xi} = u,_x\,x,_{\,\xi} + u,_y \,y,_{\,\xi} \qquad
\varepsilon_{\eta\,\eta}= v,_{\,\eta} = v,_x\,x,_{\,\eta} + v,_y \,y,_{\,\eta}\,,
$$
oder allgemeiner
\bfoo
\left[ \barr {c} u,_{\,\xi} \\ u,_{\,\eta} \earr \right] = \underbrace{\left[ \barr {c
c} x,_{\,\xi} & y,_{\,\xi} \\ x,_{\,\eta} & y,_{\,\eta} \earr \right] }_{\vek J} \left[
\barr {c} u,_x \\ u,_y \earr \right] \qquad \left[ \barr {c} u,_x \\ u,_y \earr \right] =
\underbrace{\left[ \barr {c c} \xi,_{\,x} & \eta,_{\,x} \\ \xi,_{\,y} & \eta,_{\,y}
\earr \right] }_{\vek J^{-1}} \left[ \barr {c} u,_{\,\xi} \\ u,_{\,\eta} \earr \right]
\,,
\efoo
wobei sich die Elemente der inversen Matrix $\vek J^{-1}$ wie folgt berechnen,
\cite{Schwarz} S. 120,
\bfoo
\xi,_x = \frac{y,_\eta }{d} \qquad \xi,_y = - \frac{x,_\eta}{d} \qquad \eta,_x = -
\frac{y,_\xi }{d} \qquad \eta,_y = \frac{\, x,_\xi}{d} \qquad d = \mbox{det}\,\vek J\,.
\efoo
Somit gilt also z.B. f\"{u}r die Verzerrungen
\bfoo
\varepsilon_{xx} &=& u,_x = u,_{\,\xi}\,\xi,_x + u,_{\,\eta}\,\eta,_x \qquad
\varepsilon_{yy} = v,_y = v,_{\,\xi}\,\xi,_y + v,_{\,\eta}\,\eta,_y \\
\gamma_{xy} &=& u,_y + v,_x = u,_{\,\xi}\,\xi,_y + u,_{\,\eta}\,\eta,_y +
v,_{\,\xi}\,\xi,_x + v,_{\,\eta}\,\eta,_x \,.
\efoo
Um also die Originalverzerrung $\varepsilon_{xx}$ auf $\Omega_M$ zu reproduzieren, muss
man die Master-Verzerrungen $\varepsilon_{\xi\,\xi}$ und $\varepsilon_{\eta\,\eta}$ mit
$\xi,_x, \xi,_y$ etc. wichten. So kann man durch Integration \"{u}ber das {\em master
element\/} $\Omega_M$ die Elementsteifigkeitsmatrix berechnen, s. Kapitel
\ref{Steifigkeitsmatrizen},
\bfoo
\vek K = t\,\int_{\Omega_M} \vek B^T\,\vek E\,\vek B \,\mbox{det}\,\vek
J\,\,d\xi\,d\eta\,.
\efoo
%----------------------------------------------------------------------
\begin{figure}[tbp]
\if \bild 2 \sidecaption \fi
\includegraphics[width=.7\textwidth]{\Fpath/JACOBI}
\caption{Bilineare Elemente a) korrekte Gestalt, b) der Eckenwinkel ist zu gro{\ss},
nichtkonvexes Element} \label{Jacobi}
\end{figure}%%
%----------------------------------------------------------------------
Damit die Beziehung zwischen dem {\em master element\/} und seinem Bild, dem
eigentlichen Element, eineindeutig ist, darf die Determinante -- $dr_1$ und $dr_2$ seien
i.a. schiefwinklige Differentiale entlang den Kanten des Fl\"{a}chenelements, s. Bild
\ref{Jacobi}, --
\bfoo
\mbox{det}\,\vek J = \frac{dr_1\,dr_2 \,\sin \alpha}{d\xi\,d\eta}
\efoo
in keinem Punkt des Elements Null werden bzw. ihr Vorzeichen wechseln (was dasselbe
ist). Daher die Regel, dass die Eckenwinkel $\alpha$ in bilinearen Elementen, also gerade
berandeten Rechtecken, nicht gr\"{o}{\ss}er als $180^\circ$ werden d\"{u}rfen. Solche Ecken sind
{\em einspringende Ecken\/}, s. Bild \ref{Jacobi}. Weil die Elemente auch nicht zu spitz
auslaufen sollten, sollten die Eckenwinkel $\alpha$ bilinearer Scheibenelemente und auch
des Elements von Wilson die Grenzen $30^\circ < \alpha < 120^\circ $ einhalten.

%%%%%%%%%%%%%%%%%%%%%%%%%%%%%%%%%%%%%%%%%%%%%%%%%%%%%%%%%%%%%%%%%%%%%%%%%%%%%%%%%%%%%

%%%%%%%%%%%%%%%%%%%%%%%%%%%%%%%%%%%%%%%%%%%%%%%%%%%%%%%%%%%%%%%%%%%%%%%%%%%%%%%%
{\textcolor{blue}{\section{Numerische Integration}}}\label{NumerischeIntegration}\index{Numerische Integration}
%%%%%%%%%%%%%%%%%%%%%%%%%%%%%%%%%%%%%%%%%%%%%%%%%%%%%%%%%%%%%%%%%%%%%%%%%%%%%%%%
Das Thema ist deswegen nicht ganz unwichtig, weil man durch {\em reduzierte
Integration\/}\index{reduzierte Integration} die Elemente weicher machen kann. Die
Standardelemente sind ja in vielen F\"{a}llen zu steif, weil sie sich den Bewegungen mehr
als erforderlich entgegensetzen. Insbesondere hat das bilineare Rechteckelement M\"{u}he,
Biegezust\"{a}nde einer Scheibe darzustellen. Eine M\"{o}glichkeit dies zu korrigieren, ist der
Einsatz einer sogenannten {\em unvollst\"{a}ndigen numerischen Integration\/}.

Nur bei einfachen Ans\"{a}tzen und geradlinig berandeten Elementen ist es m\"{o}glich, Integrale
wie
\bfoo
k_{\,ij} = \int_{\Omega} \vek \sigma_i \dotprod \vek \varepsilon_j \,d\Omega\,, \qquad
f_i = \int_{\Omega}\vek p \dotprod \vek \Np_i\,d\Omega\,,
\efoo
analytisch zu berechnen.
%----------------------------------------------------------------------
\begin{figure}[tbp]
\if \bild 2 \sidecaption \fi
\includegraphics[width=1.0\textwidth]{\Fpath/QUADRATUR}
\caption{Zwei-Punkte- und Drei-Punkte-Gaussquadratur. Das Integral der Funktion $f(x)$
im zweiten Bild ist Null, weil $f(x)$ in den St\"{u}tzstellen der Quadratur durch Null geht.
Weil eine 3-Punkte-Gauss-Quadratur Polynome vom Grad $2 \times 3 - 1 = 5$ exakt
integriert, muss die Funktion -- wenn sie denn ein Polynom ist -- einen Grad gr\"{o}{\ss}er 5
haben} \label{Quadratur}
\end{figure}%%
%----------------------------------------------------------------------
In allen anderen F\"{a}llen benutzt man meistens eine Gauss-Quadratur $n$-ter Ordnung
\bfoo
\int_{-1}^{+1} f(\xi)\,d\xi = \sum_{i = 1}^n f(\xi_i)\, w_i + O(f^{(2n)}) \quad
\leftarrow \quad\mbox{Fehlerterm}
\efoo
die Polynome bis zur Ordnung $2\,n - 1$
exakt integriert, weil in den Fehler nur Ableitungen $f^{(2n)}$ und h\"{o}her eingehen,
s. Bild \ref{Quadratur}. Man beachte
jedoch, dass wegen $f(\xi) = \varepsilon(\xi)\,\sigma(\xi)$ der Polynomgrad des
Integranden in der Wechselwirkungsenergie immer doppelt so gro{\ss} ist, wie der von
$\varepsilon$ und $\sigma$ alleine.

Um \"{u}ber ein ebenes Element $\Omega$ zu integrieren, wendet man die obige Formel zweimal
an
\bfoo
\int_{-1}^{+1}\int_{-1}^{+1} f(\xi,\eta)\,d\xi\,d\eta \simeq  \sum_{i,j = 1}^n
f(\xi_i,\eta_j)\, w_i\,w_j\,.
\efoo
Das Ziel muss es zum einen sein, die Integrale genau zu berechnen, zum anderen soll $n$
nicht zu gro{\ss} sein, um die Rechenzeit klein zu halten. Sind die Integranden Polynome,
wie bei Elementen mit geraden Kanten, kann man die n\"{o}tige Zahl $n$ der Punkte an Hand
des Grads $p$ der Ansatzfunktionen und der Ordnung $m$ der Energie exakt ausrechnen,
$2\,(p - m) = 2\,n - 1$. Bei gekr\"{u}mmten Elementen hingegen ist es sinnvoll, $n$ so gro{\ss}
zu w\"{a}hlen, dass das Element mit geraden Kanten von der Formel exakt integriert w\"{u}rde,
\cite{Bathe}.

Nimmt man zu wenig Gausspunkte, dann kann es passieren, dass {\em spurious modes\/}
auftreten. Das sind Verformungen, die keine Energie haben. Bei der numerischen Quadratur
\bfoo
k_{\,ij} &=& \int_{\Omega_M} \vek \sigma^{(i)}\dotprod \vek
\varepsilon^{(j)}\,\mbox{det}\, \vek J\, d\xi\,d\eta \\
&\simeq& \sum_{l,m = 1}^n \left\{ \,\vek \sigma^{(i)}(\xi_l,\eta_m) \dotprod \vek
\varepsilon^{(j)}(\xi_l,\eta_m) \, \mbox{det}\, \vek J(\xi_l,\eta_m) \,\right
\}\,w_l\,w_m
\efoo
wird ja nicht echt integriert, sondern der Integrand nur in gewissen Punkten des {\em
master-elements\/} gemessen, gewichtet, und aufsummiert.

Nimmt man z.B. f\"{u}r ein bilineares Element nur einen Gausspunkt, den Mittelpunkt  $\xi =
\eta = 0$  des {\em master elements\/}, dann stellen die Verschiebungsfelder $\vek \Np_a
= \{\xi\,\eta,0\}^T, \vek \Np_b = \{0,\xi\,\eta\}^T$ und $\vek \Np_c = \{\eta(1 -
\xi),\xi\,(1-\eta)\}^T$ {\em spurious modes\/} dar, weil im Gausspunkt alle Verzerrungen
$\varepsilon_{\xi\,\xi} = \varepsilon_{\eta\,\eta} = \gamma_{\xi\,\eta} = 0$ Null sind.

Man nennt solche Verschiebungsfelder auch {\em zero energy modes\/}\index{zero energy
modes}
 oder in einigen
F\"{a}llen, auf Grund ihrer speziellen Gestalt, auch {\em hourglass modes\/}\index{hourglass
modes}.

Es ist anschaulich klar, dass man diese hier betrachteten {\em zero energy modes\/}
$\vek \Np_a, \vek \Np_b, \vek \Np_c$ \"{u}ber das Element hinaus fortsetzen kann, indem man
auf allen Elementen dieselben Verformungen einstellt, und daher ist auch die globale
Steifigkeitsmatrix -- auch wenn man die Scheibe ausreichend lagert -- singul\"{a}r.
%----------------------------------------------------------------------
\begin{figure}[tbp]
\if \bild 2 \sidecaption \fi
\includegraphics[width=.8\textwidth]{\Fpath/LAMBDA}
\caption{Bilineares Element mit der Kantenl\"{a}nge Eins. Die ersten drei Eigenformen sind
Starrk\"{o}rperbewegungen mit $\lambda_1 = \lambda_2 = \lambda_3 = 0$} \label{Lambda}
\end{figure}%%
%----------------------------------------------------------------------

Wird die Ordnung der Quadratur absichtlich erniedrigt, um die Elemente weicher zu
machen, dann spricht man von {\em reduzierter Integration\/}. Behindern die
Nachbarelemente die Ausbildung von {\em zero energy modes\/}, sind die Moden nicht
kommunikabel, dann ist es m\"{o}glich die Zahl der Gausspunkte zu reduzieren, ohne
bef\"{u}rchten zu m\"{u}ssen, dass die Gesamtsteifigkeitsmatrix singul\"{a}r wird.

Die Elemente werden weicher, weil, wie wir im folgenden zeigen wollen, die Eigenwerte
der Elementmatrizen kleiner werden.

Eine $n \times n$ Matrix $\vek K$ hat $n$ Eigenvektoren $\vek u_i$ und $n$ Eigenwerte
$\lambda_i \geq 0$, (gr\"{o}{\ss}er gleich Null, weil $\vek K$ positiv semidefinit ist), s. Bild
\ref{Lambda}
\bfoo
\vek K\,\vek u_i = \lambda_i\,\vek u_i\,.
\efoo
Ohne Beschr\"{a}nkung der Allgemeinheit d\"{u}rfen wir annehmen, dass die Eigenvektoren $\vek
u_i$ alle die L\"{a}nge Eins haben und aufeinander senkrecht stehen
\bfo\label{Eigenwert41}
\vek u_i \dotprod \vek u_j = \left\{ \barr {l r} 1 & \qquad i = j
\\ 0 & \qquad i \neq j \earr \right. \qquad \Rightarrow \qquad
\vek u_i\,\vek K\,\vek u_j = \left\{ \barr {l r} \lambda_i & \qquad i = j \\ 0 & \qquad
i \neq j \earr \right.
\efo
und die kleinsten Eigenwerte ganz links stehen und die gr\"{o}{\ss}ten ganz rechts, $\lambda_1
\leq \lambda_2 \leq \ldots \leq \lambda_n$.

Die Eigenwerte $\lambda_i$ stellen gem\"{a}{\ss} (\ref{Eigenwert41}) die innere Energie
($\times\,\, 2$) dar, die zu den Eigenvektoren geh\"{o}ren. Entwickeln wir einen beliebigen
Verschiebungsvektor $\vek u$ des Elements nach den Eigenvektoren, $\vek u = \sum_i
q_i\,\vek u_i$, dann folgt, wie leicht zu sehen ist,
\bfoo
a(\vek u,\vek u) = \vek u^T\,\vek K\,\vek u = \sum_i \lambda_i\, q_i^2 =
\lambda_1\,q_1^2 + \lambda_2\,q_2^2 + \lambda_3\,q_3^2 + \lambda_4\,q_4^2 + \ldots \,.
\efoo
Man kann die innere Energie ($\times 2) $ also nicht kleiner machen als
$\lambda_4\,q_4^2$, wenn $\lambda_4$ der kleinste, positive Eigenwert ist. Die ersten
drei Eigenwerte $\lambda_1 = \lambda_2 = \lambda_3 = 0$ sind ja Null, weil die
zugeh\"{o}rigen Eigenvektoren $\vek u_i$ Starrk\"{o}rperbewegungen der Scheibe darstellen.

Wenn die Wechselwirkungsenergie Biege- und Membran-Anteile zusammen enth\"{a}lt, werden
manchmal nur die Schubanteile reduziert integriert. Dann spricht man von {\em selektiver
Integration\/}, \cite{Hughes}.\index{selektive Integration}

Bathe \cite{Bathe} warnt davor, die reduzierte Integration in kommerziellen Programmen
einzusetzen, weil die Zuverl\"{a}ssigkeit und Stabilit\"{a}t der Programme dadurch unn\"{o}tig
gef\"{a}hrdet w\"{u}rden. Unter streng kontrollierten Bedingungen kann die reduzierte
Integration jedoch sehr n\"{u}tzlich sein und Rechenzeit sparen. So werden bei der
Simulation von {\em crash tests\/} im Automobilbau sogenannte {\em stabilisierte hour
glass Elemente\/} verwendet, bei denen pro Element nur ein Gausspunkt verwendet wird,
\cite{Belytschko1} , \cite{Belytschko2}.

%%%%%%%%%%%%%%%%%%%%%%%%%%%%%%%%%%%%%%%%%%%%%%%%%%%%%%%%%%%%%%%%%%%%%%%%%%%%%%%%%%%%%

%%%%%%%%%%%%%%%%%%%%%%%%%%%%%%%%%%%%%%%%%%%%%%%%%%%%%%%%%%%%%%%%%%%%%%%%%%%%%%%%
{\textcolor{blue}{\section{Vergleichsrechnungen}}}\label{VergleichsrechnungenScheibe}\index{Vergleichsrechnung, Scheiben}
%%%%%%%%%%%%%%%%%%%%%%%%%%%%%%%%%%%%%%%%%%%%%%%%%%%%%%%%%%%%%%%%%%%%%%%%%%%%%%%%
Es w\"{a}re sch\"{o}n, wenn man FE-Berechnungen mit anderen FE-Programmen einfach gegenrechnen
k\"{o}nnte und verl\"{a}ssliche Antworten bek\"{a}me, aber wie die Erfahrung zeigt, streuen die
Ergebnisse {\em punktweise\/} manchmal doch relativ stark.

Die in Bild \ref{Benchmark1} skizzierte und verma{\ss}te 24 cm starke Stahlbetonwandscheibe
(E = 30\,000 MN/m$^2$, $\nu = 0.2$) wurde ohne angrenzende Deckenplatten untersucht,
wobei kein Eigengewicht sondern nur eine obere Randlast von 150 kN/m sowie drei
netzunabh\"{a}ngige Einzellasten von 300, 210 und 440 kN ber\"{u}cksichtigt wurden. F\"{u}r den
automatischen Netzgenerator wurde eine mittlere Elementkantenl\"{a}nge von 30 cm vorgegeben.

Als Variante B wurde die Wandscheibe danach statt mit den beiden elastischen St\"{u}tzen mit
starrer Lagerung untersucht. Dabei wurden jeweils die beiden Knoten an den
\"{U}bergangsstellen Scheibe--Pfeiler in vertikaler Richtung festgehalten.

Das Beispiel wurde mit verschiedenen FE-Programmen durchgerechnet, \cite{Schaper}. Die
Ergebnisse sind Schnappsch\"{u}sse und nicht die Ergebnisse von wissenschaftlichen
Reinraumversuchen unter genau kontrollierten Bedingungen, sondern es sollte einfach nur
demonstriert werden, wie gro{\ss} in der Praxis die Streuung sein kann.

Wie man an Tabelle \ref{TabBenchmark1} ablesen kann, liegen die Ergebnisse in der
Zugzone, Punkte 1 und 2, relativ dicht beieinander w\"{a}hrend sie in den Punkten 3 und 4,
wohl weil dort das Spannungsfeld komplexer ist, st\"{a}rker schwanken, s. Bild
\ref{Benchmark1Duo}.
%----------------------------------------------------------------------------------------------------------
\begin{figure}[tbp]
\if \bild 2 \sidecaption \fi
\includegraphics[width=1.0\textwidth]{\Fpath/BENCHMARK1}
\caption{Zweifeldrige Scheibe aus dem Hochbau} \label{Benchmark1}
\end{figure}%%
%----------------------------------------------------------------------------------------------------------
\begin{table}
\caption{ Maximale Zugspannungen $\sigma_{xx}$ in kN/m$^2$} \label{TabBenchmark1}
\begin{tabular}{rrrrrrr}
\noalign{\hrule\smallskip} \\[1mm]
Ergebnisort &      ANSYS &    CS-FEBA &  InfoGraph &    MicroFE &   Sofistik &        BEM \\
\noalign{\hrule\smallskip} \\[1mm]
 A Punkt 1 &       4324 &       4751 &       4710 &       5000 &       4640 &       4447 \\
 A Punkt 2 &       5490 &       6986 &       7090 &       6450 &       6890 &       5431 \\
 A Punkt 3 &        844 &        928 &       1090 &       1530 &        610 &        476 \\
 A Punkt 4 &       3044 &       3831 &       4690 &            &            &       2224 \\
           &            &            &            &            &            &            \\
 B Punkt 1 &       3597 &       3994 &       3930 &       4290 &       3780 &       4057 \\
 B Punkt 2 &       3063 &       4713 &       4440 &       3290 &       3650 &       3067 \\
 B Punkt 3 &       4181 &       4964 &       5540 &       5330 &       4820 &       3214 \\
 B Punkt 4 &       4290 &       4862 &       4900 &            &            &       3785 \\
\end{tabular}
\end{table}

%%%%%%%%%%%%%%%%%%%%%%%%%%%%%%%%%%%%%%%%%%%%%%%%%%%%%%%%%%%%%%%%%%%%%%%%%%%%%%%%%%%%%


%----------------------------------------------------------------------------
\begin{figure}
\centering
{\includegraphics[width=0.95\textwidth]{\Fpath/U83}}
  \caption{Die Biegelinie unter der Streckenlast $p$ ist die Einh\"{u}llende der Seilecke aus den Einzelkr\"{a}ften $dP$}
  \label{U83}
\end{figure}%%
%----------------------------------------------------------------------------

%%%%%%%%%%%%%%%%%%%%%%%%%%%%%%%%%%%%%%%%%%%%%%%%%%%%%%%%%%%%%%%%%%%%%%%%%%%%%%%%%%%%%


%----------------------------------------------------------------------
\begin{figure}[tbp] \centering
\if \bild 2 \sidecaption \fi
\includegraphics[width=1.0\textwidth]{\Fpath/WAND1}
\caption{Wandscheibe {\bf a)} Stockwerkslasten und Einzelkr\"{a}fte {\bf b)} Verformungen
{\bf c)} Hauptspannungen {\bf d)} qualitative Darstellung der horizontalen Bewehrung
as-x } \label{Wand1}
\end{figure}%%
%----------------------------------------------------------------------
%----------------------------------------------------------------------
\begin{figure}[tbp] \centering
\if \bild 2 \sidecaption \fi
\includegraphics[width=1.0\textwidth]{\Fpath/W1SCHNITTE}
\caption{Spannungen und Bewehrung {\bf a)} $\sigma_{xx}$ {\bf b)} $\tau_{xy}$ {\bf c)}
$\sigma_{yy}$ {\bf d)} Ausschnitt mit der Bewehrung as-y} \label{W1Schnitte}
\end{figure}%%
%----------------------------------------------------------------------

%%%%%%%%%%%%%%%%%%%%%%%%%%%%%%%%%%%%%%%%%%%%%%%%%%%%%%%%%%%%%%%%%%%%%%%%%%%%%%%%%%%%%

%----------------------------------------------------------------------
\begin{figure}[tbp] \centering
\if \bild 2 \sidecaption \fi
\includegraphics[width=.9\textwidth]{\Fpath/SHORT}
\caption{Einzelkraft {\bf a)} Modellierung als Linienkraft {\bf b)} Spannungen in einem
horizontalen Schnitt {\bf c)} Zug- und Druckspannungen oberhalb und unterhalb der
Linienkraft} \label{Short}
\end{figure}%%
%----------------------------------------------------------------------

%%%%%%%%%%%%%%%%%%%%%%%%%%%%%%%%%%%%%%%%%%%%%%%%%%%%%%%%%%%%%%%%%%%%%%%%%%%%%%%%%%%%%%%%%%%%%%%%%%%%
{\textcolor{blue}{\section{Gleichgewicht im Schnitt}}}\label{Gleichgewicht im Schnitt}\index{Gleichgewicht im Schnitt}
%%%%%%%%%%%%%%%%%%%%%%%%%%%%%%%%%%%%%%%%%%%%%%%%%%%%%%%%%%%%%%%%%%%%%%%%%%%%%%%%%%%%%%%%%%%%%%%%%%%%
Wir hatten in Kapitel \ref{EinflussfunktionenIntegrale} schon bemerkt, dass die
Schnittkr\"{a}fte der FE-L\"{o}sung in der Regel nicht mit der \"{a}u{\ss}eren Belastung im
Gleichgewicht sind, weil die FE-Programme die exakten Einflussfunktionen f\"{u}r die
Schnittkr\"{a}fte nicht darstellen k\"{o}nnen.
%----------------------------------------------------------------------------------------------------------
\begin{figure}[tbp] \centering
\if \bild 2 \sidecaption \fi
%\includegraphics[width=0.8\textwidth]{\Fpath/INFWAND1}
\includegraphics[width=1.0\textwidth]{\Fpath/EDISS4}
\caption{Wandscheibe {\bf a)} Schnitte A--A und B--B {\bf b)} exakte Einflussfunktion f\"{u}r die
horizontale Schnittkraft $N_{yx}$ in dem Schnitt A--A {\bf c)} FE-N\"{a}herung} \label{InfWand1}
\end{figure}%%%
%----------------------------------------------------------------------------------------------------------

So ist die Einflussfunktion f\"{u}r die horizontale Schnittkraft $N_{yx}$ (die Scherkraft)
in dem Schnitt A--A in Bild \ref{InfWand1} eine Versetzung des Tragwerksteils oberhalb
des Schnittes um eine L\"{a}ngeneinheit nach rechts.

Will man diese Einflussfunktion berechnen, so muss man die Knoten mit den entsprechenden
\"{a}quivalenten Knotenkr\"{a}ften belasten. Diese \"{a}quivalenten Knotenkr\"{a}fte sind die l\"{a}ngs A--A
aufintegrierten Scherspannungen $\sigma_{xy}$ aus den Einheitsverformungen $\vek \Np_i$ der Knoten
\bfoo
f_i = \int_{A-A} \sigma_{xy}^{(i)} \,dx \qquad \sigma_{xy}^{(i)} \mbox{= Scherspannung aus $\vek \Np_i$ im Schnitt A-A}\,.
\efoo
Betrachtet man das auf bilinearen Elementen basierende Resultat, die Verformungsfigur in
Bild \ref{InfWand1} c, dann ist anschaulich klar, dass das Gleichgewicht in dem Schnitt
A--A in der Regel verletzt ist. So betr\"{a}gt z.B. die Verschiebung in der oberen linken
Ecke nicht 1.0 sondern 2.09. Ein einfacher Test mit dem FE-Programm best\"{a}tigt dieses
Resultat: Von einer Kraft $P$ in der Ecke kommt das 2.09-fache im Schnitt A--A an
\bfoo
\int_{A-A} \sigma_{yx}^{h}\,dx = 2.09 \,P\,.
\efoo
Diese Abweichung ist sehr gro{\ss}. Bei einer \"{a}hnlichen Wandscheibe ohne \"{O}ffnungen mit dem
Verh\"{a}ltnis H\"{o}he:Breite = 2:1 ergab sich eine horizontale Verschiebung der Ecke von 1.3
L\"{a}ngeneinheiten. W\"{u}rde bei dem Beispiel in Bild \ref{InfWand1} eine konstante
Streckenlast (Wind) oberhalb des Schnittes angreifen, dann w\"{a}re die Schnittkraft $N_{yx}
= 1.58 \cdot W$, wenn $W$ die resultierende Windlast ware. Die 1.58 sind gerade der
Mittelwert aller horizontalen Knotenverschiebungen an der Verformungsfigur in Bild
\ref{InfWand1} c auf der dem Wind zugewandten Seite der Scheibe oberhalb vom Schnitt A--A.

Eine Berechnung mit dem Element von Wilson, Q4+2, ergab statt 2.09 den Faktor 1.41, also
ein deutlich besseres Ergebnis als mit dem bilinearen Element. Dies ist auch ein Hinweis
darauf, dass das Gleichgewicht (\"{u}ber Kreuz\footnote{Wir sollten nicht vergessen, dass
wir die Schnittkr\"{a}fte des LF $p_h$ mit den Lasten aus dem LF $p$ vergleichen.}) in dem
Schnitt A--A wesentlich davon abh\"{a}ngt, wie gut die Biegeverformungen der Stiele und Riegel
von dem FE-Programm dargestellt werden k\"{o}nnen.

Setzt man eine horizontale Einzelkraft $P = 1$ in die obere linke Ecke und rechnet mit
dem Element von Wilson, dann ist in dem Schnitt B--B, siehe Bild \ref{InfWand1}, die
Schnittkraft $N_{yx}^h$ identisch mit der Eckkraft. Das d\"{u}rfte darauf zur\"{u}ckzuf\"{u}hren
sein, dass in dem Schnitt B--B die Biegeverformungen wesentlich geringer sind, als in
dem Schnitt A--A.

%%%%%%%%%%%%%%%%%%%%%%%%%%%%%%%%%%%%%%%%%%%%%%%%%%%%%%%%%%%%%%%%%%%%%%%%%%%%%%%%%%%%%%%%%%%%%%%%%%%%

%----------------------------------------------------------------------------
\begin{figure}
\centering
{\includegraphics[width=0.8\textwidth]{\Fpath/U222}}
  \caption{Plot der Knotenvektoren $\vek g_i$ des Funktionals $J(u_h) = \sigma_{yy}$, also der vertikalen Spannung im Rissgrund. Die {\em Lagrangepunkte\/} sind die Punkte, in denen der Einfluss der Knotenkr\"{a}fte $\vek f_i$ auf $\sigma_{yy}$ praktisch null ist}
  \label{U222}
\end{figure}%%
%----------------------------------------------------------

%%%%%%%%%%%%%%%%%%%%%%%%%%%%%%%%%%%%%%%%%%%%%%%%%%%%%%%%%%%%%%%%%%%%%%%%%%%%%%%%%%%%%%%%%%%%%%%%%%%%

Weil man bei einer Rahmenberechnung auf dem Bildschirm immer die FE-L\"{o}sung plus der lokalen L\"{o}sung sieht, kann man die Methode der finiten Elemente an Rahmen eigentlich gar nicht erl\"{a}utern, weil die lokale L\"{o}sung immer den Blick auf die L\"{o}sung darunter, die FE-L\"{o}sung, verstellt. Man muss pausieren, die Addition der lokalen L\"{o}sung auf sp\"{a}ter verschieben, wenn man einem Laien die Methode der finiten Elemente erl\"{a}utern will. So haben wir es in den Abbildungen XX und YY gemacht. Es sind Momentaufnahmen vor der Addition der lokalen L\"{o}sung.

%%%%%%%%%%%%%%%%%%%%%%%%%%%%%%%%%%%%%%%%%%%%%%%%%%%%%%%%%%%%%%%%%%%%%%%%%%%%%%%%%%%%%%%%%%%%%%%%%%%%

Einflussfunktionen sind ein klassisches Thema der Statik im Grunde so alt wie das Hebelgesetz von Archimedes
Das spielerische Drehen einer Waage hat Archimedes auf sein Hebelgesetz, $ P_l \, h_l = P_r \, h_r$, gef\"{u}hrt und damit indirekt auch auf den Begriff der Einflussfunktion, $P_l = h_r/h_l \, P_r$, mit der man die linke Kraft $P_l$ aus der rechten Kraft $P_r$ berechnen kann.

Das sind zwei Begriffe, die zentral f\"{u}r die finiten Elemente sind. Von den virtuellen Arbeiten und virtuellen Verr\"{u}ckungen wissen wir es, aber f\"{u}r viele Leser d\"{u}rfte es neu sein, das ein FE-Programm alles, was wir auf dem Bildschirm oder im Ausdruck sehen, mit (gen\"{a}herten) Einflussfunktionen berechnet---ganz wie die klassische Statik auch.





%%%%%%%%%%%%%%%%%%%%%%%%%%%%%%%%%%%%%%%%%%%%%%%%%%%%%%%%%%%%%%%%%%%%%%%%%
{\textcolor{blue}{\section{Endliche und unendliche Energie}}}\label{Endliche und unendliche Energie}\index{endliche und unendliche Energie}
%%%%%%%%%%%%%%%%%%%%%%%%%%%%%%%%%%%%%%%%%%%%%%%%%%%%%%%%%%%%%%%%%%%%%%%%%
Belastet man einen Stab mit zwei  entgegengesetzt gerichteten Einzelkr\"{a}ften, die um so
gr\"{o}{\ss}er werden, je kleiner ihr Abstand $\Delta x$ wird, s. Bild \ref{Riss},
\begin{align}
P = \frac{1}{\Delta x} \qquad   P = - \frac{1}{\Delta x}\,,
\end{align}
und l\"{a}sst man den Abstand $\Delta x$ gegen null gehen, so erh\"{a}lt man als
Horizontalverschiebung $u$ des Stabes die Treppenfunktion in Bild \ref{Riss}. An dem
Punkt, an dem die beiden Kr\"{a}fte sich schlie{\ss}lich treffen, rei{\ss}t es den Stab auseinander.
Die beiden Risskanten verschieben sich um den Betrag $\Delta u = 1/EA$ gegeneinander.
Die Horizontalverschiebung $u$ springt dort also um diesen Betrag. Sie ist dort unstetig.
%----------------------------------------------------------------------------------------------------------
\begin{figure}[tbp] \centering
\if \bild 2 \sidecaption \fi
\includegraphics[width=.6\textwidth]{\Fpath/RISS}
\caption{Je n\"{a}her die Einzelkr\"{a}fte zusammenr\"{u}cken, um so gr\"{o}{\ss}er werden sie, bis der Stab
zerrei{\ss}t. Die horizontale Verschiebung wurde, um ihre Gr\"{o}{\ss}e zeigen zu k\"{o}nnen, nach unten
(+) und oben (-) abgetragen} \label{Riss}
\end{figure}%
%----------------------------------------------------------------------------------------------------------
Nicht \"{u}berraschend w\"{a}chst bei der Ann\"{a}herung der beiden Kr\"{a}fte auch die innere Energie
\"{u}ber alle Grenzen ($EA = 1$)
\begin{align}
\frac{1}{2}\int_0^{\,l} \frac{N^2}{EA}dx = \frac{1}{2}\int_0^{\,l} \frac{1}{{\Delta
x}^2} dx = \frac{1}{2} \frac{1}{{\Delta x}^2} \Delta x = \frac{1}{2}\frac{1}{\Delta x} =
\infty \qquad  \Delta x \mapsto 0 \,.
\end{align}
Betrachten wir ein zweites Beispiel: Belastet man einen Balken mit zwei entgegengesetzt
drehenden Einzelmomenten, die um so gr\"{o}{\ss}er werden, je kleiner der Abstand $\Delta x$
zwischen ihnen ist,
\begin{align}
M = \frac{1}{\Delta x} \qquad  M  = - \frac{1}{\Delta x}\,,
\end{align}
so bildet sich beim Grenz\"{u}bergang $\Delta x \mapsto 0$ ein Knick, ein plastisches
Gelenk, aus, und die innere Energie w\"{a}chst wieder \"{u}ber alle Grenzen
\begin{align}
 \frac{1}{2}\int_0^{\,l}
\frac{M^2}{EI} dx = \frac{1}{2}\int_0^{\,l} \frac{1}{{\Delta x}^2} dx =
\frac{1}{2}\frac{1}{\Delta x} = \infty \qquad  \Delta x \mapsto 0\,.
\end{align}
%----------------------------------------------------------------------------------------------------------
\begin{figure}[tbp] \centering
\if \bild 2 \sidecaption \fi
\includegraphics[width=.5\textwidth]{\Fpath/KNICK}
\caption{Je n\"{a}her die Momente aneinander r\"{u}cken, um so gr\"{o}{\ss}er werden sie, bis der Balken
knickt} \label{RissM}
\end{figure}%
%----------------------------------------------------------------------------------------------------------
Was wir mit diesen Beispielen sagen wollen ist: Wenn man einen Stab zerrei{\ss}t oder einen
Balken knickt, dann geht das immer nur mit roher Gewalt. Unendlich gro{\ss}e Kr\"{a}fte sind
dazu n\"{o}tig, und daher ist auch die Verzerrungsenergie unendlich gro{\ss}. Das Material
beginnt zu plastifizieren, und es bilden sich Flie{\ss}gelenke aus.

Mathematisch ist ein solcher Riss in einem Stab oder ein solcher Knick in einem Balken
eine {\em Unstetigkeit\/} in einer Weggr\"{o}{\ss}e -- beim Balken z\"{a}hlt die Tangentenneigung
$w{'}$ zu den Weggr\"{o}{\ss}en. Und die Botschaft ist, dass solche Unstetigkeiten in den
Weggr\"{o}{\ss}en unendliche Energie bedeuten.

%%%%%%%%%%%%%%%%%%%%%%%%%%%%%%%%%%%%%%%%%%%%%%%%%%%%%%%%%%%%%%%%%%%%%%%%%

%----------------------------------------------------------
\begin{figure}[tbp]
\centering
\includegraphics[width=0.85\textwidth]{\Fpath/U77}
\caption{Deckenplatte Einflussfunktionen \textbf{ a)} f\"{u}r eine Durchbiegung, \textbf{ b)} f\"{u}r eine Querkraft, \textbf{ c)} f\"{u}r ein Moment}
\label{U77}%
\end{figure}%%
%----------------------------------------------------------
%%%%%%%%%%%%%%%%%%%%%%%%%%%%%%%%%%%%%%%%%%%%%%%%%%%%%%%%%%%%%%%%%%%%%%%%%


%----------------------------------------------------------------------------------------------------------
\begin{figure}[tbp] \centering
\if \bild 2 \sidecaption \fi
\includegraphics[width=.8\textwidth]{\Fpath/DualraeumeD}
\caption{Der Ansatzraum $\mathcal{V}_{h} \subset \mathcal{V}$ und der dazu duale Raum $P_{h}  \subset P$ mit
all den Lastf\"{a}llen, die die Biegelinien in $\mathcal{V}_{h}$ erzeugen} \label{Dualraeume}
\end{figure}%
%----------------------------------------------------------------------------------------------------------

\begin{remark}
Auch die Kontrolle der Lagerkr\"{a}fte, die $\sum V$ und die $\sum H$, an einem Tragwerk sind im Grunde 'Wackeltests', wie jedes Kind wei{\ss}, dass mit Baukl\"{o}tzen spielt. Wir schneiden das Tragwerk frei und erteilen dem Tragwerk eine Verschiebung $u = 1$ bwz. $w = 1$ und z\"{a}hlen die Arbeit, die die Lagerkr\"{a}fte $A$ und $B$ und die Belastung bei dieser Verschiebung leisten.

\end{remark}

Das Thema ist hochinteressant, weil es direkt auf das Verst\"{a}ndnis des Ingenieurs zielt, der ja beim Tragwerksentwurf mit solchen Fragen konfrontiert ist. Nicht unbedingt im Sinne einer adaptiven Verfeinerung, aber im Sinne einer Nachverfolgung von Effekt und Ursache, wie entscheidet {\em design\/} das Tragverhalten.




Dem \"{u}berlagert sich ein zweites Thema, der Unterschied zwischen schubstarren und schubweichen Plattenelementen. Bis zu den finiten Elemente wurden Platten schubstarr gerechnet. Der relativ hohe numerische Aufwand f\"{u}r schubstarre Elemente ($C^1$-Elemente) hat jedoch bei der FE-Modellierung zu einer Verschiebung der Gewichte in Richtung der schubweichen Platten ($C^0$-Elemente) gef\"{u}hrt. Die meisten FE-Programme benutzen heute schubweiche Elemente.

%%%%%%%%%%%%%%%%%%%%%%%%%%%%%%%%%%%%%%%%%%%%%%%%%%%%%%%%%%%%%%%%%%%%%%%%%%%%%%%%%%5
\begin{figure}[tbp]
\if \bild 2 \sidecaption \fi
\includegraphics[width=0.85\textwidth]{\Fpath/LAGER4}
\caption{Die Lagerkr\"{a}fte im Ausdruck sind die in Linienkr\"{a}fte umgerechneten \"{a}quivalenten
Knotenkr\"{a}fte {\bf a)} Die obere Platte wurde auf starren Lagern gerechnet, $EA = \infty$
und die untere Platte {\bf b)} auf Mauerwerk. Starre Lager vergr\"{o}{\ss}ern der Tendenz nach
die Ausschl\"{a}ge der Lagerkr\"{a}fte} \label{Lager4}
\end{figure}%%

%%%%%%%%%%%%%%%%%%%%%%%%%%%%%%%%%%%%%%%%%%%%%%%%%%%%%%%%%%%%%%%%%%%%%%%%%%%%%%%%%%5
\begin{figure}[tbp]
\if \bild 2 \sidecaption \fi
\includegraphics[width=1.0\textwidth]{\Fpath/PLATTEN10}
\caption{Wohnhausdecke \"{u}ber W\"{a}nden mit T\"{u}r- und Fenster\"{o}ffnungen, Deckenst\"{a}rke $h = 0.2$
m {\bf a)} System LF $g$ {\bf b)} Momente $m_{xx}$ {\bf c)} Hauptmomente {\bf d)}
Momente \"{u}ber den Innenw\"{a}nden} \label{Platten10}
\end{figure}%%


Diese singul\"{a}ren Anteile w\"{a}ren auch dann noch dominant, wenn wir die Einzelkraft $P$
durch die Pressung $p = P/|\Omega_K|$, am St\"{u}tzenkopf ($\Omega_K$) ersetzen
w\"{u}rden
\begin{align}
m_{\,ij}(\vek x) = m_{\,ij}^{p}(\vek x) + \int_{\Omega_K} m_{\,ij}^{s}(\vek y,\vek
x)\,p(\vek y) \,d\Omega_{\vek y}\,,
\end{align}
denn nicht weit weg von der St\"{u}tze w\"{a}re das Integral n\"{a}herungsweise identisch mit dem
Einfluss der im Mittelpunkt $\vek y_S$ der St\"{u}tze konzentriert gedachten St\"{u}tzenkraft $P$
\begin{align}
\int_{\Omega_K} m_{\,ij}^{s}(\vek y,\vek x)\,p(\vek y) \,d\Omega_{\vek y} \cong P
\,m_{\,ij}^{s}(\vek y_S,\vek x)\,.
\end{align}
Die Schwierigkeiten der FEM r\"{u}hren daher, dass man mit Polynomen die Singul\"{a}rfunktionen
$m_{\,ij}^{s}$ schlecht approximieren kann.


%-------------------------------------------------------------------------
\begin{figure}[tbp]
\if \bild 2 \sidecaption \fi
\includegraphics[width=.8\textwidth]{\Fpath/MODELLSTUETZEN}
\caption{Modellierung einer punktf\"{o}rmigen Lagerung}\label{Modellstuetzen}
\end{figure}%%
%-------------------------------------------------------------------------

%-------------------------------------------------------------------------
\begin{figure}[tbp]
\centering
\if \bild 2 \sidecaption \fi
\includegraphics[width=0.9\textwidth]{\Fpath/U490}
\caption{Dieselbe Platte wie zuvor {\bf a)} Einflussfunktion f\"{u}r das Anschnittmoment $m_{xx}$ an der Kante der St\"{u}tze, {\bf b)} Einflussfunktion f\"{u}r die Normalkraft in der St\"{u}tze. Weil sich die Einflussfl\"{a}che nach oben w\"{o}lbt, ist das Moment $m_{xx}$ aus einer abw\"{a}rts gerichteten Kraft negativ } \label{U490}
\end{figure}%%
%-------------------------------------------------------------------------
{\textcolor{blue}{\subsection{Beispiel}}}
 Wir betrachten die Wand in Abb. \ref{WandInf}.
Die exakte Einflussfl\"{a}che $G_W$ f\"{u}r die Lagerkraft der vertikal verlaufenden Wand ist in Abb. \ref{WandInf} b dargestellt und die FE-N\"{a}herung in Abb. \ref{WandInf} d. Die letztere
erh\"{a}lt man, wenn man die Knoten der Wand um $w = 1 $ cm absenkt. Der in Abb. \ref{WandInf} d markierte Knoten $\vek y_k$ senkt sich dabei um 1.8504747 cm ab. Das ist
umgekehrt, bis auf die letzte Stelle nach dem Komma, auch genau die Summe der
\"{a}quivalenten Knotenkr\"{a}fte in der Wand, wenn man in den markierten Knoten eine Kraft $P =
1$ kN stellt
\begin{align}
\sum_i f_i = \int_{\Omega} G_W^h(\vek y) \,p(\vek y)\,\,d\Omega =
G_W^h(\vek y_k) \times 1 \,\mbox{kN} = 1.8504747\, \mbox{kN}\,.
\end{align}
Der genaue Wert (auf einem sehr feinen Netz) f\"{u}r die Lagerkraft betr\"{a}gt 2.2 kN, was
bedeutet, dass die FE-L\"{o}sung die Lagerkraft auf dem obigen Netz um 16 \% Prozent
untersch\"{a}tzt. F\"{u}r den Lastfall Vollast ergab sich ein Fehler in der Lagerkraft von 5 \%
und bei einer Belastung des Feldes rechts von der Wand ebenfalls ein Fehler von 5 \%.

{\textcolor{blue}{\subsection{Ausrei{\ss}er}}\index{Ausrei{\ss}er}
Gerade bei Lagerkr\"{a}ften gibt es oft unangenehme {\em peaks\/} und
Oszillationen. Besonders betroffen davon sind die Enden von frei stehenden W\"{a}nden oder
die Ecken von Innenw\"{a}nden, s. Abb. \ref{MEcke}. Warum das so ist, versteht man, wenn man
sich die Einflussfl\"{a}chen f\"{u}r die Knotenkraft vorne an der Kante, Knoten A, bzw. f\"{u}r
einen r\"{u}ckw\"{a}rtigen Knoten, Knoten B, anschaut. Die Knoten vorne an der Kante haben ein
viel gr\"{o}{\ss}eres Einzugsgebiet als die weiter hinten liegenden Knoten, und es wir so auch
klar, warum in dem Knoten hinter der Kante oft Zugkr\"{a}fte auftreten -- einfach weil eine
Bewegung dieses Knotens nach unten den vor der Wand liegenden Teil der Platte nach oben
zieht, und die Einflussfl\"{a}che dort damit negativ wird.


%%%%%%%%%%%%%%%%%%%%%%%%%%%%%%%%%%%%%%%%%%%%%%%%%%%%%%%%%%%%%%%%%%%%%%%%%%%%%%%%%%%%%%%%%%%%%%%%%%%%%%%%
\textcolor{chapterTitleBlue}{\section{One-Click Reanalysis}}
Wir hatten oben davon gesprochen, dass man die neue L\"{o}sung $\vek u_c$
\begin{align}
(\vek K + \vek \Delta K)\,\vek u_c = \vek f
\end{align}
entweder durch Iteration oder durch eine direkte L\"{o}sung (eines kleinen Hilfssystems) berechnen kann.

In einem Rahmenprogramm, BE-FRAMES, s. S. \pageref{SoftwareDownload}, sind diese beiden Techniken, Iteration und direkte L\"{o}sung, zu Lehrzwecken als {\em One-Click Reanalysis\/} implementiert, s. Abb. \ref{U509}. Der Student kann durch einfache Klicks Ver\"{a}nderungen an einem Rahmen vornehmen und die Effekte, die dadurch entstehen, studieren\index{BE-FRAMES}.

Implementiert sind \"{A}nderungen der Art $\vek \Delta \vek K_e = c \cdot \vek K_e$,
also eine Skalierung einzelner Elementmatrizen mit einem Faktor $c$, wobei $c = 0$ einem kompletten Verlust des Elements entspricht.

Der Student w\"{a}hlt einen Faktor $c$ und klickt dann nacheinander auf die einzelnen Elemente, um zu sehen, wie das Tragwerk auf die Modifikationen reagiert. (Der Faktor $c$ kann unterschiedlich gew\"{a}hlt werden).

Eine Serie von Modifikationen, etwa \"{A}nderungen der Elemente  5, 7 und 9, bedeutet einfach, dass $\vek \Delta \vek K$ eine Ansammlung von skalierten Elementmatrizen $\vek \Delta \vek K_e$ ist
\begin{align}
\vek u_c = - \vek K^{-1}\,(\vek \Delta\,\vek K_5 + \vek \Delta\,\vek K_7 + \vek \Delta\,\vek K_9 )\,\vek u_c + \vek u \qquad \text{(direkte Lsg.)}
\end{align}
Es k\"{o}nnen auch einzelne Lager entfernt werden, nachtr\"{a}glich Gelenke eingebaut werden (das geht mit Dirac Deltas) und Einflussfunktionen f\"{u}r alle interessierenden Gr\"{o}{\ss}en berechnet werden.

Alternativ kann man nat\"{u}rlich auch das Originalsystem modifizieren und neu l\"{o}sen, denn die Computer sind heute so schnell geworden, dass auch das kaum noch Zeit in Anspruch nimmt und so der {\em overhead\/}, den die ersten beiden Techniken mit sich f\"{u}hren, umgangen wird.

Um gleich konkret zu werden, betrachten wir eine Kugelschale $\Gamma$ auf der ein \"{o}rtlich variierendes Potential $u$ (= Spannung) vorliege auf der elektrische Ladungen
Die Potentialtheorie besch\"{a}ftigt sich mit den Feldern, die von Punktladungen, die auch verteilte Ladungen sein k\"{o}nnen, erzeugt werden. Sitzen auf der Innenseite einer Schale positive Ladungen und auf der Au{\ss}enseite negative Ladungen, dann nennt man eine solche Belegung eine {\em dipolartige\/} Belegung. nahm ihren Anfang mit dem Studium des elektrischen Felds,
\begin{align}
u(\vek y) = c\,\frac{1}{r}  \qquad r = |\vek y - \vek x|
\end{align}
das von einer Punktladung erzeugt wird. Dieses Feld gen\"{u}gt au{\ss}erhalb des Aufpunkt der Differentialgleichung $- \Delta u = 0$. Die Potentialtheorie besch\"{a}ftigt sich mit solchen Feldern. Die zentrale Rolle der Potenzialtheorie wird klar, wenn man sich daran erinnert welche gro{\ss}e Bedeutung der Feldbegriff in der Physik hat. Die Potenzialtheorie ist im Grunde nur Anwendung der partiellen Integration. Der Gau{\ss}sche Integralsatz in der Elektrostatik wie auch der {\em Faradaysche K\"{a}fig\/} sind direkte Konsequenzen des Feldbegriffs.

Wenn wir uns hier der Einfachheit halber auf ebene Probleme beschr\"{a}nken, dann gilt, dass man jede L\"{o}sung der {\em Poisson Gleichung\/} $- \Delta u = p$ als ein Feld lesen kann,
\begin{align}
u(\vek x) = \int_{\Gamma} (g(\vek y, \vek x) \,t(\vek y) - \frac{\partial g}{\partial n}\,u(\vek y))\,ds_{\vek y} + \int_{\Omega} g(\vek y, \vek x)\,p(\vek y)\,\,d\Omega_{\vek y}
\end{align}
das von Ladungen $u, \partial u/ \partial n$ (Monopolen und Dipolen) auf dem Rand $\Gamma $ des Gebiets $\Omega$ und einer Ladung $p $ im Innern erzeugt wird, die um einen Aufpunkt $\vek x$ verteilt sind.

Die Funktion
\begin{align}
g(\vek y, \vek x) =
\end{align}
ist das Potenzial, das von einer Punktladung, die im Punkt $\vek x$ sitzt, in der unendlich ausgedehnten Ebene erzeugt wird.

Wenn man genau hinschaut, dann erkennt man, dass das eine Einflussfunktion ist und das leitet \"{u}ber zur Statik. Alle Einflussfunktionen in der Statik sind genau so aufgebaut, wenn nat\"{u}rlich auch die Bedeutung der einzelnen Terme differiert. Jeder, der mit der Potenzialtheorie vertraut ist, wird sehr schnell in der Statik ihm bekannte Ph\"{a}nomene wiedererkennen. Das gilt \"{u}brigens nicht nur f\"{u}r die Statik sondern eigentlich f\"{u}r die ganze Physik, denn der Feldbegriff ist einer der fundamentalen Begriffe der Physik ohne den auch die Quantenmechanik nicht auskommt.

Man denke z.B. an die Mohrsche Arbeitsgleichung
\begin{align}
\bar{1} \cdot \delta = [\text{kN}] \cdot [\text{m}] = \int_{0}^{l}\frac{M\,\bar{M}}{EI}\,dx = \int_{0}^{l} \frac{[\text{kNm}]^2}{[\text{kNm$^2$}]}\,[\text{m}] = [\text{kN}]\cdot [\text{m}]
\end{align}



{\small {\em Anmerkung:} Die virtuelle innere Arbeit $\delta A_i$ des Waagebalkens ist
null, weil wir den Waagebalken hier wie einen starren K\"{o}rper behandeln. Das \"{a}ndert aber
nichts an der Logik. Ein starrer K\"{o}rper ist genau dann im Gleichgewicht, wenn die
virtuelle \"{a}u{\ss}ere Arbeit null ist
\begin{align}
 \mbox{Gleichgewicht} \qquad \Longleftrightarrow \qquad  \delta
A_a =  0\,.
\end{align}
}%ENDSMALL


%%%%%%%%%%%%%%%%%%%%%%%%%%%%%%%%%%%%%%%%%%%%%%
{\textcolor{blue}{\section{Die FEM und das Drehwinkelverfahren}}}
%%%%%%%%%%%%%%%%%%%%%%%%%%%%%%%%%%%%%%%%%%%%%%
\glqq Wenn man einen Rahmen mit einem FE-Programm berechnet, ist
dann die L\"{o}sung exakt oder nur eine N\"{a}herung?\grqq

Sie ist dann exakt, wenn man den Rahmen auch mit dem Drehwinkelverfahren behandeln
k\"{o}nnte, denn f\"{u}r Standardf\"{a}lle -- also ohne gevoutete Tr\"{a}ger oder Fischbauchtr\"{a}ger -- ist
das Ergebnis der FEM mit den Ergebnissen nach dem Drehwinkelverfahren identisch. Das
liegt im wesentlichen daran, dass bei der Reduktion der Belastung in die Knoten sich die
Knotenverformungen nicht \"{a}ndern. Dies wollen wir im folgenden erl\"{a}utern.
%----------------------------------------------------------------------------------------------------------
\begin{figure}[tbp] \centering
\if \bild 2 \sidecaption \fi
\includegraphics[width=.6\textwidth]{\Fpath/ZERLEGUNG}
\caption{Jede Biegelinie kann in zwei Anteile zerlegt werden}
\label{Zerlegung}%
\end{figure}%%
%----------------------------------------------------------------------------------------------------------

Wir beginnen mit einer Beobachtung: Die Biegelinie $w = w_0 + w_p$ eines Balkens kann
man immer in zwei Anteile, eine {\em homogene Biegelinie}\index{homogene Biegelinie} $w_0$ (eine 'Null-L\"{o}sung', $p  = 0$) und eine {\em partikul\"{a}re Biegelinie}\index{partikul\"{a}re Biegelinie} $w_p$, zerlegen, s. Abb. \ref{Zerlegung},\\

\bite
\item{$w_0 = w_1 \,\Np_1(x) + w_2 \,\Np_2(x) + w_3 \,\Np_3(x) + w_4 \,\Np_4(x)$ }
\item{$w_p = $ Biegelinie aus $p$ am beidseitig eingespannten Balken}
\eite
Die Biegelinie $w_0$ ist eine homogene L\"{o}sung der Balkengleichung,
$EI w^{IV}_0 = 0$, w\"{a}hrend $w_p$ eine partikul\"{a}re L\"{o}sung ist, $EI w^{IV}_p = p$. Die
Biegelinie $w_0$ ist der 'Tr\"{a}ger' der Balkenendverformungen $u_i$. Die
Biegelinie $w_p$ dagegen ist der 'Tr\"{a}ger' der Belastung $p$. Sie ist 'stumm' an den Balkenenden.

Entsprechend kann man auch die Verformungsfigur eines Rahmens in zwei Anteile zerlegen:
Die Bewegungen auf Grund der Knotenverformungen und die
Verformungen, die stabweise aus der Querbelastung $p$ hinzukommen. Die
Knotenverformungen sind die eigentlich ma{\ss}gebenden Bewegungen, weil sie zur {\em
Interaktion} zwischen den einzelnen Bauteilen f\"{u}hren.

Indem also die FEM die Bewegungen eines Rahmens nach den {\em Einheitsverformungen}
$\Np_i$ der Knoten entwickelt, die ja alle Null-L\"{o}sungen sind,
\begin{align}
w_0(x) = \sum_i w_i \Np_i(x)\,,
\end{align}
vernachl\"{a}ssigt sie die Anteile $w_p$, d.h. sie tut so, als ob alle Riegellasten null
w\"{a}ren.

Weil man mit solchen Einheitsverformungen nur Knotenlastf\"{a}lle l\"{o}sen kann, reduziert die
FEM daher die Streckenlasten in die Knoten. Hierzu l\"{a}sst sie die Belastung $p$ gegen die
Einheitsverformungen $\Np_i(x)$ arbeiten
\begin{align}
\!\!(-1) \times \mbox{Festhaltekr\"{a}fte} = p_{\,i} &= \il p \, \Np_i \, dx\\
 &= \il p
\,\Np_i\, dx = p_{\,i} = \mbox{\"{a}quivalente Knotenkraft}
\end{align}
und stellt in die Knoten Knotenkr\"{a}fte
$p_i$, die dabei die gleiche Arbeit leisten
\begin{align}
\delta A_a(p,\Np_i) =  p_i \times 1 \,.
\end{align}
Zu diesen $p_i$ in den Knoten addiert sie nun noch die Einzelkr\"{a}fte $f_i$, die direkt in den Knoten angreifen. Das Gleichgewicht an den Knoten ist dann identisch mit dem Gleichungssystem
\begin{align}\label{System10}
\vek K \vek w = \vek f + \vek p\,.
\end{align}
Um auf die gewohnte Gestalt $\vek K \vek w = \vek f$ zu kommen, m\"{u}ssen wir den Vektor $\vek f + \vek p$ durch einen neuen Vektor $\vek f$ ersetzen
\begin{align}
\vek f = \vek f + \vek p\,,
\end{align}
weil die FEM alles in einen Vektor packt, den wir hier, um mit der FEM konform zu gehen, auch $\vek f$ genannt haben.

Diese Technik ist aber mit dem Drehwinkelverfahren identisch: Erst reduziert man die Belastung in die Knoten, berechnet die Festhaltekr\"{a}fte ({\em reactio\/}), l\"{o}st dann den ersten Knoten, gleich den Knoten aus  und leitet die Ausgleichskr\"{a}fte an den n\"{a}chsten Knoten weiter. Das ist die L\"{o}sung von $\vek K\,\vek w = \vek f$ in Einzelschritten, ist das {\em Gauss-Jordan Verfahren\/}.

Die Voraussetzung ist nat\"{u}rlich, dass die Ansatzfunktionen, die das FE-Programm
benutzt, die exakten Einheitsverformungen $\Np_i$ sind. Die $p_i$, oder die Dr\"{u}cke, die die Belastung in den Lagern generiert, sind gegengleich zu den Festhaltekr\"{a}ften. Also sind die Einheitsverformungen $\Np_i$ die Einflussfunktionen f\"{u}r die umgedrehten Festhaltekr\"{a}fte.

Ein FE-Programm ist also eine
Implementierung des Drehwinkelverfahrens als 'Gesamtschrittverfahren'\index{Gesamtschrittverfahren}.

Bei dieser Gelegenheit noch eine Bemerkung: Das Drehwinkelverfahren und damit die FEM funktionieren, weil es keinen Unterschied macht, ob man die Riegellasten in die Knoten reduziert, oder auf dem Riegel l\"{a}sst! Die Knotenverformungen sind dieselben. Dies gilt leider nur bei Stabtragwerken (gew\"{o}hnlichen Differentialgleichungen).

Die FEM benutzt das System (\ref{System10}) also nur, um die Knotenverformungen zu berechnen. Danach ist die FE-L\"{o}sung 'vergessen', denn die Nachbearbeitung, die Berechnung
der Schnittkr\"{a}fte und der Verformungen, geschieht elementweise durch Umstellung der
Beziehung
\begin{align}\label{ElementGlg}
\vek K^{\,e} \vek u^{\,e} =\vek f^{\,e} + \vek p^{\,e} \qquad \Rightarrow \qquad \vek
f^{\,e} = \vek K^{\,e}\,\vek u^{\,e} - \vek p^{\,e}
\end{align}
am einzelnen Element. Mit der Kenntniss der Balkenendkr\"{a}fte $\vek f^{\,e}$ kann man dann von Hand die
Schnittkr\"{a}fte l\"{a}ngs des Balkens berechnen. Programme benutzen hierzu meist
\"{U}bertragungsmatrizen.



%%%%%%%%%%%%%%%%%%%%%%%%%%%%%%%%%%%%%%%%%%%%%%%%%%%%%%%%%%%%%%%%%%%%%%%%
{\textcolor{blue}{\section{Genauigkeit}}}\label{Genauigkeit}
%%%%%%%%%%%%%%%%%%%%%%%%%%%%%%%%%%%%%%%%%%%%%%%%%%%%%%%%%%%%%%%%%%%%%%%%
Bei FE-Berechnungen gibt es den Modellfehler und den numerischen Fehler. Die Modellfehler kann man einem
FE-Programm nicht anlasten, denn er beruht darauf, dass der Aufsteller ein FE-Modell erstellt hat,
dass mit dem wahren Tragwerk nicht deckungsgleich ist. Der Modellfehler ist nat\"{u}rlich, weil er doch
relativ h\"{a}ufig ist, jede Untersuchung wert, aber wir wollen uns hier auf den numerischen Fehler beschr\"{a}nken, also
so tun, als ob es keinen Modellfehler gibt.

Grob gesagt ist die FE-L\"{o}sung dort genau, wo die exakte L\"{o}sung relativ glatt verl\"{a}uft, also im Feld, und sie kommt dort in Schwierigkeiten, wo auch die exakte L\"{o}sung diverse H\"{u}rden nehmen muss, vorzugsweise wo die Schnittgr\"{o}{\ss}en springen. Der Mathematiker w\"{u}rde sagenDie einfachste Skala ist die Ordnung der Ableitungen.
Je h\"{o}her die Ableitung der gesuchten Gr\"{o}{\ss}e ist, desto ungenauer werden die Ergebnisse. Das ergibt bei einer
schubstarren Platte die folgenden vier Abstufung
\begin{align}
w \qquad w,_x, w,_y \qquad m_{xx}, m_{yy}, m_{xy} \qquad q_x, q_y\,.
\end{align}
die Einflussfunktion f\"{u}r die Durchbiegung in einem Punkt ist eine einfache Delle in der Platte und die
l\"{a}sst sich auch auf groben Netzen schon relativ gut ann\"{a}hern. Am m\"{u}hsamsten ist es die Einflussfunktion f\"{u}r die
Querkr\"{a}fte zu berechnen, weil diese Einflussfunktionen von zwei Spitzen, $\pm \infty $, sehr
rasch auf null abfallen.


%----------------------------------------------------------------------------------------------------------
\begin{figure}[tbp]
\centering
\includegraphics[width=1.0\textwidth]{\Fpath/U512}
\caption{Der LF g bei einer Platte, \textbf{ a)} Durchbiegung $w$, \textbf{ b)} Momente $m_{xx}$, \textbf{ c)} Querkr\"{a}fte $q_x$. \"{A}hnliche Spitzen werden auch auftreten, wenn das Programm die Einflussfunktionen f\"{u}r die Schnittgr\"{o}{\ss}en berechnet. Sie beeintr\"{a}chtigen die G\"{u}te der Einflussfunktionen}
\label{U512}%
\end{figure}%%
%--------------------------------------------------------------------------------------------------
%----------------------------------------------------------------------
\begin{figure}[tbp]
\centering
\if \bild 2 \sidecaption \fi
\includegraphics[width=.99\textwidth]{\Fpath/U520}
\caption{Die Hauptmomente im LF $g$ der Platte in Abb. } \label{U520}
\end{figure}%%
%----------------------------------------------------------------------

%%%%%%%%%%%%%%%%%%%%%%%%%%%%%%%%%%%%%%%%%%%%%%%%%%%%%%%%%%%%%%%55
{\textcolor{blue}{\section{Mengenlehre}}}\label{Mengenlehre}\index{Mengenlehre}
%%%%%%%%%%%%%%%%%%%%%%%%%%%%%%%%%%%%%%%%%%%%%%%%%%%%%%%%%%%%%%%%%%%%%%%%%%%%%%%%
Wir wollen hier
noch einige Anmerkungen zum Minimum der potentiellen Energie  machen.

Die tiefste Lage ist die stabilste Lage. Hinter vielen Prozessen in der Physik steckt ein
Minimumprinzip. Auch die Gleichgewichtslage eines Tragwerks gehorcht einem
Minimumprinzip. Die Seilkurve ist die Funktion $w$ in $\mathcal{V}$, die die potentielle Energie des Seils zum Minimum macht.

Ebenso in der Balkenstatik:  Die Biegelinie $w$ eines Durchlauftr\"{a}gers macht die
potentielle Energie des Tr\"{a}gers
\begin{align}
\Pi(w) = \frac{1}{2} \int_0^{\,l} \frac{M^2}{EI} \,dx - \int_0^{\,l} p\, w \, dx \qquad
\rightarrow \quad \mbox{Minimum}
\end{align}
zum Minimum auf $\mathcal{V}$, das sind alle Funktionen $w$, die die
geometrischen Lagerbedingungen des Tr\"{a}gers erf\"{u}llen, die also in allen Lagern
Nullstellen, $w = 0$, haben. Jedes solche $w$ ist zur Konkurrenz zugelassen, liegt in $\mathcal{V}$.
%----------------------------------------------------------------------------------------------------------
\begin{figure}[tbp] \centering
\if \bild 2 \sidecaption \fi
\includegraphics[width=.8\textwidth]{\Fpath/U209X}
\caption{Die potentielle Energie $\Pi(w_h)$ der FE-L\"{o}sung liegt auf der Zahlengerade
immer rechts von der exakten potentiellen Energie $\Pi(w)$} \label{Energie}
\end{figure}%
%----------------------------------------------------------------------------------------------------------

Die Biegelinie $w$ des Durchlauftr\"{a}gers ist der Sieger, weil sie den kleinsten Wert f\"{u}r die
potentielle Energie liefert. Gem\"{a}{\ss} der ersten Greenschen Identit\"{a}t gilt in der Gleichgewichtslage $w$
\begin{align}
\int_0^{\,l} \frac{M^2}{EI}\,dx = \int_0^{\,l} p\,w\,dx \,,
\end{align}
und somit ist das Minimum der potentiellen Energie {\em negativ\/}
\begin{align}
\Pi(w) = \frac{1}{2}\,\int_0^{\,l} \frac{M^2}{EI}\,dx - \int_0^{\,l} p\,w\,dx = -
\frac{1}{2} \int_0^{\,l} p \,w \, dx \,,
\end{align}
denn das Integral ohne das Minus davor ist positiv. Es ist die Arbeit der
Streckenlast $p$ auf den eigenen Wegen, und Eigenarbeit ist immer positiv.

Sind keine Streckenlasten $p$ vorhanden, sondern senkt sich ein Lager (ein LF $\delta$), dann fehlt das
Lastintegral in der potentiellen Energie
\begin{align}
 \Pi(w) =
\frac{1}{2} \int_0^{\,l} \frac{M^2}{EI}\, dx > 0 \,,
\end{align}
(Lagersenkungen $\delta $ kommen in der Energie nie vor, sondern nur in der
Definition des Raums $\mathcal{V}$), und dann ist das Minimum gr\"{o}{\ss}er als null.

Zusammenfassend k\"{o}nnen wir also f\"{u}r die beiden m\"{o}glichen Typen von Lastf\"{a}llen (Lasten $p$ oder Lagersenkung $\delta$) notieren\\

\begin{itemize}

\item {LF $p$ $\qquad \Pi < 0$} \index{LF $p$}

\item {LF $\delta$ $\qquad \Pi > 0$}\,. \index{LF $\delta$}

\end{itemize}

Dieser Beobachtung \"{u}berlagert sich nun noch eine zweite Beobachtung, dass n\"{a}mlich
die Verzerrungsenergie der FE-L\"{o}sung in einem LF $p$ immer kleiner ist, als die
Verzerrungsenergie der exakten L\"{o}sung. Bei einem Balken haben wir also
\begin{align}
\int_0^{\,l} \frac{M_h^2}{EI}\,dx \leq \int_0^{\,l} \frac{M^2}{EI}\,dx \qquad \mbox{( in
einem LF $p$)}\,,
\end{align}
w\"{a}hrend in einem LF $\delta$ (Lagersenkung) die Rollen vertauscht sind
\begin{align}
\int_0^{\,l} \frac{M^2}{EI}\,dx \leq \int_0^{\,l} \frac{M_h^2}{EI}\,dx \qquad \mbox{(in
einem LF $\delta$)}\,.
\end{align}
Eine kleinere Verzerrungsenergie in einem LF $p$ bedeutet aber, dass das Tragwerk 'steifer'
ist, sich nicht so verformt, wie ein Tragwerk, bei dem die Verzerrungsenergie gr\"{o}{\ss}er
ist. Daher r\"{u}hrt die Bemerkung, dass bei einer FE-Berechnung die Steifigkeit eines
Tragwerks \"{u}bersch\"{a}tzt wird.

Aber auch die potentielle Energie der FE-L\"{o}sung ist kleiner, als die potentielle Energie
der wahren L\"{o}sung
\begin{align}
 \Pi(w) < \Pi(w_{h}) \qquad \mbox{wegen}
\,\, \mathcal{V}_{h} \subset \mathcal{V} \,.
\end{align}
Denken wir uns die $x$-Achse als Skala, auf der wir die potentielle Energie -- sozusagen
'die Temperatur des Tragwerks' -- markieren, so liegt der Punkt $\Pi(w_h)$,
bildlich gesprochen, {\em immer rechts} von dem Punkt $\Pi(w)$, s. Abb. \ref{Energie}.

In einem LF $p$ bedeutet diese Reihenfolge: Die potentielle Energie der FE-L\"{o}sung sinkt
nicht so tief ab wie die potentielle Energie der wahren L\"{o}sung, das Tragwerk 'h\"{a}ngt
nicht so stark durch' -- die Verformungen bleiben kleiner.

In einem LF $\delta$ bedeutet dies: Die FE-L\"{o}sung 'verbraucht mehr Energie'
als die wahre L\"{o}sung, weil das Tragwerk steifer ist, der Aufwand an Verzerrungsenergie
ist gr\"{o}{\ss}er. Man muss mehr innere Arbeit hineinstecken, $\Pi(w_{h})$ liegt rechts von
$\Pi(w)$.
\begin{itemize}
\item LF $p$ \qquad $\Pi(w_h)$ liegt n\"{a}her an null als $\Pi(w)$
\item LF $\delta $ \qquad $\Pi(w_h)$ liegt weiter weg von null als $\Pi(w)$\,.
\end{itemize}
Aus den obigen Darlegungen folgt im \"{u}brigen, dass das
Minimumprinzip eigentlich ein Maximumprinzip ist -- zumindest was die Lastf\"{a}lle $p$
anbelangt. Die Rede vom Minimum ist {\em Camouflage\/}, ist Tarnung. Wir h\"{o}ren diese
Rede gern, weil viele Prozesse in der Natur einem Minimumprinzip ({\em least action\/})
\index{least action} unterliegen. In Wirklichkeit aber stellt sich die Balkenbiegelinie
$w$ so ein, dass {\em betragsm\"{a}{\ss}ig\/} die potentielle Energie maximal wird, dass
$|\Pi(w)|$ also m\"{o}glichst weit weg von null liegt. Man kann sich das so vorstellen, dass
die Streckenlast $p$ auf einem Riegel m\"{o}glichst weit nach unten durchsacken will, um
m\"{o}glichst viel Lageenergie in potentielle Energie\index{Gleichgewichtspunkt}
\begin{align}
\Pi(w) = - \frac{1}{2}\,\int_0^{\,l} p\,w\,dx\,, \qquad w = \mbox{Gleichgewichtslage}
\end{align}
umzuwandeln.

%----------------------------------------------------------------------------------------------------------
\begin{figure}[tbp] \centering
\if \bild 2 \sidecaption \fi
\centering
\includegraphics[width=0.85\textwidth]{\Fpath/U208}
\caption{Dort, wo $A_i = A_a$ ist, liegt der Gleichgewichtspunkt $u$ der Feder. Weil die
innere Energie $A_i$ quadratisch mit $u$ w\"{a}chst, die \"{a}u{\ss}ere Arbeit $A_a$ aber nur
linear, holt $A_i$ immer $A_a$ ein, gibt es immer eine Gleichgewichtslage}
\label{U208}
\end{figure}%
%----------------------------------------------------------------------------------------------------------

Die Bewegung stoppt am Gleichgewichtspunkt. Das ist der Punkt, an dem die \"{a}u{\ss}ere Arbeit
$A_a$ gleich der inneren Energie $A_i$ ist,
\begin{align}
A_a = \frac{1}{2}\,\int_0^{\,l} p\,w\,dx = \frac{1}{2}\,\int_0^{\,l} \frac{M^2}{EI}\,dx =
A_i\,,\qquad \mbox{Gleichgewichtspunkt}\,,
\end{align}
denn je mehr die Last den Balken nach unten dr\"{u}ckt ($A_a$ w\"{a}chst), um so mehr Widerstand
w\"{a}chst der Last entgegen: Der Balken verkr\"{u}mmt sich, die Momente nehmen zu, und somit
w\"{a}chst die innere Energie $A_i$, s. Abb. \ref{U208}.

Nur bei Lastf\"{a}llen $\delta$ kommt das Minimum in seiner urspr\"{u}nglichen Bedeutung zu
Wort. Dann strebt das Tragwerk danach, die Zwangsverformungen auf m\"{o}glichst kleinem
Energieniveau -- also $\Pi(w)$ m\"{o}glichst nahe an null -- \"{u}ber
sich ergehen zu lassen.

Ob nun aber positiv oder negativ, richtig ist auf jeden Fall, dass die potentielle
Energie eine nach {\em oben\/} offene Parabel ist, und dies bedeutet, dass man Energie
zuf\"{u}hren muss, will man das Tragwerk aus der Gleichgewichtslage auslenken. Es
herrscht also sowohl in einem LF $p$ wie einem LF $\delta $ ein stabiles Gleichgewicht.

Abb. \ref{U208} illustriert sehr sch\"{o}n die sogenannte \index{Elliptizit\"{a}t}{\em
Elliptizit\"{a}t\/} der Feder, denn weil die Steifigkeit $k$ der Feder gr\"{o}{\ss}er als null ist,
gilt
\begin{align}
a(u,u) = k\,u^2 \geq k\,|u|^2\,,
\end{align}
und, weil, wie wir annehmen d\"{u}rfen, $P$ nicht unendlich gro{\ss} ist, gibt es immer eine L\"{o}sung $u = P/k$, holt die
quadratische Parabel $1/2\,k\,u^2$ die Gerade $1/2\,P\,u$ immer ein, gibt
es immer einen Gleichgewichtspunkt zwischen innerer Energie und \"{a}u{\ss}erer Arbeit.\\


Und auch glatte L\"{o}sungen k\"{o}nnen falsch sein. Man denke an einen Kragtr\"{a}ger, dessen Biegelinie unter einer falschen Steigung aus der Wand l\"{a}uft, weil die Drehsteifigkeit der Wand falsch angesetzt wurde.


%------------------------------------------------------------------
\begin{figure}[tbp]
\centering
\if \bild 2 \sidecaption \fi
\includegraphics[width=0.8\textwidth]{\Fpath/U487}
\caption{Links drillsteife Lagerung, rechts drillweiche Lagerung} \label{VergleichM}
\end{figure}%%
%------------------------------------------------------------------

\begin{align}
\colorbox{lightgray!25}{$\displaystyle{\Pi(w) = \frac{1}{2} \int_{0}^{l} \frac{M^2}{EI}\,dx - \int_{0}^{l}p\,w\,dx}$}
\end{align}
zum Minimum macht. Weil das Minimum aber negativ ist, es ist gerade die negativ genommene Eigenarbeit
\begin{align}

%-----------------------------------------------------------------
\begin{figure}[tbp]
\centering
\if \bild 2 \sidecaption \fi
\includegraphics[width=0.71\textwidth]{\Fpath/U477}
\caption{Das FE-Programm ersetzt die Streckenlast $p$ durch ein arbeits\"{a}quivalentes Moment, $(p,\Np_4) = p\,\ell^2/12 \cdot 1$ }
\label{U477}
\end{figure}%%
%-----------------------------------------------------------------

%----------------------------------------------------------------------------------------------------------
\begin{figure}[tbp] \centering
\if \bild 2 \sidecaption \fi
\includegraphics[width=.6\textwidth]{\Fpath/PATCH11}
\caption{Die \"{U}berst\"{a}nde sind f\"{u}r das fehlende Gleichgewicht \"{u}ber Kreuz verantwortlich.
Die gestrichelte Linie soll andeuten, wo man ungef\"{a}hr schneiden m\"{u}sste, damit
Gleichgewicht herrscht} \label{Patch1}
\end{figure}%
%----------------------------------------------------------------------------------------------------------

Wollte man all das in Worten beschreiben, so k\"{o}nnte man sagen: {\em Ein FE-Programm
misst falsch.\/} Wenn ihm ein Pr\"{u}fingenieur den Auftrag gibt, die FE-Lasten $p_h$
in dem {\em patch\/} $\Omega_p$ so einzustellen, dass sie gleich gro{\ss} sind wie im LF
$p$, so 'mogelt' das Programm. Es legt nicht nur die FE-Lasten in dem {\em
patch} selbst auf die Waage, sondern auch noch die in den angrenzenden Elementen
(anteilig in etwa zur H\"{a}lfte). Es erweitert sozusagen heimlich den Messbereich
$\Omega_p$.

Ins Positive gewendet bedeutet dies, dass es so schlimm mit dem fehlenden lokalen
Gleichgewicht auch nicht bestellt sein kann, denn wir haben ja Gleichheit zwischen den
Lasten $p$ in $\Omega_p$ und den Lasten $p_h$ in $\Omega_p + 1$ {\em Elementreihe\/}, wobei
die Lasten im angrenzenden Streifen, wie gesagt, nur mit der H\"{a}lfte in die Bilanz
eingehen. Zwischen dem Rand des {\em patchs\/} $\Omega_p$ und dem Rand von $\Omega_p +
1$ {\em Elementreihe\/} verl\"{a}uft also eine imagin\"{a}re Linie, s. Abb. \ref{Patch1}. W\"{u}rden wir die Platte l\"{a}ngs
dieser gedachten Linie aufschneiden, dann h\"{a}tten wir das Gleichgewicht zwischen den
dortigen Schnittkr\"{a}ften und den Kr\"{a}ften im {\em patch\/} $\Omega_p$.


Wir wollen hier noch einmal wiederholen, welchen Zwecken die globale Steifigkeitsmatrix dient. Multipliziert man die Steifigkeitsmatrix mit einem Knotenvektor, $\vek K\,\vek u = \vek f_h$,  so erh\"{a}lt man die \"{a}quivalenten Knotenkr\"{a}fte, die zu dem Vektor $\vek u $ geh\"{o}ren. Das wird anschaulicher, wenn wir als Knotenvektor die Einheitsvektoren $\vek e_i$  w\"{a}hlen. Die Eintr\"{a}ge in der Spalte $i$
\begin{align}
\{k_{1i}, k_{2i}, k_{3i}, \ldots, k_{ni}\}^T
\end{align}
sind die Kr\"{a}fte, $k_{ii} $, die f\"{u}r die Bewegung $u_i = 1 $ n\"{o}tig sind bzw. die Kr\"{a}fte $k_{ji}$, (oberhalb und unterhalb davon), die die Bewegung an den Nachbarknoten stoppen.

Insbesondere folgt daraus, dass die Elemente $k_{ij}$ der Steifigkeitsmatrix eines Balkens sowohl als innere wie als \"{a}u{\ss}ere Arbeit gez\"{a}hlt werden k\"{o}nnen
\begin{align}
\delta A(p_i,\textcolor{red}{\Np_j}) = [V_i\,\textcolor{red}{\Np_j} -
M_i\,\textcolor{red}{\Np_j'}]_{\,0}^{\,l} = \int_0^{\,l} \frac{M_i\,\textcolor{red}{M_j}}{EI}\,dx = k_{ij} = \delta A_i(\Np_i,\textcolor{red}{\Np_j})\,.
\end{align}

%%%%%%%%%%%%%%%%%%%%%%%%%%%%%%%%%%%%%%%%%%%%%%%%%%%%%%%%%%%%%%%%%%%%%%%%%%%%%%%%
{\textcolor{sectionTitleBlue}{\section{Finite Elemente am Balken}}}
%%%%%%%%%%%%%%%%%%%%%%%%%%%%%%%%%%%%%%%%%%%%%%%%%%%%%%%%%%%%%%%%%%%%%%%%%%%%%%%%
Wir wollen hier noch kurz zeigen, wie man einen Durchlauftr\"{a}ger mit finiten Elementen berechnet.

Das erste, was man macht ist, dass man die Belastung in die Knoten reduziert. Man zieht also die Belastung aus dem Feld in die Knoten zur\"{u}ck. Statisch geht das so, dass man die Festhaltekr\"{a}fte berechnet und diese dann mit umgedrehten Vorzeichen als Belastung auf die Knoten wirken l\"{a}sst.

Weil die Belastung in den Knoten angreift, ist die Biegelinie in jedem Feld eine homogene L\"{o}sung der Balkengleichung \begin{align}
w(x) = w_1\,\Np_1(x) + w_2\,\Np_2(x)
\end{align}
wobei die $\Np_i(x)$ die Einheitsverformungen der Balkenenden sind.

Der Raum $\mathcal{V}_h $ dient also zum einen als Ansatzraum, aus ihm kommt die FE-L\"{o}sung, zum andern dienen die einzelnen $\Np_i$ als Testfunktionen.

Die eigentliche Statik steckt in den {\em shape forces\/} $p_i$, die sich durch Differenzieren direkt aus den $\Np_i $ berechnen lassen. Sie bestimmen die Eintr\"{a}ge $k_{ij} = \delta A_i(\Np_i,\Np_j) = \delta A_a(p_i,\Np_j)$ in den Steifigkeitsmatrizen.

Die Steifigkeitsmatrizen m\"{u}ssen also stimmen, denn sonst operiert man mit falschen Werten $f_{hi}$ in der Gegen\"{u}berstellung $f_{hi} = f_i$, und dann liefert das System $\vek K\,\vek u = \vek f$ falsche Verformungen $\vek u$.

Das ist \"{u}brigens auch genau die Technik, die in der \glq Originalarbeit\grq{} \cite{Turner} angewandt wurde. {\em Turner et alteri\/} haben die Steifigkeitsmatrizen auf dem Weg $f_{hi} = f_i$ gefunden, also \"{u}ber $\delta A_a(p_h,\Np_i) = \delta A_a(p,\Np_i)$ und nicht \"{u}ber das Prinzip vom Minimum der potentiellen Energie, s. \cite{HaJa2}.
Die Steifigkeitsmatrix, die sie hergeleitet haben, multipliziert mit dem Vektor $\vek u$ ist der Vektor $\vek f_h = \vek K\,\vek u$, den sie f\"{u}r die \"{A}quivalenz $\vek f_h = \vek f $ brauchten. Dieser Vektor hat sie interessiert, nicht die Steifigkeitsmatrix \glq an sich\grq{}.

\begin{align}
G_h(\vek y,\vek x) &= \int_{\Gamma} (g(\vek y,\vek x)\,\boxed{\frac{\partial G_h}{\partial n}(\vek y)} - \frac{\partial g(\vek y,\vek x)}{\partial n}\,\boxed{G_h(\vek y))}\,ds_{\vek y}\nn \\
&+ \int_{\Omega} g(\vek y,\vek x)\,\boxed{(- \Delta G_h(\vek y))} \,d\Omega_{\vek y}\,,
\end{align}

\begin{remark}
 Bei linearen Elementen besteht das $\delta_h$ aus Linienkr\"{a}ften $l_k$ l\"{a}ngs den Kanten des Netzes, die gleich den Spr\"{u}ngen der Normalableitung von $G_h$ zwischen Elementen sind und dann h\"{a}tte die obige Gleichung die Gestalt
\begin{align}
G_h(\vek y,\vek x) &=  \int_{\Gamma} g(\vek \xi,\vek y)\,\frac{\partial G_h}{\partial n}(\vek \xi, \vek x)\,ds_{\vek \xi} +\sum_k \int_{0}^{l_k} l_k(\vek y)\,g(\vek y, \vek x)\,ds_{\vek y}\,.
\end{align}
Gibt es Elementlasten und Spr\"{u}nge, dann ist es eine Mischung aus beiden.
\end{remark}

Die klassische Stabtheorie besch\"{a}ftigt sich mit dem Bernoulli-Balken. Querschnitte bleiben eben und behalten ihre Gestalt. Schubspannungen werden nachtr\"{a}glich aus Gleichgewichtsbetrachtungen ermittelt, Schubverformungen werden mit einer Schubverformungsfl\"{a}che und einem globalen Gleitwinkel hinzugef\"{u}gt. Die klassischen Freiheitsgrade sind die Verschiebungen und Verdrehungen der Stabknoten. W\"{o}lbkrafttorsion kann mit einem weiteren Freiheitsgrad der Verwindung ber\"{u}cksichtigt werden, jedoch ergeben sich Probleme in Rahmenecken, die man mit hohem Aufwand oder n\"{a}herungsweise ber\"{u}cksichtigen muss. F\"{u}r die Berechnung der Elementmatrizen sind drei Methoden bekannt:
\begin{itemize}
  \item Analytische Formulierungen prismatischer St\"{a}be
  \item Numerisch integrierte Energieterme eines Verformungsansatzes
  \item Numerische Integration der Differentialgleichung (\"{U}bertragungsverfahren)
\end{itemize}


%%%%%%%%%%%%%%%%%%%%%%%%%%%%%%%%%%%%%%%%%%%%%%%%%%%%%%%%%%%%%%%%%%%%%%%%
{\textcolor{sectionTitleBlue}{\section{Warnung}}}\label{Warnung}
%%%%%%%%%%%%%%%%%%%%%%%%%%%%%%%%%%%%%%%%%%%%%%%%%%%%%%%%%%%%%%%%%%%%%%%%
Wir wollen dieses einleitende Kapitel mit einer Warnung ausklingen lassen. Ein Pr\"{u}fingenieur, der unseren Ausf\"{u}hrungen bis hierher gefolgt ist, k\"{o}nnte nun leicht versucht sein, den Aufstellern in Zukunft zur Pflicht zu machen den FE-Lastfall $p_{h} $ zu dokumentieren, damit er sich schnell ein Bild davon machen kann, wie \glq gut\grq\ die FE-L\"{o}sung ist, die ihm da zur Pr\"{u}fung vorgelegt wird. Theoretisch ist ein solches Ansinnen nicht von der Hand zu weisen, man muss jedoch vor \"{U}berinterpretationen warnen.

Zum einen ist es so, dass bei realen Tragwerken der Lastfall $p_{h} $ scheinbar weit, weit von dem Originallastfall entfernt liegt. Wenn man sich einmal die M\"{u}he macht und den LF $p_{h} $ f\"{u}r reale Strukturen ausrechnet, so erschrickt man, wie wenig der FE-Lastfall mit dem Original zu tun hat. Das ist auch der Grund, warum die kommerziellen Programme diese Lastf\"{a}lle nicht darstellen, denn der nicht entsprechend vorgebildete Aufsteller w\"{u}rde am Programm (ver)zweifeln.

Das  eigentliche Geheimnis der finiten Elemente ist, dass die Ergebnisse trotzdem etwas taugen. Und wenn wir als Tragwerksplaner mehr Zutrauen zu den finiten Elementen gewinnen wollen, dann m\"{u}ssen wir uns mit diesem Punkt intensiver auseinandersetzen. Dies ist ein Problem der {\em Statik} und erst in zweiter Linie ein Problem der numerischen Mathematik.

Was hei{\ss}t {\em nah} und {\em fern} in der Statik? Welche Unsch\"{a}rfe k\"{o}nnen wir uns in der Statik erlauben und welche nicht?\footnote{Man lese den Aufsatz von B\"{u}rg und Schneider \cite{Buerg} \"{u}ber die Bemessung einer einfachen Garagendecke durch 32 Ingenieure!}

Was wir -- etwa bei einer Plattenberechnung -- sehen und vergleichen k\"{o}nnen sind die Lasten, das $p$ und das $p_{h} $, also -- wir vereinfachen etwas -- die vierten Ableitungen der Biegefl\"{a}che. Die Schnittmomente aber sind die zweiten Ableitungen, also das zweifach unbestimmte Integral der Belastung
\begin{align}
m = \int \int  p \,d\Omega \,d\Omega \,,\qquad m_{h} = \int \int p_{h}\,d\Omega\,
d\Omega\,.
\end{align}
Bei der Integration werden die Oszillationen in den Lasten gegl\"{a}ttet, und so kommt es, dass die Schnittmomente der FE-L\"{o}sung relativ gut den exakten Momenten folgen. Das ist der mathematische Hintergrund. Integration ist ein Gl\"{a}ttungsprozess, und kein Pr\"{u}fingenieur kann aus der Differenz der Lasten Aussagen \"{u}ber die Differenz der Schnittmomente machen.

Wenn im Dachgeschoss, in der vierten Etage, die W\"{a}nde 20 cm aus der Flucht stehen, wer will dann sagen, wie gro{\ss} die Abweichung im zweiten Geschoss ist? Den Abstand im Dachgeschoss k\"{o}nnen wir messen. Die Abweichung im zweiten Geschoss nicht. \"{U}ber sie k\"{o}nnen wir nur spekulieren. Das ist das Problem.

Als Ingenieure sind wir versucht die Fl\"{a}chen $p$ und $p_h $ direkt zu vergleichen und daraus auf die G\"{u}te der FE-L\"{o}sung zu schlie{\ss}en.

Wie die adaptiven Methoden erfolgreich demonstrieren, ist es durchaus sinnvoll, sich bei der Verfeinerung von den Fehlerkr\"{a}ften $p - p_h$ leiten zu lassen. Auf der anderen Seite, so k\"{o}nnte man einwenden, strebt ein FE-Programm ja nicht danach, den Abstand $p -p_h$ m\"{o}glichst klein zu machen, sondern den Abstand in den Schnittgr\"{o}{\ss}en. Schauen wir also auf die falschen Daten?

Nein, die Daten sind richtig, nur: {\em Das FE-Programm schaut genauer hin\/}. Als Ingenieure sprechen wir von einem gro{\ss}en Abstand, wenn -- etwa beim Balken -- die beiden Streckenlasten $p$ und $p_h$ weit auseinander liegen. Nicht so das FE-Programm. Das FE-Programm h\"{a}lt die Differenz $p - p_h$ gegen die Einheitsverformungen $\Np_i$ der Knoten, und nur wenn das Integral
\begin{align}
\int_0^{\,l} (p - p_h) \,\Np_i\,dx\,,
\end{align}
also die \"{U}berlagerung der Fehlerkr\"{a}fte mit {\em allen\/} $\Np_i$ null ist, dann ist der Abstand $p - p_h$ \glq richtig austariert\grq! Das ist der entscheidende Unterschied, den wir auf dem Bildschirm nicht sehen k\"{o}nnen.

Unser Auge misst die Fl\"{a}che zwischen den beiden Kurven, es h\"{a}lt sozusagen die Differenz
$|p-p_h|$ gegen die Eins
\begin{align}
\int_0^{\,l} |p - p_h| \cdot 1\,dx\,.
\end{align}
Es misst die Resultierende $R$ der absolut genommenen Fehlerkr\"{a}fte, oder, was wegen $R \times 1 = R$, dasselbe ist: Es sch\"{a}tzt die Arbeit ab, die die Betr\"{a}ge der Fehlerkr\"{a}fte auf der virtuellen Verr\"{u}ckung $\delta w = 1$ leisten.

Das FE-Programm h\"{a}lt dagegen die Fehlerkr\"{a}fte gegen {\em alle\/} Einheitsverformungen der Knoten und versucht all diese virtuellen \"{a}u{\ss}eren Arbeiten zu null zu machen. Wegen $\delta A_a = \delta A_i$ ist dies gleichbedeutend damit, dass das FE-Programm den Fehler in den Momenten so einstellt, dass er orthogonal zu den Momenten $M_i$ der Einheitsverformungen $\Np_i$ ist,
\begin{align}
\delta A_a = \int_0^{\,l} (p - p_h) \,\Np_i\,dx = \int_0^{\,l} \frac{(M -
M_h)\,M_i}{EI}\,dx = \delta A_i\,.
\end{align}
So gelingt der Wechsel von Au{\ss}en nach Innen, kann man au{\ss}en am Tragwerk messen, was im
Innern passiert.

Beim Blick auf die Fehlerkr\"{a}fte $p - p_h$ muss man also mitdenken, und in diesem Sinne war die \"{U}berschrift dieses Abschnitts gemeint. Es w\"{a}re sicherlich zu naiv, die Fehlerkr\"{a}fte {\em prima facie\/} zu nehmen, sich darauf zu versteifen, dass das Lastbild $p_h$ der FE-L\"{o}sung m\"{o}glichst \glq gut\grq\ aussieht, denn das Auge operiert nur mit einer virtuellen Verr\"{u}ckung, $\Np = 1$, ein FE-Programm dagegen mit 10, 100 oder 1000 Einheitsverformungen $\Np_i$, und man muss einem FE-Programm zugute halten, dass es versucht den Ersatzlastfall $p_h$ so optimal einzustellen, dass der Fehler in der Energie -- also das Fehlerquadrat der Schnittkr\"{a}fte -- m\"{o}glichst klein wird.


{\textcolor{sectionTitleBlue}{\subsection{Koppelkr\"{a}fte}}}\index{Koppelkr\"{a}fte}
Was wir \"{u}ber die Lagerkr\"{a}fte gesagt haben, gilt sinngem\"{a}{\ss} auch f\"{u}r die Koppelkr\"{a}fte zwischen unterschiedlichen Bauteilen. Im Ausdruck passt scheinbar alles zusammen, weil nicht die echten Koppelkr\"{a}fte ausgedruckt werden, sondern die aus den \"{a}quivalenten Knotenkr\"{a}ften $f_i$ berechneten Koppelkr\"{a}fte. Diese sind dann trivialerweise auf beiden Schnittufern in jedem Punkt gleich gro{\ss}, w\"{a}hrend die Koppelkr\"{a}fte in Wirklichkeit nur arbeits\"{a}quivalent sind.

\begin{alignat}{4}
J(w) &= w(x) \cdot 1 \qquad &&1 = \text{Kraft} \qquad &&J(w) = V(x) \cdot 1 \qquad &&1 = \text{Spreizung} \nn  \\
J(w) &= M(x) \cdot 1 \qquad &&1 = \tan 45^\circ \qquad &&J(u) = \sigma_{xx} \cdot 1 \qquad &&1 = \text{Versetzung} \nn
\end{alignat}

 Wenn nun der Benutzer die Elemente immer kleiner macht, $h \to 0$, dann ist das f\"{u}r das FE-Programm ein Signal, dass es die Spannungen in der N\"{a}he dieses Knotens unendlich gro{\ss} machen muss, weil sonst die Bilanz
\begin{align}\label{Eq16}
\int_{\Omega} \sigma_{ij}^h\,\delta \varepsilon_{ij} \,d\Omega = f_i^h
\end{align}
nicht einzuhalten ist\footnote{Das ist der Wackeltest $f_{hi} = f_i$ in Richtung von $f_i$ bei der Auslenkung $\vek \Np_i$ des Knotens, alle anderen Knoten sind fest. Das $f_{hi} = \delta A_a(\vek u_h,\vek \Np_i) = \delta A_i(\vek u_h,\vek \Np_i)$ wird in (\ref{Eq16}) \glq innen\grq\ gemessen, $\delta A_i = (\sigma_{ij},\delta \varepsilon_{ij})$}.

Die Verzerrungen $\delta \varepsilon_{ij}$ kommen aus der Einheitsverschiebung des Knotens in Richtung der Kraft und weil sie nur auf den Elementen nicht null sind, auf denen der Knoten liegt, muss immer weniger Gebiet, $h \to 0$, immer gr\"{o}{\ss}ere Spannungen $\sigma_{ij}^h$ und Verzerrungen $\delta \varepsilon_{ij}$ produzieren.  Auch so kommen singul\"{a}re Spannungen in die finiten Elemente hinein. Sie sind in dieser Situation eine \glq Schutzma{\ss}nahme\grq, um bei immer kleiner werdenden Elementen am Ziel $f_i^h = f_i$ festhalten zu k\"{o}nnen.

%----------------------------------------------------------------------
\begin{figure}[h]
\if \bild 2 \sidecaption \fi
\includegraphics[width=0.8\textwidth]{\Fpath/MUELLER}
\caption{{\small Fehlerindikatoren bei einer Wandscheibe}}\label{Mueller}
\end{figure}
%----------------------------------------------------------------------


\begin{align}
 \frac{EI}{l^3} \left[
\begin{array}{r r r r}
 12 & -6l & -12 &-6l \\
 -6l & 4l^2 & 6l &2l^2 \\
 -12 & 6l & 12 & 6l \\
 -6l &2l^2 &6l &4l^2
 \end{array}
  \right]\,\left [\barr{c} w_1 \\ w_2 \\ w_3 \\ w_4 \earr \right ] = \left [\barr{c}  f_1 \\ f_2 \\ f_3 \\ f_4 \earr \right ]\,.
\end{align}

Das Thema h\"{a}lt auch noch eine kleine \"{U}berraschung bereit, denn es hilft zu verstehen, warum die Spannungen in den Gausspunkten genauer sind als in den \"{u}brigen Punkten.

Zahlenm\"{a}{\ss}ig erh\"{a}lt man nat\"{u}rlich auf beiden Wegen dieselbe Matrix.

Dass dabei {\em zahlenm\"{a}{\ss}ig\/} dasselbe herauskommt, kann man wie folgt erkl\"{a}ren. Es bezeichne $f_{ik} $ die Balkenendkr\"{a}fte in Richtung des Freiheitsgrades $u_k $, die zu der Auslenkung $\Np_i $ geh\"{o}ren. Wenn man nun dem Balken in der ausgelenkten Lage und gehalten von diesen Kr\"{a}ften, LF $p_i$, eine Einheitsverformung $\Np_j $ erteilt, dann leistet nur die Kraft $f_{ij} $ eine Arbeit, die gerade $f_{ij} \cdot 1$ ist, also zahlenm\"{a}{\ss}ig gleich der treibenden/haltenden Kraft, die den Knoten in Richtung von $u_i $ auslenkt bzw. seine Bewegung in diese Richtung sperrt.

Nun ist \glq {\em innen = au{\ss}en\/}\grq, ist
\begin{align}
\delta A_i(\Np_i,\Np_j) = k_{ij} = f_{ij} = \delta A_a(p_i,\Np_j)
\end{align}
und so kommt es, dass in der Steifigkeitsmatrix die Balkenendkr\"{a}fte der Einheitsverformungen stehen, allerdings als Arbeiten $f_{ij} \cdot 1$. Dividiert man jetzt noch alle $f_{ij} $ durch die Dimension der zugeh\"{o}rigen Weggr\"{o}{\ss}e, also durch [m] bzw. [$\,/\,$] ({\em tangens\/}), dann hat man genau die Steifigkeitsmatrix des statischen Weges vor sich.


%%%%%%%%%%%%%%%%%%%%%%%%%%%%%%%%%%%%%%%%%%%%%%%%%%%%%%%%%%%%%%%%%%%%%%%%%%%%%%%%%%%%%%%%%%%%%%%%%%%
\textcolor{chapterTitleBlue}{\subsection{Die Dimension der $f_i$}}\label{Dimensionsbetrachtung}
Gelegentlich wird \"{u}ber die Dimension der $f_i$ in dem Vektor $\vek K\,\vek w = \vek f$ beim Balken diskutiert. Sind es Kr\"{a}fte oder Arbeiten? Die korrekte Antwort lautet -- je nachdem. Wenn man die Eintr\"{a}ge $k_{ij}$ der Steifigkeitsmatrix mit der Formel
\begin{align}\label{Eq1280}
k_{ij } = a(\Np_i,\Np_j) = EI\,\int_0^{\,l} \Np_i''\,\Np_j''\,dx = \text{kNm$^2$ } \frac{1}{\text{m}}\frac{1}{\text{m}}\,\text{m} = \text{kNm}
\end{align}
berechnet, wie man das bei finiten Elementen tut, dann haben die $k_{ij}$ die Dimension einer Arbeit, die $u_i$ sind dimensionslos und die $f_i$ sind Arbeiten.

Zur Erl\"{a}uterung von (\ref{Eq1280}): wenn $\Np_1(x)$ die Dimension Meter hat, dann haben die Ableitungen die Dimension
\begin{align}
\Np_1(x) \,\,\text{[m]} \qquad \Np_1'\,\,[\,] \qquad\Np_1'' = [\frac{1}{\text{m}}] \qquad\Np_1''' = [\frac{1}{\text{m$^2$}}]\,,
\end{align}
weil bei jeder Ableitung $d/dx$ durch m dividiert wird.

Man kann die Matrix $\vek K$ aber auch auf statischem Wege herleiten, indem man die  Balkenendkr\"{a}fte und -momente der Einheitsverformungen $\Np_i(x)$, s. (\ref{Eq219X}) berechnet und diese Werte in die jeweilige Spalte $i$ eintr\"{a}gt. Wenn man so vorgeht, dann sind die $k_{ij}$ der Dimension nach Kr\"{a}fte  bzw. Momente pro Auslenkung/Verdrehung $w_i = 1$, weil bei der statischen Interpretation das Ergebnis
ja die Kr\"{a}fte und Momente sind, die zu der Auslenkung $\vek w$ geh\"{o}ren, $\vek K\,\vek w = \vek f$.

Dividiert man nicht durch die Auslenkung/Verdrehung, dann hat das Ergebnis $\vek K\,\vek w = \vek f$ dieselbe Dimension, wie beim energetischen Zugang.

%%%%%%%%%%%%%%%%%%%%%%%%%%%%%%%%%%%%%%%%%%%%%%%%%%%%%%%%%%%%%%%%%%%%%%%%%%%%%%%%%%%%%%%%%%%%%%%%%%%
\textcolor{sectionTitleBlue}{\section{Vorverformungen}}\index{Vorverformungen}\label{Korrektur42}
 \glq {\em Bei Ansatz von Vorverformungen bedeutet Theorie II. Ordnung die Formulierung des Gleichgewichts am {\em gesamtverformten\/} System, wobei Gesamtverformung = Vorverformung + Lastverformung ist\grq\/}, \cite{Rubin} S. 77.

 Wir schreiben das als
\begin{align}
w(x) = s(x) + w_{L}(x) \qquad s\,\text{wie \glq Schlangenlinie\grq}\,.
\end{align}
Die Vorverformung wird durch ihre Interpolierende $s_I$ ersetzt
\begin{align}\label{Eq86}
s_I(x) = \sum_j s_j\,\Np_j(x)\,,
\end{align}
wobei die $s_j$ die Knotenwerte von $s(x)$ sind (Durchbiegungen und Verdre\-hungen -- Hermite Interpolation!).
Man erh\"{a}lt den Vektor der \"{a}quivalenten Knotenkr\"{a}fte aus der Vorverformung, wenn man die {\em geometrische Steifigkeitsmatrix\/} (\ref{IIN}) mit den Knotenwerten $\vek s$ der Vorverformung multipliziert
\begin{align}
\vek f_s = \vek K_G\,\vek s\,.
\end{align}
Mit Vorverformungen l\"{o}st man also das System $(\vek K - P \cdot \vek K_G)\,\vek u = \vek f + \vek f_s$, \cite{HaJa2}.



%
Der Tragwerksplaner versteht die finiten Elemente als ein Baukasten, wo er aus verschiedenen Elementen, Stabelementen, Plattenelementen und Scheibenelementen das Tragwerk nachgebaut, die Belastung in die Knoten reduziert und an diesem Ersatzmodell die Schnittgr\"{o}{\ss}en und Verformungen ermittelt.

Nach unseren Ausf\"{u}hrungen sollte es klar sein, dass das ein Modell als ob ist. Man kann so einem Laien sehr schnell die finiten Elemente erkl\"{a}ren, aber in Wirklichkeit sind die Dinge doch etwas komplizierter.

Sie finiten Elemente sind ein Energieverfahren und wesentlich f\"{u}r den Erfolg des Verfahrens sind die Steifigkeit und die Abmessungen weil \"{u}ber diese Gr\"{o}{\ss}en das Spiel von Kraft und Verformungen l\"{a}uft.

Wie die Diskussion \"{u}ber die Plattenbalken belegt durchdringen sie teilweise die Elemente haben nur in den Knoten gleiche Verformungen ist also das FE-Modell weit davon entfernt ein geschrumpftes Modell des Originaltragwerks zu sein.

Man kann die finiten Elemente als ein Rechnen im \glq Energieraum\grq{} betrachten. Die Steifigkeiten, die Energie die sich in einem Bauteil aufbaut, wenn man es verformt, sind die entscheidenden Ma{\ss}gr\"{o}{\ss}en. Es ist schwer, dieses Rechnen im Energieraum anschaulich zu fassen. Man kann den Tragwerksplanung nur den Rat geben, dass er auf die Steifigkeit achten sollte.

Wenn man einen Tragwerksplaner zusieht, dann wundert man sich manchmal, mit welcher Raffinesse dabei ans Werk gegangen wird wie genau Details teilweise modelliert werden, weil der Tragwerks Planer in Sorge ist das eine \"{u}bersehene Ecke die Ergebnisse verf\"{a}lschen k\"{o}nnten.

Die Methode der finiten Elemente ist ein Energieverfahren und Bewegung, Verformung erzeugt Energie.
Es ist die Kette
\begin{align}
Element \quad \to \quad Ecken \quad \to \quad u \quad \to \quad  Energie
\end{align}
Das Bauteil, das Element hat Ecken, die Ecken k\"{o}nnen sich verschieben, $u$, und dadurch entsteht Energie.

Unter dem Druck der Lasten verformt sich ein Tragwerk, leisten die \"{a}u{\ss}eren Kr\"{a}fte also einer Arbeit wie umgekehrt durch die Verformungen des Tragwerks Verzerrungen in dem Tragwerk entstehen d.h. es werden innere Energie aufgebaut. Der Gleichgewichtspunkt ist der Punkt, an dem die \"{a}u{\ss}ere und innere Energie gleich gro{\ss} sind.

Der Druck der Lasten erzeugt Verformungen, das Tragwerk gibt nach und so leisten die Lasten Arbeiten, die \"{u}ber die Verzerrungen als innere Energie in dem Tragwerk gespeichert wird.

Die Methode der finiten Elemente ist ein Energieverfahren und Bewegung, Verformung erzeugt Energie.
Es ist die Kette
\begin{align}
Element \quad \to \quad Ecken \quad \to \quad u \quad \to \quad  Energie
\end{align}
Das Bauteil, das Element hat Ecken, die Ecken k\"{o}nnen sich verschieben, $u$, und dadurch entsteht Energie.

\colorbox{highlightBlue}{
\parbox{\linewidth\fboxsep}{
\begin{align}
u_h(x) = \int_{0}^{l}G_h(y,x)\,p(y)\,dy\,.
\end{align}}
}}



\definecolor{shadecolor}{rgb}{.85,.964,1.0}
\begin{shaded}
{\parbox{0.98\textwidth}{
\begin{align}
u_h(x) = \int_{0}^{l}G_h(y,x)\,p(y)\,dy\,.
\end{align}
}}
\end{shaded}


{\parbox{0.95\textwidth-15}{
\begin{shaded}
\begin{align}
u_h(x) = \int_{0}^{l}G_h(y,x)\,p(y)\,dy\,.
\end{align}
\end{shaded}}

\newif\ifpdf
    \ifx\pdfoutput\undefined
    \pdffalse % we are not running pdflatex
    \else
    \pdfoutput=1 % we are running pdflatex
    \pdfcompresslevel=9     % compression level for text and image;
    \pdftrue
    \fi
%\usepackage{ifpdf}
%\ifpdf
\newcommand{\Fpath}{d:/astatikbilder/pdf}
%\newcommand{\Fpath}{C:/Users/Friedel/OneDrive/Dokumente/ASTATIK3/Pdf}
%\else
%\newcommand{\Fpath}{d:/astatikbilder/pictures}
%\newcommand{\Fpath}{C:/Users/Friedel/OneDrive/Dokumente/ASTATIK3/Pictures}
%\fi

%-------------------------------------------------------------------------
\begin{figure}[tbp]
\centering
\if \bild 2 \sidecaption \fi
\includegraphics[width=0.7\textwidth]{\Fpath/QWANDD}
\caption{Deckenplatte auf Mauerwerksw\"{a}nden {\bf a)} Querkraft $q_x$ in einem
horizontalen Schnitt und Querkraft $q_y$ in einem vertikalen Schnitt bis zur Wand,{\bf
b)} Querkraft $q_x$ in 3D Darstellung, {\bf c)} Plot der Querkr\"{a}fte in 20 cm Abstand von den W\"{a}nden} \label{QWand}
\end{figure}%%
%-------------------------------------------------------------------------


%-----------------------------------------------------------------
\begin{figure}[tbp]
\if \bild 2 \sidecaption[t] \fi
\centering
\includegraphics[width=0.9\textwidth]{\Fpath/U456}
\caption{Kopplung Scheibe---Balken, \cite{Werkle3} }
\label{U456}
\end{figure}%
%-----------------------------------------------------------------


Die Transformationsmatrix $\vek A = \vek P^T\,\vek Q^T$ nach (\ref{Eq990}) erh\"{a}lt man damit
zwischen den Weggr\"{o}{\ss}en $\vek u_B = \{u_B, v_B, \tan\,\Np\}^T$ auf der Seite des Balkens und den Weggr\"{o}{\ss}en $\vek u_S = \{u_1, v_1, u_2, v_2, u_3, v_3\}^T$ auf der Seite der Scheibe.

Das eigentliche Ziel, die $3 \times 3$ Steifigkeitsmatrix des Knotens, ergibt sich schlie{\ss}lich, wenn man den $3 \times 3$-Block der $6 \times 6$-Balkenmatrix, der der Scheibe gegen\"{u}ber liegt, nennen wir diesen Block $\vek K_B'$ mit $\vek A$ und der transponierten Matrix $\vek A^T$ multipliziert
\begin{align}\label{Eq1700}
\vek K_{EST} = \vek A^T\,\vek K_B'\,\vek A \,.
\end{align}
Die Matrix $\vek K_{EST}$ ist die Steifigkeitsmatrix des sogenannten {\em EST-Elements\/}, das die Kopplung zwischen Balken und Scheibe beschreibt. Die Eintr\"{a}ge in $\vek K_{EST}$ werden, wie andere Elementmatrizen auch, an entsprechender Stelle in die Gesamtsteifigkeitsmatrix der Scheibe eingebaut.

Bei dem Beispiel in Abb. \ref{U548} ergibt sich die Beziehung (\ref{Eq990}) z.B.  zu
\begin{align}
\left [\barr{c}  u_B \\  v_B \\\Np \earr \right ] =  \left [\barr{r @{\hspace{4mm}}r @{\hspace{4mm}}r
@{\hspace{4mm}}r @{\hspace{4mm}}r @{\hspace{4mm}}r}  1/4 & 0 & 1/2 & 0 & 1/4 & 0 \\ 0 & 1/8 & 0 & 0 & 3/4 & 1/8\\ 1/d & 0 & 0 & 0 & -1/d & 0 \earr \right ] \,\left [\barr{c} u_1 \\ v_1 \\ u_2\\ v_2\\ u_3\\ v_3 \earr \right ]\,.
\end{align}

Man kann so auch St\"{u}tzen an Platten koppeln. In Kapitel 5 gehen wir n\"{a}her darauf ein. Wieder indem man erst die aus den Schnittgr\"{o}{\ss}en $N, V_x, V_y, M_x, M_y$ resultierenden Spannungen in \"{a}quivalente Knotenkr\"{a}fte $f_i$ auf der Plattenseite umrechnet, so auf die \glq Inzidenzmatrix\grq\ $\vek A^T$ gef\"{u}hrt wird und dann -- sinngem\"{a}{\ss} wie in (\ref{Eq1700}) -- die Matrix $\vek K_{EST}$ bildet. In Kapitel 5 gehen wir n\"{a}her darauf ein.

{\textcolor{sectionTitleBlue}{\subsubsection*{Statik ist nicht statisch, sondern kinematisch}}}


Das FE-Programm w\"{a}hlt die Knotenverschiebungen $u_i $ so aus, dass der FE-Lastfall $p_h$ arbeits\"{a}quivalent ist zu dem urspr\"{u}nglichen Lastfall $p $. Das garantiert, dass der Fehler in den Spannungen m\"{o}glichst klein wird.

\begin{itemize}

\item{Die FEM ist ein Projektionsverfahren.}

\end{itemize}

wie z.B. ein Test mit der Funktion $\Np_1(x)$ zeigt
\begin{align}
(EI \frac{d^4}{dx^4} + P \frac{d^2}{dx^2}) \,\Np_1(x) = P\,(\frac{12\,x}{l^3} -
\frac{6}{l^2})\,,\qquad \Np_1(x) = 1 - \frac{3x^2}{l^2} + \frac{2x^3}{l^3}\,,
\end{align}

%----------------------------------------------------------------------------------------------------------
\begin{figure}[tbp] \centering
\if \bild 2 \sidecaption \fi
\includegraphics[width=.6\textwidth]{\Fpath/KIPPEN1}
\caption{Kippgef\"{a}hrdete St\"{a}be}
\label{Kippen1}%
\end{figure}%
%----------------------------------------------------------------------------------------------------------

{\small {\em Beispiel\/} Die Ermittlung der kritischen Last f\"{u}r den starren,
drehelastisch gelagerten Stab in Abb. \ref{Kippen1} a f\"{u}hrt sinngem\"{a}{\ss} auf das gleiche
Problem: Das Momentengleichgewicht $P\,e = e/h\,k_\varphi $ in der ausgelenkten Lage $e$
verlangt
\begin{align}
(k_\varphi - P\,h)\,e = 0\,,
\end{align}
und die kritische Last $P_{krit}$ ist somit $P_{krit} = k_\varphi /h$, denn dann ist
jede Auslenkung $e$ eine L\"{o}sung der Momentenbedingung $0 \times \,e = 0$.

Bei Systemen mit einem Freiheitsgrad muss man also $P$ so einstellen, dass die runde
Klammer null wird; bei Systemen mit mehreren Freiheitsgraden muss man $P$ so einstellen,
dass die Determinante des Gleichungssystems null wird. Abb. \ref{Kippen1}b illustriert
das Prinzip f\"{u}r einen federelastisch gest\"{u}tzten Stab. Hier lautet die
Knickbedingung $(k - P/h)\,w = 0$, und die kritische Knicklast ist somit $P = k\,h$, wenn
$k$ die Steifigkeit der seitlichen St\"{u}tzung ist.
}\\ % ENDE SMALL

Verschiedene Modellierungen ($b$ bis $f$) eines HEB 300 in einem Kastenquerschnitt mit mitwirkenden Bereichen konstanter Schubspannungen 